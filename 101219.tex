\subsubsection{Schwarze Löcher im Zentrum von Galaxien}
\begin{itemize}
	\item Was ist ein schwarzes Loch?
		\begin{itemize}
			\item[Laplaxe (1795)] Fluchtgeschwindigkeit eines Objektes von der Oberfläche eines Himmelskörpers der Masse $M$ und Radius $R$:
				\begin{equation*}
					v_\text{Flucht}=\sqrt{\frac{2GM}{R}}
				\end{equation*}
				Bei genügend kleinem Radius $v_{Flucht}=c=$ Lichtgeschwindigkeit
		\end{itemize}
	\item Schwarzschildradius: $R=\frac{2GM}{c^2}\leadsto 3\cdot 10^4\left(\frac{M}{M_\odot}\right)\si{\cm}$\\
		z.B. für Sonne: $3\cdot 10^5\si{cm} (\SI{3}{\km})$\\
		Definition: Ein Schwarzes Loch hat einen Radius von $R<R_{BH}$
		\begin{itemize}
			\item[] Nachweis schwarzer Löcher?
			\item[] indirekt über Detektion kompakter Massenkontraktion
		\end{itemize}
		$M_0$ mit Geschwindigkeitsdispersion $\sigma$\\
		Charakteristische Rotationsgeschwindigkeit im Abstand $R$:
		\begin{equation*}
			V\sim\sqrt{\frac{G\cdot M_0}{R}} \text{ für Abstände } R\leq R_{BH}:=\frac{GM}{\sigma^2}\sim\num{0.4}\left(\frac{M_0}{10^6M_\odot}\right)\left(\frac{\sigma}{\SI{100}{\frac{\km}{\s}}}\right)^{-2}\si{\ps}
		\end{equation*}
		wird der Einfluss des SMBH auf die Kinematic der Systeme bemerkbar
		\begin{itemize}
			\item zugehöriger Winkel $=\theta_{BH}=\frac{R_{BH}}{D}\sim 0.1\left(\frac{M_0}{10^6M_\odot}\right)\left(\frac{\sigma}{\SI{100}{\frac{\km}{\s}}}\right)^{-2}\left(\frac{D}{\si{M\ps}}\right)^{-1}$
			\item $\sigma$ steigt an wie $\sqrt{R}$ für $R\leq R_{BH}$
		\end{itemize}
		Empirisch findet man:
		\begin{equation*}
			M=\num{1.2e8}M_\odot\left(\frac{\sigma}{\SI{200}{\frac{\km}{\s}}}\right)^{\num{3.75}}
		\end{equation*}
		kann für relativ nahe Galaxien detektiert werden
\end{itemize}
\subsection{Extragalaktische Entfernungsbestimmungen}
\begin{itemize}
	\item $\exists$ Methoden für verschiedene Längenskalen
	\item 1. Schritt: Bestimmung der Entfernung zur großen Magellanschen Wolke (LMC)
	\item Supernova SN1987A $\Rightarrow $ beleuchtet einen elliptischen Ring\\
		Ring der aus Material stammt, das vom Vorläuferstern der SN herausgeschleudert wurde\\
		Elliptizität aufgrund geometrische Projektion (intrinsisch Kreisförmig)\\
		Aufleuchten des Rings nicht gleichförmig, da uns das Licht von dem Teil des Rings, der näher ans uns ist, früher erreicht $\Rightarrow \frac{\text{Licht auf Verzögerung}}{\text{Licht an Verzögerung}} \& $ Inklinationswinkel des Rings $\Rightarrow $ Durchmesser des Rings $\Rightarrow$ Abstand $D_{LMC}=\SI{51.8}{k\ps}\pm\SI{6}{\%}$\\
		Aktuell (2019) $\num{49.59}\pm\num{0.09} (tot)\pm\num{0.54}(system)\si{k\ps}(\pm\SI{0.8}{\%})$
	\item Eichung der Perioden-Leuchtkraft-Relation der Cepheiden (\num{28.6}) mit Hilfe der LMC, zur Entfernungsmessung weiter entfernter Galaxien
	\item sekundäre Entfernungsindikatoren (Entfernungsindikatoren Entfernungsverhältnis z.B. über Skalengesetz Tully-Fischer, Farber-Jackson)
\end{itemize}
\section{Homogene Iostrope Weltmodelle}
\begin{itemize}
	\item \textbf{Ziel}: Verständnis des Universums auf großen Skalen
	\item Schwierigkeiten:
		\begin{itemize}[label={$\cdot$}]
			\item $\exists$ nur ein Universum
			\item große Entfernung $\Rightarrow$ lichtschwache Quellen
			\item[$\Rightarrow$] Erkenntnisgewinn durch Entwicklung großer Teleskope ($\diameter>\SI{8}{\m}$) 
		\end{itemize}
	\item \textbf{wichtigster Aspekt} für Beobachtungen: endliche Lichtgeschwindigkeit $\Rightarrow$ Beobachtungen von entfernten Objekten erlauben in die Vergangenheit zu schauen!
\end{itemize}
\subsection{Grundlegende Beobachtungen auf kosmologischen Skalen}
\begin{enumerate}[label={$(\Roman*)$}]
	\item Nachts ist es dunkel (Olders Paradoxon)
	\item Lichtschwache (weiter entfernte) Galaxien sind am Himmel Gleichförmig verteilt
	\item Spektrallinien in Spektren von Galaxien zeigen eine systematischen Verschiebung
		\begin{equation*}
			z:=\frac{\lambda_B-\lambda_0}{\lambda_0}\quad \lambda :=\text{ Wellenlänge der Spektrallinie im Ruhesystem (Laborsystem)}
		\end{equation*}
		$\lambda_B:=$ von Beobachtern gemessene Wellenlänge\\
		$\Rightarrow \lambda_B=(1+z)\lambda_0$ i.d. $z>0\rightarrow $ Rotverschiebung (Ausnahme: Gleichgewicht nahe Galaxien, M31) f+r kleinere $z$ gilt für die Relativgeschwindigkeit $v\approx z\cdot c$\\
		Hubble-Gesetz: $v=\underset{\underset{\text{Hubble-Konstante}}{\uparrow}}{H_0}\cdot D$ zur Galaxie\\
		$\Rightarrow D\approx 1000\cdot z h^{-1}\si{M\ps}$ (falls $z<<1$)
	\item In fast allen Kosmischen Objekten beträgt der Massenanteil von Helium etwa $\num{25}-\SI{30}{\%}$
	\item Die ältesten Sternhaufen in unserer Galaxis haben ein Alter von $\sim\SI{12e9}{a}$
		\begin{itemize}
			\item mit Hilfe des Hertzsprung-Russel-Diagramms
				\begin{figure}[H]
					\begin{tikzpicture}
						\def\k{2}
						\draw[<->] (0,4)node[above left]{$v$}--(0,0)--(6,0)node[below right]{$B-V$};
						\draw[domain=-{sqrt(4/\k)}:{sqrt(4/\k)},samples=50,densely dashed] plot({\k*(\x)^2+1.5},{\x+1.5});
					\end{tikzpicture}
				\end{figure}
				Sterne der Masse, für die das Lebensalter der Hauptreihe gleich dem Alter des Sternhaufens ist $\to$ Alter des Sternhaufens durch Vergleich mit theoretischem Modell der Sternentwicklung
		\end{itemize}
	\item $\exists$ eine Mikrowellenstrahlung (kosmischen Mikrowellenhintergrundstrahlung, imB) isolation bis auf fluktuation der relativen Stärke $\sim 10^{-5}$
	\item Spektrum dieser Hintergrundstrahlung entspricht einer perfekten Schwarzkörperstrahlung mit Temperatur $T=\num{2.728}\pm\SI{0.004}{\K}$
	\item Die Anzahl dicht von Radioquellen mit Fluss folgt mit den einfachen Gesetzen $N(>s)\propto s^{-\frac{3}{2}}$ (Beobachtet bei hoher galaktischen Breite) um Quellen der Milchstraße auszuschließen
\end{enumerate}
\textbf{\underline{Ziel}}: Entwicklung eines kosmologischen Modells, das diese Beobachtungen erklärt
\subsection{Ist das Universum unendlich, euklidisch und statisch?}
\begin{itemize}
	\item Naive Annahme eines
		\begin{itemize}[label={\textbullet}]
			\item räumliche unendlichen
			\item euklidischen
			\item statischen
		\end{itemize}
		Universums ist im Widerspruch zu (I) und (VIII)
\end{itemize}
Zu (I) Olders-Paradoxon:
\begin{itemize}[label={\textbullet}]
	\item[] Der Nachthimmel in solch einem Universum wäre (ungemütlich) hell!
	\item Betrachte dazu:
		\begin{itemize}[label={}]
			\item $n_\ast$: mittlere Anzahldichte der Sterne
			\item $R_\ast$: mittlerer Radius eines Sterns
		\end{itemize}
	\item Eien Kugelschale mit Radius $r$ und Dicke $dr$ um die $\odot$ enthält $4\pi r^2n_\ast dr$ Sterne, jeder mit Raumwinkel $\frac{\pi R^2_\ast}{r^2}\Rightarrow $ gesamter von $\ast$ eingenommener Raumwinkel:\\
	$d\omega = 4\pi r^2drn_\ast\frac{\pi R_\ast^2}{r^2}=4\pi^2n_\ast R_\ast^2$ dr unabhängig von $r$ $\Rightarrow $ im gesamten Universums
		$\omega=\int\limits_0^\infty dr\frac{d\omega}{dr}=4\pi^2n_\ast R_\ast^2\int\limits_0^\infty dr=\infty$ !
	\item Offensichtlich haben wir den Effekt von sich überlappenden Sternscheiben an der Sphäre nicht berücksichtigt
		\begin{itemize}
			\item Jedoch zeigt diese Betrachtung, dass der Himmel von Sternscheiben vollständig gefüllt wäre
			\item Der Himmel wäre so hell wie die Oberfläche eines typischen Sterns (z.B. die Sonne)
		\end{itemize}
\end{itemize}
zu (VIII)
\begin{itemize}[label={\textbullet}]
	\item Sei $n(>L)$ die räumliche Anzahldichte von Radioquellen mit Leuchtkräften $>L$.
	\item eine Kugelschale mit Radius $r$ und Dicke $dr$ um die $\odot$ enthält wiederum $4\pi r^2 drn(>L)$ Quellen
	\item $L=4\pi r^2\cdot S$, mit $S$: beobachteter Fluss
		\begin{itemize}
			\item $dN(>S)=4\pi r^2 dr n(>(4\pi r^2S))$
			\item $N(>S)=\int\limits_0^\infty dr 4\pi r^2n(>(4\pi r^2S))\underset{\underset{r=\sqrt{\frac{L}{4\pi S}}}{\uparrow}}{=}\int\limits_0^\infty\frac{dL}{2\sqrt{4\pi LS}}\frac{L}{4\pi S}n(>L)$\\
				$=\frac{1}{16\pi^\frac{3}{2}}S^{-\frac{3}{2}}\int\limits_0^\infty DL\sqrt{L}n(>L)\propto S^{-\frac{3}{2}}$
		\end{itemize}
		\begin{itemize}
			\item Wenigstens eine der drei Ausganshypothesen ist falsch.\\
				Rotverschiebung der Galaxien/Hubble-Gesetz $\Rightarrow $ nicht-statisches Universum
		\end{itemize}
\end{itemize}
zu (V) $\Rightarrow $ Alter des Universums $>\SI{12e9}{\year}$
zu (II) und (IV) $\Rightarrow $ Das Universum scheint auf ausreichend großen Skalen isotrop zu sein.
\begin{itemize}
	\item Falls unser Standort im Kosmos nicht ausgezeichnet ist
		\begin{itemize}
			\item Das Universum ist auch homogen.
		\end{itemize}
\end{itemize}
\textbf{\underline{\smash{Kosmologisches Prinzip}}: Das Universum ist homogen und Isotrop}
\begin{itemize}
	\item Homogenität ist nicht direkt beobachtbar und auf kleinen Skalen hinfällig (bis zu $\sim \SI{100}{h^{-1}M\ps}$), allerdings bisher keine Hinweise auf Strukturen $>>\SI{100}{h^{-1}M\ps}$
	\item Dies ist klein im Vergleich zum Hubble-Radius (= charakteristische Größe des beobachtbaren Universums)
		\begin{equation*}
			R_H=\frac{c}{H_0}=\SI{3000}{h^{-1}M\ps}
		\end{equation*}
		\begin{itemize}
			\item $\underset{\underset{\text{1. Annäherung (später zu präzisieren)}}{\uparrow}}{\text{Homogenität}}$ und Isotropie auf Skalen von $(100-3000)\ \si{h^{-1}M\ps}$
		\end{itemize}
\end{itemize}
