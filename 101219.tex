\subsection{Ist das Universum unendlich, euklidisch und statisch?}
\begin{itemize}
	\item Naive Annahme eines
		\begin{itemize}[label={\textbullet}]
			\item räumliche unendlichen
			\item euklidischen
			\item statischen
		\end{itemize}
		Universums ist im Widerspruch zu (I) und (VIII)
\end{itemize}
Zu (I) Olders-Paradoxon:
\begin{itemize}[label={\textbullet}]
	\item[] Der Nachthimmel in solch einem Universum wäre (ungemütlich) hell!
	\item Betrachte dazu:
		\begin{itemize}[label={}]
			\item $n_\ast$: mittlere Anzahldichte der Sterne
			\item $R_\ast$: mittlerer Radius eines Sterns
		\end{itemize}
	\item Eien Kugelschale mit Radius $r$ und Dicke $dr$ um die $\odot$ enthält $4\pi r^2n_\ast dr$ Sterne, jeder mit Raumwinkel $\frac{\pi R^2_\ast}{r^2}\Rightarrow $ gesamter von $\ast$ eingenommener Raumwinkel:\\
	$d\omega = 4\pi r^2drn_\ast\frac{\pi R_\ast^2}{r^2}=4\pi^2n_\ast R_\ast^2$ dr unabhängig von $r$ $\Rightarrow $ im gesamten Universums
		$\omega=\int\limits_0^\infty dr\frac{d\omega}{dr}=4\pi^2n_\ast R_\ast^2\int\limits_0^\infty dr=\infty$ !
	\item Offensichtlich haben wir den Effekt von sich überlappenden Sternscheiben an der Sphäre nicht berücksichtigt
		\begin{itemize}
			\item Jedoch zeigt diese Betrachtung, dass der Himmel von Sternscheiben vollständig gefüllt wäre
			\item Der Himmel wäre so hell wie die Oberfläche eines typischen Sterns (z.B. die Sonne)
		\end{itemize}
\end{itemize}
zu (VIII)
\begin{itemize}[label={\textbullet}]
	\item Sei $n(>L)$ die räumliche Anzahldichte von Radioquellen mit Leuchtkräften $>L$.
	\item eine Kugelschale mit Radius $r$ und Dicke $dr$ um die $\odot$ enthält wiederum $4\pi r^2 drn(>L)$ Quellen
	\item $L=4\pi r^2\cdot S$, mit $S$: beobachteter Fluss
		\begin{itemize}
			\item $dN(>S)=4\pi r^2 dr n(>(4\pi r^2S))$
			\item $N(>S)=\int\limits_0^\infty dr 4\pi r^2n(>(4\pi r^2S))\underset{\underset{r=\sqrt{\frac{L}{4\pi S}}}{\uparrow}}{=}\int\limits_0^\infty\frac{dL}{2\sqrt{4\pi LS}}\frac{L}{4\pi S}n(>L)$\\
				$=\frac{1}{16\pi^\frac{3}{2}}S^{-\frac{3}{2}}\int\limits_0^\infty DL\sqrt{L}n(>L)\propto S^{-\frac{3}{2}}$
		\end{itemize}
		\begin{itemize}
			\item Wenigstens eine der drei Ausganshypothesen ist falsch.\\
				Rotverschiebung der Galaxien/Hubble-Gesetz $\Rightarrow $ nicht-statisches Universum
		\end{itemize}
\end{itemize}
zu (V) $\Rightarrow $ Alter des Universums $>\SI{12e9}{\year}$
zu (II) und (IV) $\Rightarrow $ Das Universum scheint auf ausreichend großen Skalen isotrop zu sein.
\begin{itemize}
	\item Falls unser Standort im Kosmos nicht ausgezeichnet ist
		\begin{itemize}
			\item Das Universum ist auch homogen.
		\end{itemize}
\end{itemize}
\textbf{\underline{\smash{Kosmologisches Prinzip}}: Das Universum ist homogen und Isotrop}
\begin{itemize}
	\item Homogenität ist nicht direkt beobachtbar und auf kleinen Skalen hinfällig (bis zu $\sim \SI{100}{h^{-1}M\ps}$), allerdings bisher keine Hinweise auf Strukturen $>>\SI{100}{h^{-1}M\ps}$
	\item Dies ist klein im Vergleich zum Hubble-Radius (= charakteristische Größe des beobachtbaren Universums)
		\begin{equation*}
			R_H=\frac{c}{H_0}=\SI{3000}{h^{-1}M\ps}
		\end{equation*}
		\begin{itemize}
			\item $\underset{\underset{\text{1. Annäherung (später zu präzisieren)}}{\uparrow}}{\text{Homogenität}}$ und Isotropie auf Skalen von $(100-3000)\ \si{h^{-1}M\ps}$
		\end{itemize}
\end{itemize}
