\begin{itemize}
	\item Falls $\rho_V\neq 0$ wird die Vakuumdichte ab einem gewissen Zeitpunkt dominant!
	\item dimensionslose Dichteparameter:
		\begin{equation*}
			\Omega_m :=\frac{\rho_{m,0}}{\rho_c},\quad\Omega_r :=\frac{\rho_{r,0}}{\rho_c},\quad\Omega_\Lambda :=\frac{\Lambda}{3H_0^2}
		\end{equation*}
		mit der kritischen Dichte $\rho_c=\frac{3H_0^2}{8\pi G}$
	\item heutige Werte:
		\begin{itemize}[label={}]
			\item Staub:
				\begin{itemize}[label={}]
					\item Galaxien (inklusive ihrer dunklen Halos): $\Omega_m\gtrsim\num{0.02}$
					\item Galaxienhaufen $\Omega_m\gtrsim\num{0.1}$
					\item Kosmologie $\Omega_m\sim\num{0.3}$
				\end{itemize}
			\item Strahlung: Photonen der CMB + Neutrinos aus dem frühen Universum $\Omega_r\sim\num{4.2e-5}\cdot \underset{\underset{\approx\num{0.72}}{\rotatebox{90}{=}}}{h^{-2}}$
			\item Vakuum: $\Omega_\Lambda\sim\num{0.7}$
		\end{itemize}
	\item Da $H(t)=\frac{\dot{a}(t)}{a(t)}$ und $\rho=\rho_{m,0}\cdot a^{-3}(t)+\rho_{r,0}\cdot a^{-4}(t)$
		\begin{itemize}
			\item $H^2(t)=H_0^2\left[a^{-4}(t)\cdot\Omega_r+a^{-3}(t)\Omega_m-a^{-2}(t)\frac{K\cdot c^2}{H_0^2}+\Omega_\Lambda\right]$
		\end{itemize}
	\item Für $t=t_0$ (heute) mit $a(t_0)=1$ ergibt sich (mit $H(t_0)=H_0$):
		\begin{equation*}
			K=\left(\frac{H_0}{c}\right)^2\cdot\big(\Omega_m+\underset{\underset{\text{(für $t=t_0$)}}{\text{vernachlässigbar}}}{\underbrace{\Omega_r}}+\Omega_\Lambda -1\big)
		\end{equation*}
		und schließlich:
		\begin{equation*}
			\left(\frac{\dot{a}}{a}\right)^2=H^2(t)=H_0^2\left(a^{-4}(t)\cdot\Omega_r+a^{-3}(t)\cdot\Omega_m+a^{-2}(t)\cdot (1-\Omega_m-\Omega_\Lambda)+\Omega_\Lambda\right)\qquad (\ast)
		\end{equation*}
\end{itemize}
