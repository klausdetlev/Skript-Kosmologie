\section{Übersicht}
\begin{itemize}
	\item[] Kosmologie=gr. $\kappa o \sigma\mu o \lambda o\gamma i \alpha$=Lehre von der Welt als \underline{Ganzes}
		\begin{itemize}
			\item[$\to$] Ursprung, Entwicklung, Struktur des \underline{Universums} (=der wahrnehmbaren Welt)
			\item[$\to$] Grenzbereich der Physik/Astronomie + Einfüsse von Religion und Philosophie
			\item[$\to$] Beispiele historischer Ideen zur Schöpfung:
				\begin{enumerate}[label={(\alph*)}]
					\item Sumerer $\sim $ 1800 v. Chr. "'Atra\underline{h}aris-Epos"'
					\item griechischie Tradition: Hesiod - "'Werke und Tage"' $\sim$ 700 v. Chr., "'Gaia"'
					\item nord-germanische Mythen: Edda
					\item altes Testament - Genesis 1, 1-9
				\end{enumerate}
				Bei aller Verschiedenheit, zwei Gemeinsamkeiten:
				\begin{enumerate}
					\item Die Welt entsteht aus dem Chaos/Nichts/Ungeformten
					\item Es gibt einen definierten Anfang
				\end{enumerate}
			\item[$\to$] \underline{Frage}: Wie alt ist die Welt?
				\begin{itemize}
					\item[] laut griechischer Mythologie: Prometheus $\sim$ 1600 v. Chr.
					\item[] traditionelle christliche Antwort:\\
						Erschaffung der Welt am Sonntag den 23. Oktober 4004 v. Chr. um 9:00 Uhr morgens (Chronologie des irischen Bischofs Ussher (1581-1656))\\
						$\Rightarrow $ $\simeq$ 6000 Jahre!?
				\end{itemize}
			\item[$\bullet$] Schwierigkeiten: viele geologische + paläontologische + achräologische Befunde weisen klar auf ein höheres Alter hin!
				\begin{itemize}
					\item älteste bekannte Schrift $\sim$ 3000 v. Chr. = 5000 BP (="'before present"')
					\item Beginn des Ackerbaus $\sim$ 7000 BP
					\item Ende der letzten Kaltzeit 10000 BP
					\item erster moderner Homo sapiens 160000 BP
					\item erste Hominimen $\sim $ 7-9 Mil Jahre
					\item Erdalter $\sim $ 4.5 Mrd Jahre
					\item älteste Sterne $\sim $ 12-13 Mrd Jahre
					\item heutige Schätzung für Weltalter $\sim $ 13.7 Mrd Jahre
				\end{itemize}
			\item[$\to$] große Bedeutung der radiometrische Altersbestimmung!
			\item[$\Rightarrow$] Notwendigkeit einer auf verifizierbaren physikalischen Argumenten aufgebauten Kosmologie!
			\item[$\to$] Einige historische Daten:
				\begin{itemize}
					\item[2. Jhd. n. Chr.] ptolemäisches geozentrisches Weltbild (C. Ptolemäus $\sim$ 100-180)
					\item[1543] Kopernikus "'De revolutionibus orbitum celestinum"'
					\item[1609/10] Erdfindung des Teleskops (Galilei) $\to$ Michstraße besteht aus Einzelsternen
					\item[1785] Herschel: erstes Bild vom Aufbau der Milchstraße (in Wahrheit zwei der Spiralarme)
					\item[1837] Bessel (Struve): erste direkte Entfernungsbestimmung eines Sterns
					\item[1916] ART
					\item[1923] erste exagalaktische Entferungen
					\item[1927] erste Urknalltheorie (Lemaître)
					\item[1929] Hubble: Rotverschiebung der Galaxie
					\item[1932/33] erste Hinweise auf dunkle Materie (Oort/Zwicky) $\to$ lange Zeit ignoriert
					\item[1948] Urknall + Elemententstehung (Alpher, Gamov, Herman) $\to$ Vorhersage der Kosmischen Hintergrundstrahlung
					\item[1964] Penzias \& Wilson: Entdeckung der Kosmischen Hintergrundstrahlung im Mikrowellenbereich (schwarze Strahlung, $T\ \sim \ \SI{3}{\K}$)
					\item[1981] Inflationsscenario (Guth)
					\item[1986] blasenartige Anordnung von Galaxienhaufen (inhomogen!)
					\item[1989-93]: genaue Vermessung des Mikrowellenhintergrundes
					\item[1998] Hinweise auf beschleunigte Expansion $\to$ "'Dunkle Energie"'
					\item[2001-10]: Satelliten COBE + WMAP
				\end{itemize}
			\item[$\to$] Energieinhalt des Universums:\\
				$\SI{4.6}{\%}$ baryonische Materie\\
				$\SI{23}{\%}$ dunkle Materie\\
				$\SI{72}{\%}$ dunkle Energie\\
			\item[$\Rightarrow$] Wir kennen nur wenige Prozente des Energieinhaltes des Universums
		\end{itemize}
\end{itemize}
\section{Astronomische Grundlagen}
Ziel: Einführung in einige simple Fakten und Grundlagen der Astronomie und Astrophysik
\begin{itemize}[label={$\to$}]
	\item Eigenschaften der Sterne werden typischerweise mithilfe der Werte für die Sonne ausgedrückt:
	\begin{itemize}
		\item Luminosität: $L_\ast \ \sim 10^{-4}-10^{4} \si{\sl}$
		\item Massen: $M_\ast \ \sim 0.05 - 100 \si{\sm}$
		\item Temperaturen $T_\ast \ \sim 10^3 - 5\cdot 10^4 \si{\K}$
	\end{itemize}
	\item[$\Rightarrow$] sehr heiße Gaskugeln
	\item Sonne:
		\begin{itemize}[label={$\bullet$}]
			\item Radius: $R_\odot = \SI{6.96e8}{\m} = \SI{6.96e10}{\cm}$\\
				nahezu Kugelförmig (Abplattung $\sim \ \num{5e-5}$
			\item Energiefluss: $L_{\odot \ tot} \simeq \SI{3.9e26}{\J\per\s}=\SI{3.9e33}{\text{erg}\per\s}$\\
				im sitbaren Spektrum: $L_{\odot \ vis} \sim \ \num{0.5}L_{\odot \ tot}$\\
				der Rest wird hauptsächlich im IR und NV abgestrahlt
			\item Masse $\si{\sm} \ \sim \SI{1.99e30}{\kg}=\SI{1.99e33}{\g}$
			\item sichtbare Teile der Sonne:
				\begin{enumerate}[label={(\alph*)}]
					\item Photosphäre: unterste Schicht der Sonnenatmosphäre emittiert das sichtbare Licht der Sonne
					\item Chromosphäre: Gasschicht zwischen Photosphäre und Korona, Dicke $\sim \ \num{10e3}-\SI{13e3}{\km}$\\
						während einer totalen Sonnenfinsternis sichtbar
					\item Korona: erstreckt sich über mehrere $R_\odot$, $T \sim \ \SI{1.5e6}{\K}$
					\item Sonnenflecken: auf der Photosphäre (kühler, recht statisch)
						\begin{itemize}[label={$\to$}]
							\item Rotationsperiode der Sonne wurde so nachgemessen: ca. $\SI{25.5}{\text{d}}$
							\item Sonnenfleckenzyklus $\sim \ 2\cdot\SI{11}{\text{a}}$\\
								(zw. $\SI{0.0}{\%}$ und $\SI{0.4}{\%}$ der gesamten Oberfläche)
						\end{itemize}
				\end{enumerate}
		\end{itemize}
	\item Sterne finden sich oft im Paar, Sternhaufen und (auf noch größerer Skala) in Galaxien
	\item Galaxien enthalten zusätzlich Gas und (Sternen)staub
\end{itemize}
