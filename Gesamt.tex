\documentclass[12 pt]{article}

\usepackage[utf8x]{inputenc}
\usepackage[T1]{fontenc}
\usepackage[ngerman]{babel}

\usepackage{amsmath,amssymb,siunitx,lmodern,enumitem,wrapfig,wasysym,setspace,stmaryrd}
%\usepackage{microtype}
\usepackage{multicol}
\usepackage{tikz}
\usetikzlibrary{calc,babel,patterns,angles,quotes}
\usepackage{hyperref}
\usepackage{float}

\DeclareSIUnit\year{\text{a}}
\DeclareSIUnit\mrd{\text{Mrd }}
\DeclareSIUnit\sm{M_{\odot}}
\DeclareSIUnit\sl{L_{\odot}}
\DeclareSIUnit\pc{\text{pc}}
\DeclareSIUnit\mag{\text{mag}}
\DeclareSIUnit\au{\text{AU}}
\DeclareSIUnit\gr{\text{\grqq}}
\DeclareSIUnit\d{\text{d}}
\DeclareSIUnit\erg{\text{erg}}

\renewcommand{\labelitemi}{$\to$}
\renewcommand{\labelitemii}{$\Rightarrow$}

\newcommand{\geqleq}{\overset{>}{\underset{<}{=}}}
\newcommand{\misotope}[2]{
	\text{\textsuperscript{#1}#2}
}
\newcommand{\isotope}[2]{
	\textsuperscript{#1}#2
}
\begin{document}
\title{Einführung in die Kosmologie}
\author{Martin Michael Müller}
\pagenumbering{gobble}
\maketitle
\clearpage
\begin{itemize}[label={}]
	\item \noindent\textbf{E-Mail-Adresse:}
	\begin{itemize}[label={}]
		\item \textbf{\underline{Martin-Michael.Mueller@univ-lorraine.fr}}
		\item \textbf{\underline{mueller5@univ-lorraine.fr}}
	\end{itemize}
	\item Termine:
	\begin{itemize}[label={-}]
		\item Dienstags 10:15 Uhr - 11:45 Uhr Vorlesung
		\item Dienstags 13:00 Uhr - 15:00 Uhr 30 min Vorlesung + Rest Übung
	\end{itemize}
	\item "`Engagement in den Übungen ist notwendig um die Klausur schreiben zu dürfen"'
	\item Es gibt eine Klausur
\end{itemize}
\clearpage
\tableofcontents
\clearpage
\pagenumbering{arabic}
\section*{Literatur}
\begin{enumerate}
	\item P. Schneider, "'Extragalaktische Astronomie \& Kosmologie"', Springer (2008)\\
	\item T.-P. Cheng, "'Relativity, Gravitation and Cosmology"', Oxford Univ. Press (2008)
\end{enumerate}
\clearpage
\section{Übersicht}
\begin{itemize}
	\item[] Kosmologie=gr. $\kappa o \sigma\mu o \lambda o\gamma i \alpha$=Lehre von der Welt als \underline{Ganzes}
		\begin{itemize}
			\item[$\to$] Ursprung, Entwicklung, Struktur des \underline{Universums} (=der wahrnehmbaren Welt)
			\item[$\to$] Grenzbereich der Physik/Astronomie + Einfüsse von Religion und Philosophie
			\item[$\to$] Beispiele historischer Ideen zur Schöpfung:
				\begin{enumerate}[label={(\alph*)}]
					\item Sumerer $\sim $ 1800 v. Chr. "'Atra\underline{h}aris-Epos"'
					\item griechischie Tradition: Hesiod - "'Werke und Tage"' $\sim$ 700 v. Chr., "'Gaia"'
					\item nord-germanische Mythen: Edda
					\item altes Testament - Genesis 1, 1-9
				\end{enumerate}
				Bei aller Verschiedenheit, zwei Gemeinsamkeiten:
				\begin{enumerate}
					\item Die Welt entsteht aus dem Chaos/Nichts/Ungeformten
					\item Es gibt einen definierten Anfang
				\end{enumerate}
			\item[$\to$] \underline{Frage}: Wie alt ist die Welt?
				\begin{itemize}
					\item[] laut griechischer Mythologie: Prometheus $\sim$ 1600 v. Chr.
					\item[] traditionelle christliche Antwort:\\
						Erschaffung der Welt am Sonntag den 23. Oktober 4004 v. Chr. um 9:00 Uhr morgens (Chronologie des irischen Bischofs Ussher (1581-1656))\\
						$\Rightarrow $ $\simeq$ 6000 Jahre!?
				\end{itemize}
			\item[$\bullet$] Schwierigkeiten: viele geologische + paläontologische + achräologische Befunde weisen klar auf ein höheres Alter hin!
				\begin{itemize}
					\item älteste bekannte Schrift $\sim$ 3000 v. Chr. = 5000 BP (="'before present"')
					\item Beginn des Ackerbaus $\sim$ 7000 BP
					\item Ende der letzten Kaltzeit 10000 BP
					\item erster moderner Homo sapiens 160000 BP
					\item erste Hominimen $\sim $ 7-9 Mil Jahre
					\item Erdalter $\sim $ 4.5 Mrd Jahre
					\item älteste Sterne $\sim $ 12-13 Mrd Jahre
					\item heutige Schätzung für Weltalter $\sim $ 13.7 Mrd Jahre
				\end{itemize}
			\item[$\to$] große Bedeutung der radiometrische Altersbestimmung!
			\item[$\Rightarrow$] Notwendigkeit einer auf verifizierbaren physikalischen Argumenten aufgebauten Kosmologie!
			\item[$\to$] Einige historische Daten:
				\begin{itemize}
					\item[2. Jhd. n. Chr.] ptolemäisches geozentrisches Weltbild (C. Ptolemäus $\sim$ 100-180)
					\item[1543] Kopernikus "'De revolutionibus orbitum celestinum"'
					\item[1609/10] Erdfindung des Teleskops (Galilei) $\to$ Michstraße besteht aus Einzelsternen
					\item[1785] Herschel: erstes Bild vom Aufbau der Milchstraße (in Wahrheit zwei der Spiralarme)
					\item[1837] Bessel (Struve): erste direkte Entfernungsbestimmung eines Sterns
					\item[1916] ART
					\item[1923] erste exagalaktische Entferungen
					\item[1927] erste Urknalltheorie (Lemaître)
					\item[1929] Hubble: Rotverschiebung der Galaxie
					\item[1932/33] erste Hinweise auf dunkle Materie (Oort/Zwicky) $\to$ lange Zeit ignoriert
					\item[1948] Urknall + Elemententstehung (Alpher, Gamov, Herman) $\to$ Vorhersage der Kosmischen Hintergrundstrahlung
					\item[1964] Penzias \& Wilson: Entdeckung der Kosmischen Hintergrundstrahlung im Mikrowellenbereich (schwarze Strahlung, $T\ \sim \ \SI{3}{\K}$)
					\item[1981] Inflationsscenario (Guth)
					\item[1986] blasenartige Anordnung von Galaxienhaufen (inhomogen!)
					\item[1989-93]: genaue Vermessung des Mikrowellenhintergrundes
					\item[1998] Hinweise auf beschleunigte Expansion $\to$ "'Dunkle Energie"'
					\item[2001-10]: Satelliten COBE + WMAP
				\end{itemize}
			\item[$\to$] Energieinhalt des Universums:\\
				$\SI{4.6}{\%}$ baryonische Materie\\
				$\SI{23}{\%}$ dunkle Materie\\
				$\SI{72}{\%}$ dunkle Energie\\
			\item[$\Rightarrow$] Wir kennen nur wenige Prozente des Energieinhaltes des Universums
		\end{itemize}
\end{itemize}
\section{Astronomische Grundlagen}
Ziel: Einführung in einige simple Fakten und Grundlagen der Astronomie und Astrophysik
\begin{itemize}[label={$\to$}]
	\item Eigenschaften der Sterne werden typischerweise mithilfe der Werte für die Sonne ausgedrückt:
	\begin{itemize}
		\item Luminosität: $L_\ast \ \sim 10^{-4}-10^{4} \si{\sl}$
		\item Massen: $M_\ast \ \sim 0.05 - 100 \si{\sm}$
		\item Temperaturen $T_\ast \ \sim 10^3 - 5\cdot 10^4 \si{\K}$
	\end{itemize}
	\item[$\Rightarrow$] sehr heiße Gaskugeln
	\item Sonne:
		\begin{itemize}[label={$\bullet$}]
			\item Radius: $R_\odot = \SI{6.96e8}{\m} = \SI{6.96e10}{\cm}$\\
				nahezu Kugelförmig (Abplattung $\sim \ \num{5e-5}$
			\item Energiefluss: $L_{\odot \ tot} \simeq \SI{3.9e26}{\J\per\s}=\SI{3.9e33}{\text{erg}\per\s}$\\
				im sitbaren Spektrum: $L_{\odot \ vis} \sim \ \num{0.5}L_{\odot \ tot}$\\
				der Rest wird hauptsächlich im IR und NV abgestrahlt
			\item Masse $\si{\sm} \ \sim \SI{1.99e30}{\kg}=\SI{1.99e33}{\g}$
			\item sichtbare Teile der Sonne:
				\begin{enumerate}[label={(\alph*)}]
					\item Photosphäre: unterste Schicht der Sonnenatmosphäre emittiert das sichtbare Licht der Sonne
					\item Chromosphäre: Gasschicht zwischen Photosphäre und Korona, Dicke $\sim \ \num{10e3}-\SI{13e3}{\km}$\\
						während einer totalen Sonnenfinsternis sichtbar
					\item Korona: erstreckt sich über mehrere $R_\odot$, $T \sim \ \SI{1.5e6}{\K}$
					\item Sonnenflecken: auf der Photosphäre (kühler, recht statisch)
						\begin{itemize}[label={$\to$}]
							\item Rotationsperiode der Sonne wurde so nachgemessen: ca. $\SI{25.5}{\text{d}}$
							\item Sonnenfleckenzyklus $\sim \ 2\cdot\SI{11}{\text{a}}$\\
								(zw. $\SI{0.0}{\%}$ und $\SI{0.4}{\%}$ der gesamten Oberfläche)
						\end{itemize}
				\end{enumerate}
		\end{itemize}
	\item Sterne finden sich oft im Paar, Sternhaufen und (auf noch größerer Skala) in Galaxien
	\item Galaxien enthalten zusätzlich Gas und (Sternen)staub
\end{itemize}

\subsection{Das elektrische Strahlungsfeld}
\begin{itemize}[label={$\to$}]
\item experimentelle Beobachtungen: Licht/elektro magnetische Strahlung ausgesandt von Sternen
\item während 1000-en von Jahren die \underline{einzige} Infomationsquelle\\
\begin{tikzpicture}
\draw (0,0)circle(1 cm);
\draw[xshift=0.3 cm,yshift=0.5 cm] (-0.2,-0.4)--(0.3,-0.4)--(0.3,0.1)--(-0.2,0.1)--cycle;
\draw[->] (0.35,0.3)--(1,1.5);
\draw[->] (0.35,0.3)--(1.2,0.6)node[midway,below]{$\vec{n}$};
\draw (0.35,0.3)--(2,0.6)(0.35,0.3)--(2,1.3)(2,0.6)--(2,1.3)--(2.7,1.3)--(2.7,0.6)--(2,0.6);
\node at (2.35,0.95){$d\omega$};
\draw (0.35,0.1)--(2,-0.5)node[right]{$dA$};
\coordinate (A) at (1,1.5);
\coordinate (B) at (0.35,0.3);
\coordinate (C) at (1.2,0.6);
\pic[draw,angle eccentricity=3] {angle=C--B--A};
\node[above right] at (0.7,0.6){$\theta$};
\node[right] at (3,1){$d\omega $: Infinitesimales Raumwinkelelement};
\end{tikzpicture}\\
$dA\cos(\theta)\hat{=}$ in Richtung der einfallenden Strahlung projezierte Fläche
\item \underline{spezifische Intensität} $I_\nu$ (=spektrale Strahlungsdichte):
\begin{align*}
dE&=I_\nu dA\cos(\theta)dtd\omega d\nu\\
\nu &= \text{ Frequenz der Strahlung}\\
E &= \text{ emittierte Energie}\\
I_\nu &\text{ entspricht der Flächenhelligkeit einer (kosmischen) Quelle}\\
[I_\nu]&=\frac{\text{erg}}{\si{\cm^2\hertz\text{ ster }\s}} \quad (1 \text{erg}=\num{10}^{-7}\ \si{\J})
\end{align*}
\item spezifischer Nettofluss:
\begin{align*}
F_\nu = \int\limits_{\Omega}d\omega I_\nu\cos(\theta)\quad ,[F_\nu]=\frac{\text{erg}}{\si{cm^2\hertz\s}}
\end{align*}
der durch das Flächenelement strömt. Typischerweise (kosmologische Quellen) $\Omega << 1\ \Rightarrow \cos(\theta)\approx 1$ (in diesem Zusammenhang wird $F_\nu$ mit $S_\nu$ bezeichnet)
\item \underline{mittlere spezifische Intensität}
\begin{align*}
J_\nu&=\frac{1}{4\pi}\int d\omega I_\nu \quad , \text{Mittelwert von } I_\nu \text{über alle Winkel}\\
&\text{bei isotropem Strahlungsfeld: } J_\nu=I_\nu
\end{align*}
\item \underline{spezifische Energiedichte}:
\begin{align*}
u_\nu &= \frac{4\pi}{c}J_\nu \qquad [u_\nu]=\frac{\text{erg}}{\si{\cm^3\hertz}}\\
\text{Energie des Strahlungsfeldes}&\text{ pro Volumenelement und Frequenzintervall}
\end{align*}
\item \underline{Gesamtenergiedichte der Strahlung}: $u=\int\limits_0^\infty d\nu u_\nu$
\end{itemize}
\subsection{Strahlungstransport}
$I_\nu=\text{const.}$ entlang der Ausbreitungsrichtung eines Lichtstrahls (falls keine Emissions- oder Absorptionsprozesse stattfinden)\\
$s=$ Länge entlang des Strahls\\
$\Rightarrow \frac{dI_\nu}{ds}=\sigma \ \Rightarrow $ Flächenhelligkeit einer Quelle ist \underline{unabhängig} von ihrer Entfernung.\\
Aber: Der beobachtbare Fluss einer Quelle hängt von ihrer Entfernung $D$ ab, weil der von der Quelle eingenommene Raumwinkel abnimmt: $F_\nu \propto\frac{1}{D^2}$
\begin{itemize}[label={$\to$}]
\item inklusive Emission \& Absorption (bzw. Streuung von Licht)
\begin{align*}
\frac{dI_\nu}{ds}=-\underset{\underset{\underset{\underset{[\kappa_\nu]=\frac{1}{\si{\cm}}}{\kappa_\nu: \text{Absorptionskoeffizient}}}{\text{Absorption}}}{\uparrow}}{\kappa_\nu}\cdot I_\nu +\underset{\underset{\underset{\underset{[j_\nu]=\frac{\text{erg}}{\si{\cm^3\s\hertz}\ \text{ster}}}{\text{Emissionskoeffizient}}}{\text{Emission}}}{\uparrow}}{j_\nu} \qquad (\ast)\quad (\text{Strahlungstransportgleichung})
\end{align*}
Absorption/Emission=echte Absorption/Emission + Streuung
\item optische Tiefe $\tau_\nu(s):=\int\limits_{s_0}^s ds \kappa_\nu (s')$
\begin{align*}
\Rightarrow d\tau_\nu &=\kappa_\nu\cdot ds, s_0: \text{ Referenzpunkt auf dem Lichtstrahl}\\
(\ast)\Rightarrow \frac{dI_\nu}{d\tau_\nu}&=-I_\nu +\mathcal{S}_\nu\qquad (\ast\ast)\\
\text{wobei: } \mathcal{S}_\nu&=\frac{j_\nu}{\kappa_\nu} \quad \text{\underline{Quellfunktion}}
\end{align*}
\item formale Lösung von ($\ast\ast$):
\begin{equation*}
I_\nu(\tau_\nu)=\underset{\underset{\text{aufgrund von Absorption}}{\text{Abfall der Intensität}}}{I_\nu(0)e^{-\tau_\nu}}+\underset{\underset{\underset{\text{darauffolgender Absorption}}{\text{Emission (inklusive}}}{\text{Energiegewinn durch}}}{\int\limits_0^{\tau_\nu}d\tau_\nu'e^{\tau_\nu'-\tau_\nu}\mathcal{S}_\nu(\tau_\nu')}
\end{equation*}
formale Lösung, weil Zustand der Materie (von der $\kappa_\nu$ und $j_\nu$ abhängen) vom Strahungsfeld selbst abhängt.
\end{itemize}
\subsection{Schwarzkörperstrahlung}
\begin{itemize}[label={$\to$}]
\item Für Materie im thermischen Gleichgewicht:
\begin{align*}
\mathcal{S}_\nu&=B_\nu(T)\\
\Leftrightarrow j_\nu &=B_\nu(T)\cdot\kappa_\nu\\
\text{\underline{Kirchhoff}}&\text{\underline{sches Gesetz}}
\end{align*}
hängt nur von der Temperatur ab (und nicht von $I_\nu$!) und der Zusammensetzung der Materie
\begin{align*}
\Rightarrow I_\nu (\tau)&=I_\nu(0)e^{-\tau_\nu}+B_\nu (T)\cdot\int\limits_0^{\tau_\nu}d\tau_\nu'e^{(\tau_\nu'-\tau_\nu)}\\
&=I_\nu(0)e^{-\tau_\nu}+B_\nu(T)\cdot(1-e^{-\tau_\nu})
\end{align*}
Für größere $\tau_\nu$ gilt: $I_\nu\approx B_\nu(T)$\\
Die Strahlung der Materie im thermischen Gleichgewicht wird durch die Funktion $B_\nu(T)$ beschrieben, wenn die optische Tiefe genügend groß ist.\\
\begin{tikzpicture}
\draw (1,0.8)--(1,0)node[below]{$T$}--(0,0)--(0,2)--(1,2)--(1,1.2);
\draw (0.8,0.85)--(2,-0.05/0.2*1.2+0.85)(0.8,1.15)--(2,0.05/0.2*1.2+1.15);
\node[right] at (3,1){Hohlraumstrahlung ($\tau_\nu =\infty$, da Wände undurchsichtig)};
\end{tikzpicture}
\begin{equation*}
B_\nu(T)=\frac{2h\nu^3}{c^2}\cdot\frac{1}{e^{\frac{h\nu}{k_bT}}-1}
\end{equation*}
mit:\\
$h=\SI{6.626e-27}{\text{erg}\cdot\s}$ Plank'sches Wirkungsquantum\\
$k_B=\SI{1.38e-16}{\frac{\text{erg}}{\K}}$ Boltzmann-Konstante\\
Schwarzkörperstrahlung: 
\begin{equation*}
(\ast\ast) \Rightarrow \text{falls } \tau_\nu\to\infty \text{ gilt } I_\nu=\mathcal{S}_\nu \begin{cases} I_\nu=B_\nu(T) \\ \text{thermische Strahlung: } \mathcal{S}_\nu=B_\nu(T)\end{cases}
\end{equation*}
\begin{tikzpicture}
\draw[->] (-1,0)--(4,0)node[below]{$\leftarrow \text{log}(\nu)$};
\draw[->] (0,0)--(0,3)node[left]{$\text{log}(B_\nu)$};
\draw (1,0.5) .. controls +(1,1) and +(-1,1) .. (3,0.5)node[right]{$T_1$};
\draw[xshift=-0.2 cm] (0.5,1) .. controls +(1.5,1.5) and +(-1.5,1.5) .. (3.5,1)node[right]{$T_2>T_1$};
\end{tikzpicture}
\item Maximum von $B_\nu$ bei $\frac{h\nu_{max}}{k_BT}=2.82$ (Wien'sches Verschiebungsgesetz)
\underline{NB}: $\nu_{max}\approx T \ \Rightarrow$ Messung der Temperatur
\item Wg. $B_\lambda(T)d\lambda =B_\nu(T)d\nu$ mit $\lambda=\frac{c}{\nu}$
\begin{equation*}
\Rightarrow B_\lambda (T)=\frac{2hc^2}{\lambda^5}\frac{1}{e^{\frac{hc}{k_B\lambda T}}-1}
\end{equation*}
Rayleigh-Jeans-Näherung (ergibt sich bereits aus klassischer Elektrodynamik):
\begin{equation*}
B_\nu (T)\underset{\frac{h\nu}{k_BT}<<1}{\approx} \frac{2}{c^5}\nu^2k_BT
\end{equation*}
Wien-Näherung: 
\begin{equation*}
B_\nu(T)\underset{\frac{h\nu}{k_BT}>>1}{\approx} \frac{2h\nu^3}{c^2}e^{-\frac{h\nu}{k_BT}}
\end{equation*}
\item Energiedichte:
\begin{equation*}
u=\frac{4\pi}{c}\int\limits_0^\infty d\nu B_\nu(T)=\underset{\approx \num{7.56e-15}\si{\frac{\text{erg}}{\cm^3\K^4}}}{\underbrace{\frac{8\pi^5k_B^4}{15c^3h^3}}}\cdot T^4
\end{equation*}
$\Rightarrow $ Fluss, der von der Oberfläche eines schwarzen Körpers ausgeht:
\begin{align*}
F=\int\limits_0^\infty d_\nu F_\nu&=\Pi\int\limits_0^\infty d\nu B_\nu(T)\\
&=\sigma\cdot T^4 \text{ mit } \sigma=cnst\\
\sigma &= \frac{2\pi^5k_B^4}{15c^2h^3}=cnst \qquad (\text{Stefan-Boltzmann-Konstante})
\end{align*}
\end{itemize}
\subsection{Das Magnitudensystem}
\begin{itemize}[label={$\to$}]
	\item die \underline{scheinbare Helligkeit}, die das Auge wahrnimmt, verhält sich in etwa logarithmisch mit dem Strahlungsstrom (vgl. Gehörsinn, Einheit Dezibel)
		\begin{itemize}[label={$\Rightarrow$}]
			\item seit der Antike Einteilung von Sternen in \underline{Größenklassen} (qualitativ)
			\item Einführung eines quantitativen (relativen) Maysystems
		\end{itemize}
	\item[\underline{Definition}] Für zwei Quellen, die die Flüsse $S_1$ und $S_2$ haben, verhalten sich die \underline{scheinbaren Magnituden}/\underline{scheinbaren Helligkeiten} der beiden Quellen $m_1$ und $m_2$ wie:
		\begin{align*}
			m_1-m_2&=-\num{2.5}\log\left(\frac{S_1}{S_2}\right)\\
			\Leftrightarrow \frac{S_1}{S_2}&=10^{-\num{0.4}(m_1-m_2)}
		\end{align*}
	\item NB:
		\begin{equation*}
			\underset{\underset{m_2=0}{\text{z.B.} m_1=1}}{\delta m=1} \Rightarrow \frac{S_1}{S_2}\approx \num{0.4} \Leftrightarrow \frac{S_2}{S_1}=\num{2.5}\Rightarrow S_2>S_1
		\end{equation*}
	\item je größer die scheinbare Helligkeit, desto schwächer (!) die Quelle.\\
		traditionelle Referenz: Wega $m=0 \ mag$\\
		heute "'Polsequenz"' $\Rightarrow \ m^{\text{Wega}}=\num{0.03} \ mag$
		\clearpage
	\item Beispiele:
		\begin{itemize}[label={}]
			\item Sonne: $-\num{26.73}\ \si{mag}$
			\item Vollmond: $-\num{12.73}\ \si{mag}$
			\item Sirius: $-\num{1.46}\ \si{mag}$
			\item Polarstern: $\num{1.97}\ \si{mag}$
			\item Uranus: $\num{5.5}\ \si{mag}$
			\item Pluto: $\num{13.9}\ \si{mag}$
		\end{itemize}
\end{itemize}
\subsection{Farben \& absolute Helligkeit}
\begin{itemize}[label={$\to$}]
	\item Sterne haben verschiedene Farben (besser mit (z. B.) Feldstecher zu beobachten)
	\item man misst die scheinbaren Magnituden für verschiedene wohldefinierte Frequenzen (mit Hilfe von Filtersystemen, die zur Beobachtung genutzt werden) und schreibt:
		\begin{itemize}
			\item[ultraviolett] $U=m_U$
			\item[blau] $B=m_B$
			\item[sichtbar] $V=m_V$
			\item[rot] $R=m_R$
			\item[infrarot] $I=m_I$
			\item[] etc.
		\end{itemize}
		Es existieren mehrer Filtersysteme $\Rightarrow $ verschiedene gebräuchliche Magnitudendefinitionen \& Referenzpunkte
	\item \underline{Absolute Helligkeit}:
		\begin{itemize}[label={$\bullet$}]
			\item Sei $L_\nu$ die spezifische Leuchtkraft einer (isotrop emittierenden) Quelle=$\frac{\text{abgestrahlte Energie}}{dt\cdot d\nu}$\\
				$\Rightarrow$ Fluss $S_\nu=\frac{L_\nu}{4\pi D^2}$, $D$: Abstand zwischen Quelle und Beobachter\\
				\underline{Definition}:\\
				Die \underline{absolute Magnitude} $\mathcal{M}$ (absolute Helligkeit) ist gleich der scheinbaren Magnitude der Quelle, wenn diese sich im Abstand von $\SI{10}{\pc}$ vom Beobachter befindet. ($\SI{1}{\pc}=1 \text{parsec}\approx \SI{3.089e18}{\cm}$)
				\begin{align*}
					L_\nu=4\pi D^2S_\nu &=4\pi (\SI{10}{\pc})^2 S_\nu^{\text{abs}}\\
					\Leftrightarrow -\num{2.5}\log\left(D^2\frac{S_\nu}{S_\nu^0}\right)&=-\num{2.5}\log\left[(\SI{10}{\pc})^2\cdot\frac{S_\nu^{\text{abs}}}{S_\nu^0}\right]\\
					\Leftrightarrow -\num{2.5}\log\left(\frac{S_\nu}{S_\nu^0}\right)-(-\num{2.5})\log\left(\frac{S_\nu^{\text{abs}}}{S_\nu^0}\right)&=-5+5\log\left(\frac{D}{\SI{1}{\pc}}\right)\\
					\Leftrightarrow m-\mathcal{M}&=5\log\left(\frac{D}{\SI{1}{\pc}}\right)-5 =: \underset{\underset{\text{Entfernungsmodul}}{\uparrow}}{\mu}
				\end{align*}
				z.B.:
				\begin{itemize}[label={}]
					\item $D=\SI{10}{\pc}\ \Leftrightarrow \ \mu =0$
					\item $D=\SI{1}{k\pc}\ \Leftrightarrow \ \mu =10$
					\item $D=\SI{1}{M\pc}\ \Leftrightarrow \ \mu =25$
				\end{itemize}
		\end{itemize}
	\item Die Gesamtleuchtkraft einer Queller:
		\begin{equation*}
			L=\int\limits_0^\infty d\nu L_\nu
		\end{equation*}
	\item[] Gesamtfluss:
		\begin{equation*}
			S=\int_0^\infty d\nu S_\nu
		\end{equation*}
		\begin{itemize}
			\item scheinbare bolometrische Helligkeit:
				\begin{equation*}
					m_{bol}=-\num{2.5}\log(S)+\underset{\text{def. über Vergleichsstärke}}{cnst}
				\end{equation*}
			\item[] absolute bolometrische Helligkeit:
				\begin{equation*}
					\mathcal{M}_{bol}=-\num{2.5}\log(L)+\underset{\text{def. über Vergleichsstärke}}{cnst}
				\end{equation*}
				z.B. mit Hilfe der Sonne:
				\begin{align*}
					m_{\odot,bol}&=-\num{26.83}\quad \text{\&} \quad\mu =-\num{31.47} \ (D=1 \text{AU}\approx \SI{1.5e13}{\cm})\\
					\Rightarrow \mathcal{M}_{\odot ,bol}-\mu &=\SI{4.74}{\mag}
				\end{align*}
		\end{itemize}
\end{itemize}

\subsection{Eigenschaften von Sternen}
\begin{itemize}
	\item Sterne: Gaskugeln im hydrostatischen zwischen Gravitation und Druck
		\begin{itemize}
			\item äußeres Erscheinungsbild ist charakterisiert durch
				\begin{itemize}[label={}]
					\item Radius $R$
					\item Temperatur $T$
					\item Masse $M$
				\end{itemize}
		\end{itemize}
	\item Falls das Sternspektrum der Sterne durch die Planck-Funktion gegeben wäre, so wäre: $L=4\pi R^2\sigma T^4$ ($L$: Leuchtkraft des Sterns)
		\begin{itemize}
			\item Definition der Effektivtemperatur $T_{eff}$ eines Sterns:
				\begin{equation*}
					\sigma T_{eff}^4:=\frac{L}{4\pi R^2}\qquad (\ast)
				\end{equation*}
				$\frac{L}{L_\odot}\propto 10^{-4}-10^{5}$ (Unterschied kommt entweder durch Variation von $R$ oder $T$
		\end{itemize}
	\item[\underline{\smash{Idee}}:] Klassifizierung der Sterne mit Hilfe ihrer absoluten Helligkeit und ihres Spektraltyps
		\begin{itemize}
			\item Hertzsprung-Russel-Diagramm (HRD)
		\end{itemize}
		\begin{figure}[H]
			\begin{tikzpicture}
				\draw[->] (-1,0)--(5,0)node[below right]{Spektralklasse};
				\draw[->] (0,-1)--(0,3)node[above left]{$M$};
				\draw (0.5,2.5)--(4.5,0.5)node[midway,above right]{Hauptreihe (ca. $\SI{90}{\%}$ aller Sterne)};
				\draw (0.7,1)--(1.3,0.4)node[midway,right]{Weiße Zwerge};
				\draw (4,2.3)--(4.5,2.9)node[midway,above left]{Rote Riesen};
			\end{tikzpicture}
		\end{figure}
		Spektralklassen: $\underset{\SI{30000}{\K}-\SI{50000}{\K}}{O}$, $\underset{\SI{10000}{\K}-\SI{28000}{\K}}{B}$, $\underset{\SI{7500}{\K}-\SI{9750}{\K}}{A}$, $\underset{\SI{6000}{\K}-\SI{7350}{\K}}{F}$, $\underset{\SI{5000}{\K}-\SI{5900}{\K}}{G}$, $\underset{\SI{3500}{\K}-\SI{4890}{\K}}{K}$, $\underset{\SI{2000}{\K}-\SI{3350}{\K}}{M}$
		\begin{itemize}[label={}]
			\item Sonne: G2
			\item Sirius: A
			\item Betelgeuze: M
		\end{itemize}
	\item \textbf{Die Eigenschaften von Sternen auf der Hauptreihe werden im wesentlichen nur von einem Parameter bestimmt: der Masse $M$ dieser Sterne!}
	\item Riesen: Sterne der gleichen Spektralklasse wie Hauptreiensterne, aber mit viel größerer Leuchtkraft $L\Rightarrow R$ viel größer (vgl. $(\ast)$)
	\item Dieser Größeneffekt ist spektroskopisch zu erkennen: Schwerebeschleunigung eines Stern auf seiner Oberfläche: $g=\frac{\gamma M}{R^2}$ hat Einfluss auf die Breite von Spektrallinien des Sternes
		\begin{itemize}
			\item Zusammenhang zwischen Linienbreite und $R$ 
			\item $L$ mit Hilfe von $(\ast)$
		\end{itemize}
	\item Basierend auf der Schärfe von Spektrallinien teilt man die Sterne in die Leuchtkraftklassen ein:
		\begin{enumerate}[label={\Roman*:}]
			\item Überriesen
			\item Helle Riesen
			\item Riesen
			\item Unterriesen
			\item Zwerge
			\item Unterzwerge
		\end{enumerate}
	\item Kennt man die Entfernung $D$ (und $L$) kann man mit Hilfe der Liniebreite $g$ ermitteln
		\begin{itemize}
			\item Masse $M$
		\end{itemize}
	\item empirischer Zusammenhang zwischen $L$ und $M$ für Hauptreihensterne:
		\begin{equation*}
			\frac{L}{L_\odot}=\left(\frac{M}{M_\odot}\right)^\frac{7}{2} \qquad (\ast\ast)
		\end{equation*}
\end{itemize}
\subsection{Sternentwicklung}
\begin{itemize}
	\item Energiequelle: thermonukleare Reaktionen
	\item einfachster Prozess:
		\begin{equation*}
			4 {}^1\text{H} \to {}^4\text{He} + \SI{26.73}{Me\V}
		\end{equation*}
	\item zwei Haupt-Raktionsketten:
		\begin{enumerate}[label={(\roman*)}]
			\item pp-Kette ($T<\SI{15e6}{\K}$)
				\begin{align*}
					{}^1\text{H}+{}^1\text{H}&\to{}^2\text{H}+e^++\nu_e+\SI{0.42}{Me\V}\\
					{}^2\text{H}+{}^1\text{H}&\to{}^3\text{He}+\gamma+\SI{5.49}{Me\V}\\
					{}^3\text{He}+{}^3\text{He}&\to {}^4\text{He}+2{}^1\text{H}+\SI{12.85}{Me\V}
				\end{align*}
				Energieerzeugungsrate $~\propto T^4$
			\item CNO-Zyklus (Bethe-Weizsäcker):
				\begin{align*}
					{}^{12}\text{C}+{}^1\text{H}&\to {}^{13}\text{N}+\gamma \to {}^{13}\text{C}+e^+\nu+\gamma\\
					{}^{13}\text{C}+{}^1\text{H}&\to {}^{14}\text{N}+\gamma+\SI{7.55}{Me\V}\\
					{}^{14}\text{N}+{}^1\text{H}&\to {}^{15}\text{O}+p \to {}^{15}\text{N}+e^++\nu+\gamma+\SI{10.05}{Me\V}\\
					{}^{15}\text{N}+{}^1\text{H}&\to {}^{12}\text{C}+{}^4\text{He}
				\end{align*}
				Energieerzeugungsrate $\propto T^{20}$
		\end{enumerate}
	\item erzeugte Energie während des zentralen Wasserstoffbrennens:
		\begin{equation*}
			\underset{\text{"`main sequence"'}}{E_{MS}}=\num{0.1}Mc^2\cdot\underset{\underset{\text{Energieerzeugung}}{\text{Effizienz der}}}{\num{0.007}}
		\end{equation*}
		\begin{itemize}
			\item Lebensdauer $t_{MS}$ eines Sterns der Hauptreihe: $E_{MS}=L\cdot t_{MS}\Leftrightarrow t_{MS}=\frac{E_{MS}}{L}=\num{8e9}\cdot \frac{\frac{M}{M_\odot}}{\frac{L}{L_\odot}} \si{\year}\overset{(\ast\ast)}{=} \num{8e9}\left(\frac{M}{M_\odot}\right)^{-\frac{5}{2}}\si{\year}$\\
				Stern mit $M\approx 100 M_\odot: $ $t_\odot\propto \underset{\underset{\text{astronomischer Zeitskala}}{\text{kurz auf}}}{\num{1}-\SI{3}{M\year}}$
		\end{itemize}
	\item \textbf{Sternentwicklung nach der Hauptreihe} in Abhängigkeit von $M$:
		\begin{enumerate}[label={(\roman*)}]
			\item $M<\num{0.7}M_\odot$: Entwicklung unbekannt, da $t_{ns}>$ Alter des Universums (befinden sich noch auf der Hauptreihe)
			\item $M<\num{2.5}M_\odot$: Helumbrennen im Kern $\underset{\text{triple-$\alpha$}}{(3{}^4\text{He}\to{}^{12}\text{C})}$ setzt ein und verläuft explosiv ("`Helium-Flash"')
				\begin{itemize}
					\item stabile GG-Konfiguration mit erhöhtem Radius $R$ $\Rightarrow $ Roter Riese oder Überriese
						\begin{figure}[H]
							\centering
							\begin{tikzpicture}
								\draw (0,0)circle(2.5 cm);
								\draw (0,0)circle(1 cm);
								\draw[fill=black] (0,0)circle(1 mm);
								\draw (0.5,-0.3)--(3,-1)node[right]{Fusion von ${}^4\text{He}$};
								\draw (1.8,0.5)--(3,1)node[right]{Fusion von ${}^1\text{H}$};
							\end{tikzpicture}
						\end{figure}
					\item[] Brennen in Form von Pulsen $\to$ Abstoßung der Hülle des Sternes $\Rightarrow$ Weißer Zwert ($M\sim \num{0.6}M_\odot$ und $R\sim\SI{5000}{k\m}$)
				\end{itemize}
			\item $\num{2.5}M_\odot < M < \num{8}M_\odot$:
				\begin{itemize}[label={$\cdot$}]
					\item Zentrale Helium-Brennzone + Fusion von ${}^1\text{H}$ in Schale
					\item Massenverlust durch Sternwind
						\begin{itemize}
							\item[$\Rightarrow$] Weißer Zwerg (falls $M_{final}<\num{1.4}M_\odot$)
						\end{itemize}
				\end{itemize}
			\item $M>8M_\odot$:
				\begin{itemize}[label={$\cdot$}]
					\item CNO-Zyklus und weitere Fusionen bis zur Erzeugung von $\text{Fe}$ im Kern
					\item Eisenkern kollabiert, falls $M_{final}>\num{1.4}M_\odot$
						\begin{itemize}
							\item[$\Rightarrow$] Supernova + Neutronenstern oder schwarzes Loch
						\end{itemize}
				\end{itemize}
		\end{enumerate}
\end{itemize}

Nun bekannte wichtige Formeln:
\begin{equation*}
\underset{\to M}{\underbrace{\underset{\Rightarrow R}{\underbrace{\sigma T_\text{eff}^4=\frac{L}{4\pi R^2} \qquad \underset{\Rightarrow L}{\underbrace{S=\frac{L}{4\pi D^2}}}}} \qquad g=\frac{GM}{R^2}}}
\end{equation*}
\begin{equation*}
\frac{L}{L_\odot}\approx\left(\frac{M}{M_\odot}\right)^\frac{7}{2}
\end{equation*}
\subsection{Enternungsbestimmungen}
\subsubsection{Trigonometrische Parallaxe}
\begin{itemize}
\item rein geometrische Methode
\begin{figure}[H]
\begin{multicols}{2}
\begin{figure}[H]
\begin{tikzpicture}
\draw (0,0)circle(2 cm)circle(0.2 cm);
\draw[<->] (-2,0)--(-0.2,0)node[midway,below]{$r=\SI{1}{\au}$};
\draw (-2,0)--(0.5,5)(-0.5,5)--(2,0);
\draw[<->,dashed] (0,0)--(0,4)node[midway,below right]{$D$};
\end{tikzpicture}
\end{figure}\columnbreak
\begin{align*}
\SI{1}{\au}&=\SI{1.496e13}{\cm}\\
\text{(astro}&\text{nomische Einheit)}\\
\text{große}&\text{ Halbachse der Erde}\\
\frac{r}{D}&=\tan(\varphi)\approx\varphi
\end{align*}
\end{multicols}
\end{figure}
\item \underline{Def}: $\SI{1}{\pc}$ ist der Abstand $D$, der bei einem Winkelunterschied von 1 Bogensekunde vorliegt.
\item Es gilt: $D=\left(\frac{\varphi}{1\text{"'}}\right)^{-1}\si{\pc}$
\begin{itemize}
\item Bessel 1837: Abstand zu "`G1Cygni"'
\item Erdbewegung der Teleskope: $\Delta p\sim \SI{0.01}{\gr} \to D\leq \SI{30}{\pc}$\\
Satellit HIPPARCOS: $\Delta p\sim \SI{0.001}{\gr} \to D\leq \SI{300}{\pc}$\\
aktuell: GAIA: $\Delta p \sim \SI{2e-4}{\gr}$
\end{itemize}
\end{itemize}
\subsubsection{Eigenbewegungen}
\begin{itemize}
	\item Sterne bewegen sich relativ zur Sonne!
		\begin{itemize}[label={$\cdot$}]
			\item radiale Komponente der Geschwindigkeit (mit Hilfe von Spektrallinien bestimmbar):
				\begin{equation*}
					v_r=\frac{\Delta \lambda}{\lambda_0}\cdot c=\frac{\lambda-\lambda_0}{\lambda_0}\cdot c
				\end{equation*}
				$\lambda$: gemessene Wellenlänge $\neq\lambda_0$ aufgrund der Dopplerverschiebung\\
				$\lambda_0$: Ruhewellenlänge des atomaren Übergangs (messbar im Labor)
		\end{itemize}
	\item[] Koncetion:
		\begin{itemize}
			\item[] $v_r>\sigma$ Bewegung von uns weg (Rotverschiebung)
			\item[] $v_r<\sigma$ Bewegung zu uns hin
		\end{itemize}
	\item tangentiale Komponente:\\
		messbar über die Eigenbewegung $\mu$ des Sterns auf der Himmelssphäre (in $\si{\gr\per\year}$)
		\begin{equation*}
			v_t=D\cdot \mu \Leftrightarrow \frac{v_t}{\si{\km\per\s}}=\num{4.74}\left(\frac{D}{\SI{1 }{\pc}}\right)\cdot\left(\frac{\mu}{\SI{1}{\gr\per\year}}\right)=\left(\frac{\si{\pc}}{\si{\gr}}\right)
		\end{equation*}
		HIPPARCOS:
		$\mu$ für ca $10^5$ Sterne
		\begin{itemize}
			\item $v_t$ sowie $D$ bekannt
			\item Datenbank mit Sterngeschwindigkeiten
			\item Struktur der Galaxis
		\end{itemize}
\end{itemize}
\subsubsection{Sternstromparallaxe}
\begin{itemize}
	\item Sterne eines offenen Sternhaufens haben alle eine sehr ähnliche Raumgeschwindigkeit $\vec{v}$
	\item Die Position des $i$-ten Sterns wird beschrieben durch:
		\begin{equation*}
			\vec{r_i}(t)=\vec{r_i}(0)+\vec{v}\cdot t
		\end{equation*}
		\begin{itemize}
			\item Richtungsvektor: $\vec{n_i}(t)=\frac{\vec{r_i}(t)}{\left|\vec{r_i}(t)\right|} \underset{t\to\infty}{\to} \frac{\vec{v}{\left|\vec{v}\right|}}=\hat{n}_\text{conv}$
		\end{itemize}
		\begin{figure}[H]
			\begin{tikzpicture}
				\node[name=u] at (0,0){};
				\node[name=s1] at (-0.5,4){};
				\node[name=s2] at (-0.5,5){};
				\node[name=k,right] at (2,4.5){$K$};
				\draw[dashed] (u)node[left]{$\vec{r_i}$}--(s2)node[midway,left]{$\hat{n_i}$}(s2)node[above]{$\vec{v_t}$}--(k)(s1)node[below]{$\vec{v_t}$}--(k)(u)--(k)node[midway,right]{$\hat{n}_\text{conv}$};
			\end{tikzpicture}
		\end{figure}
		$\Psi$ ist die Sternstromparallaxe, gegeb durch den Winkel zwischen $\hat{n}$ und $\hat{n}_\text{conv}$
	\item Es gilt: $\cos(\Psi)=\hat{n}\cdot\hat{n}_\text{conv}=\hat{n}\cdot\frac{\vec{v}}{|\vec{v}|}$
		\begin{itemize}
			\item $v_r=v\cos(\Psi),\ v_t=\sin(\Psi)$
			\item $v_t=v_r\cdot\tan(\Psi)$
		\end{itemize}
	\item Weiterhin gilt: $v_t=D\cdot\mu$
		\begin{itemize}
			\item $D=\frac{v_r\tan(\Psi)}{\mu}\ \Rightarrow$ Messung von $v_r$ und $\Psi$ und $\mu$ zwischen Beobachter und Sternhaufen
		\end{itemize}
	\item \underline{Beispiele}:
		\begin{itemize}[label={}]
			\item Hyaden: $D\approx \SI{45}{\pc}$
			\item Ursa-Major $D\approx \SI{24}{\pc}$
			\item Plejaden $D\approx \SI{130}{\pc}$
		\end{itemize}
	\item historisch bedeutsam, da Methode für Enternungen $>\SI{30}{\pc}$ (unterste Sprosse der Enternunsgleiter)
\end{itemize}
\subsubsection{Photometrische Entfernung}
\begin{itemize}
	\item \underline{Idee}: Sterne auf der Hauptreihe habe für eine gegebene Farbe die gleiche Leuchtkraft
	\item Für einen Sternhaufen (allen Sterne haben $\approx $ gleiche Entfernung $D$ von uns) kann man ein Farben-Helligkeitsdiagramm mit Hauptreihe erhalten, bei dem die scheinbare Helligkeit aufgetragen ist
	\item In einem zweiten Schritt erhält man das Entfernungsmodul ($m-M$), indem man die Haupreihe mit einer geeichten Hauptreihe (Sternhaufen in der Nähe, z. B. Hyaden) in Übereinstimmung bringt
		\begin{equation*}
			m-M=5\log\left(\frac{D}{\si{\pc}}\right)-5
		\end{equation*}
		\begin{figure}[H]
			\centering
			\begin{tikzpicture}
				\draw[->] (-1,0)--(5,0)node[below right]{$B-V=m_B-m_V$};
				\draw[->] (0,-1)--(0,5)node[above left]{$M$};
				\foreach \x \y in {-5/5,0/4,5/3,10/2,15/1}{
					\draw (0.1,\y )--(-0.1,\y )node[left]{\x };
				};
				\draw (0.5,4.5)--(4.5,0.5);
			\end{tikzpicture}
		\end{figure}
	\item \underline{Probleme}:
		\begin{itemize}[label={$\cdot$}]
			\item Stern wandert auf Hauptreihe während er altert
			\item HR-Sterne sind Zwerge der Klasse V (schwache Leuchtkraft)
			\item Extinktion\\
				Extinktion: Die Beziehung zwischen absoluter und scheinbarer Helligkeit wird durch die Absorption und Streuung des Sternenlichtes geändert.
		\end{itemize}
	\item Strahlungstransportgleichung ($\to$ 2.2) ohne Emission:
		\begin{align*}
			\frac{dI_\nu}{ds}&=-\kappa_\nu I_\nu\\
			\Rightarrow I_\nu(s)&=I_\nu(0)\cdot e^{-\tau_\nu(s)} \text{ mit } \tau_\nu(s)=\int\limits_{0}^sds'\kappa_\nu(s') \text{ optische Tiefe}
		\end{align*}
	\item $s_\nu=s_\nu(0)e^{-\tau_\nu(s)}$
	\item[$\mathbf{\to}$] Extinktionskoeffizient:
		\begin{equation*}
			A_\nu := m-m_0=-\num{2.5}\log\left(\frac{S_\nu}{S_\nu(0)}\right)=\num{2.5}\log(e)\tau_\nu=\num{1.086}\tau_\nu
		\end{equation*}
		mit: $m$ Magnitude mit Absorption und $m_0$ Magnitude ohne Absorption
		\begin{itemize}
			\item Quelle erscheint schwächer und ihre Farbe ändert sich, da Extinktion von $\nu$ abhängt (via $\kappa_\nu$) $\to$ Sterne erscheinen röter als sie sind
		\end{itemize}
	\item Beschreibung mit Hilfe des Farbexzesses (für Filter $X$ und $Y$)
		\begin{equation*}
			E(X-Y):=A_X-A_Y=(X-X_0)-(Y-Y_0)=(X-Y)-(X-Y)_0
		\end{equation*}
	\item Verhältnis $\frac{A_X}{A_Y}=\frac{\tau_{\nu ,X}}{\tau_{\nu, Y}}$ hängt nur von optischen Eigenschaften des Staubes ab.
		\begin{itemize}
			\item $E(X-Y)=A_X-A_Y=A_Y\left(\frac{A_X}{A_Y}-1\right)=A_Y\cdot\frac{1}{R_Y}$\\
				üblicherweise: $X=B$ und $Y=V$
			\item $A_V=R_VE(B-V)$\\
				z.B. Staub der Milchstraße (empirisch):
				\begin{equation*}
					A_V=(\num{3.1}\pm\num{0.1})E(B-V)
				\end{equation*}
				In der Sonnenumgebung (nnerhalb der Scheibe)
				\begin{equation*}
					A_V \sim \SI{1}{\mag}\frac{D}{\SI{1}{k\pc}}
				\end{equation*}
			\item nicht vernachlässigbar bei der photometrischen Entfernungsbestimmung von Sternhaufen
			\item Prozedur in 2 Schritten:
				\begin{enumerate}[label={$(\roman*)$}]
					\item Erstelle Zweifarben-Diagramm des Sternhaufens
						\begin{figure}[H]
							\begin{tikzpicture}
							\end{tikzpicture}
						\end{figure}
						\begin{itemize}[label={$\to$}]
							\item Verschiebung der HR des Sternhaufens entlang des Verfärbungsvektors bis zur Übereinstimmung der geeichten HR (ohne Absorption)
								\begin{itemize}[label={$\Rightarrow$}]
									\item $E(B-V)\Rightarrow A_V=\num{3.1}E(B-V)$
								\end{itemize}
						\end{itemize}
					\item Bestimmung des Entfernungsmoduls durch vertikale Verschiebung der Hauptreihe um Farben-Helligkeits-Diagramm bis zur Übereinstimmung mit einer geeichten HR.
						\begin{equation*}
							m-M=5\log\left(\frac{D}{\SI{1}{\pc}}\right)-5+\underset{\underset{(m-m_0)}{\uparrow}}{A_V}
						\end{equation*}
				\end{enumerate}
		\end{itemize}
\end{itemize}
\subsubsection{Visuelle Doppelsterne}
\begin{figure}[H]
	\begin{multicols}{2}
		\begin{figure}[H]
			\centering
			\begin{tikzpicture}
				\node at (0,0.8){$\Theta$};
				\coordinate (B) at (0,0);
				\coordinate (A) at (-0.5,1);
				\coordinate (C) at (0.5,1);
				\draw (B)--(A);
				\draw (B)--(C);
				\pic[draw,angle eccentricity=3] {angle=C--B--A};
			\end{tikzpicture}
		\end{figure}\columnbreak
		mit Massen $m_1$ und $m_2$\\
		Keplersches Gesetz: $P^2=\frac{4\pi^2a^3}{G(m_1+m_2)}$
	\end{multicols}
\end{figure}
\begin{itemize}
	\item Messung von Periode $P$ und Winkeldurchmesser der Bahn $2\Theta$ \& Bestimmung von $m_1$ und $m_2$ mit Hilfe ihrer spektralen Eigenschaften $\Rightarrow a \Rightarrow $ Abstand $D=\frac{a}{\Theta}$
\end{itemize}
\subsubsection{Entfernung pulsierender Sterne}
\begin{itemize}
	\item Verschiedene Arten pulsierender Sterne zeigen periodische Helligkeitsänderungen, wobei ihre Periode mit der Masse (und daher der Leuchtkraft) der Sterne korrelliert ist.
	\item Man findet (s. Übung)
		\begin{equation*}
			P  \sim \bar{\rho}^{-\frac{1}{2}}
		\end{equation*}
		wobei $\bar{\rho}\sim\frac{M}{R^3}$ mittlere Dichte des Sterns
	\item weiterhin gilt: $L\sim M^3$ und $L\sim R^3\cdot T_{eff}$
		\begin{itemize}
			\item $P\sim\frac{R^\frac{3}{2}}{\sqrt{M}}\sim L^\frac{7}{12}$, falls $T_{eff}=const$
		\end{itemize}
	\item[$\Rightarrow$] drei Sorten pulsierender Sterne:
		\begin{enumerate}[label={$(\roman*)$}]
			\item $\delta$-Cephei (klass. Cepheiden): junge Sterne
				\begin{equation*}
					\mathcal{M}_\nu =-3\log\left(\frac{P}{\SI{1}{\d}}\right)-\num{0.8} \ \text{(aus Experimenten)}
				\end{equation*}
				Zur Minimierung der Extinktion und Streuung Beobachtung der $P-L$-Relation in Nah-IR besonders nützlich.
			\item W Virginis Sterne (Population II, Cepheiden):\\
				massearme, $\underset{\text{schwerer als Helium}}{\text{metallarme}}$ Sterne
			\item RR Lyrae-Sterne (ebenfalls Population II)\\
				Metallarm\\
				sehr langsame Perioden $\mathcal{M}_\nu \in [\num{0.5};\num{1.0}]$ mit $\mathcal{M}_F=(-\num{2.0}\pm\num{0.3})\log\left(\frac{P}{\SI{1}{\d}}\right)+\num{0.06}\cdot\left[\frac{\text{Fe}}{\text{H}}\right]-\num{0.7}$
		\end{enumerate}
		I.a. für ein Element X: $\left[\frac{\text{X}}{\text{H}}\right]=\log\left(\frac{n(\text{X})}{n(\text{H})}\right)_\ast-\log\left(\frac{n(\text{X})}{n(\text{H})}\right)_\odot$ wobei $n(\text{X})$= Anzahl der Spezies X\\
		z.B. $\left[\frac{\text{Fe}}{\text{H}}\right]=-\num{1}\Rightarrow$ Eisen im Stern hat ein zehntel der solaren Häufigkeit\\
		Metallizität $Z$: Massenzahl aller Elemente schwerer als Helium\\
		z.B.: $Z_\odot=\num{0.02}\Rightarrow \SI{98}{\%}$ der Sonnenbasse besteht aus H und He
\end{itemize}
\underline{Endergebnis}: typische astronomische Distanzen:
\begin{itemize}[label={}]
	\item Sonne $\SI{1}{\au}\approx \SI{150e6}{\km}$ ($\SI{8}{\min}\ \SI{15}{\s}$ für ein Photon)
	\item $\alpha$Centauri $\SI{1.3}{\pc}$
	\item Dicke der Galaxie $\SI{0.3}{k\pc}$
	\item Abstand zum galaktischen Zentrum $\SI{8}{k\pc}$
	\item Radius der Galaxis $\SI{12.5}{k\pc}$
	\item nächste Galaxie $\SI{55}{k\pc}$
	\item Andromeda M31 $\SI{770}{k\pc}$
	\item Größe eines Galaxiehaufens $\num{1}-\SI{5}{M\pc}$
	\item Zentrum des nächsten Superhaufens (Virgo) $\SI{20}{M\pc}$
	\item Größe eines Superhaufens $\SI{260}{M\pc}$
	\item sichtbares Universum $\SI{4000}{M\pc}$
\end{itemize}

\section{Unsere Galaxis}
(=Milchstraße=$\lambda\alpha\lambda\alpha\xi$i$\zeta$)
\subsection{Struktur der Galaxis}
\begin{figure}[H]
	\centering
	\begin{tikzpicture}[scale=2]
		\draw[fill=black] (0,0)ellipse(0.2cm and 0.1 cm);
		\draw[dashed] (0,0)ellipse(1 cm and 0.5 cm);
		\draw (0,0)ellipse(1cm and 1cm);
		\draw (0,0)--(2,1)node[right]{Bulge};
		\draw (0.8,0.1)--(2,0)node[right]{(dicke und dünne) Scheibe};
		\draw ({0.95*cos(330)},{0.95*sin(330)})--(2,-1)node[right]{Halo};
	\end{tikzpicture}
\end{figure}
\begin{itemize}
	\item stellare Populationen:\\
		Population I (Pop I): Sterne mit Metallizität $Z\sim\num{0.02}\sim Z_\odot$ v.a. in der dünnen Scheibe\\
		Population II (Pop II): metallarm $Z\sim\num{0.001}$ v.a. in der dicken Scheibe, aber auch im Halo und im Bulge
	\item Metallizität und Alter:
		\begin{itemize}[label=$\cdot$]
			\item extrem alte Sterne: $\left[\frac{\text{Fe}}{\text{H}}\right]=-\num{4.5}$
			\item dicke Scheibe: $\left[\frac{\text{Fe}}{\text{H}}\right]=-\num{6.5}$
			\item dünne Scheibe: $\left[\frac{\text{Fe}}{\text{H}}\right]=-\num{0.5}$
			\item sehr junge Sterne: $\left[\frac{\text{Fe}}{\text{H}}\right]=\num{1}$
		\end{itemize}
	\item Hauptursache für die Metallanreicherung im interstellaren Medium: Supernovae!\\
		\underline{Supernova (SN)}: Sternenexplosion mit hoher Leuchtkraft $L\sim 10^9\cdot L_\odot$ (vergleichbar mit $L_B$ einer ganzen Galaxie)
	\item historische Klassifizierung anhand der spektralen Eigenschaften:
		\begin{itemize}
			\item[SN I] keine Balmerlinien des Wasserstoffs
				\begin{itemize}
					\item[SNIa] starkes SiII Emission, $\lambda=\SI{615}{n\m}$
					\item[SNIb,Ic] keine SiII Emission
				\end{itemize}
			\item[SN II] mit Balmerlinien
		\end{itemize}
	\item heute bekannt:
		\begin{enumerate}[label={$(\roman*)$}]
			\item SNII,SNIb,c: Sternenexplosion mit $M_\ast\gtrapprox 8M_\odot$ abgestrahlte Energie $\sim$ $\SI{3e53}{\erg}$ (Neutrinos!)\\
				1. (?) Nachweis von $\num{10}$ Neutrinos der SN1987A
				\begin{itemize}[label={$\to$}]
					\item Wechselwirkung zw. Neutrinus und Sternmaterie (hohe Dichte!)
						\begin{itemize}[label={$\Rightarrow$}]
							\item Explosion der Sternhülle mit $E_\text{kin}\sim 10^{51}\si{\erg}=\SI{1}{\text{foe}}=\SI{1}{\text{Bethe}}$
							\item $10^{49}~\si{\erg}$ umgesetzt in Photonen (nur Bruchteil der Gesamtenergie!)
						\end{itemize}
				\end{itemize}
			\item SNIa: Explosion eines weißen Zerges eines Doppelsternsystems
				\begin{figure}[H]
					\begin{multicols}{2}
						\begin{figure}[H]
							\begin{tikzpicture}[scale=0.8]
								\def\k{5}
								\draw[fill=black] (0,0)circle(0.4cm);
								\draw[fill=black] (\k,0)circle(2 cm);
								\draw[->] ({2*cos(110)+\k},{2*sin(110)}) .. controls +(-0.5,0.5) and +(0.5,0.5) .. ({0.4*cos(60)},{0.4*sin(60)});
							\end{tikzpicture}
						\end{figure}\columnbreak
						Massentransfer (Akkrektion) von seinem Begleiter, bis die Chandrasekhar-Masse $M_{Ch}\approx\SI{1.44}{\sm}$ überschritten wird $\Rightarrow$ SNIa
					\end{multicols}
				\end{figure}
				\begin{itemize}[label={$\to$}]
					\item homogene Anfangsbedingungen für SNIa mit etwa gleicher Leuchtkraft
						\begin{itemize}
							\item[$\Rightarrow$] \textbf{Standardkerzen, die weithin sichtbar sind}
						\end{itemize}
				\end{itemize}
		\end{enumerate}
		\begin{table}[H]
			\def\k{1.5}
			\begin{tabular}{p{2 cm}|p{\k cm}|p{\k cm}|p{\k cm}|p{\k cm}|p{\k cm}|p{\k cm}}
				& neutrales Gas & dünne Scheibe & dicke Scheibe & Bulge & stellarer Halo & DM Halo \\\hline
				$\frac{M}{10^{10}~\si{\sm}}$ & $\num{0.5}$ & $\num{6}$ & $\num{0.2}-\num{0.4}$ & $\num{1}$ & $\num{0.15}$ & - \\\hline
				$\frac{L_B}{10^{10}~\si{\sl}}$ & - & $\num{1.8}$ & $\num{0.02}$ & $\num{0.3}$ & $\num{0.1}$ & $\num{0}$ \\\hline
				$\frac{\frac{M}{L_B}}{\frac{M_\odot}{L_\odot}}$ & - & $\num{3}$ & $\num{10}$ & $\num{3}$ & $\sim\num{1}$ & - \\\hline
				Durch- messer ($\si{k\pc}$) & $\num{50}$ & $\num{50}$ & $\num{50}$ & $\num{2}$ & $\num{100}$ & $>\num{200}$\\\hline
				Form & $e^{-\frac{z}{h_z}}$ & $e^{-\frac{z}{h_z}}$ & $e^{-\frac{z}{h_z}}$ & Balken? & $r^{-3.5}$ & $\frac{1}{w^2+r^2}$ \\\hline
				Skalenhöhe $h_z$ ($\si{k\pc}$) & $\num{0.13}$ & $\num{0.325}$ & $\num{1.5}$ & $\num{0.4}$ & $\num{3}$ & $\num{2.8}$ \\\hline
				Geschwindig- keitsdispersion $\si{k\m\per\s}$& $\num{7}$ & $\num{20}$ & $\num{40}$ & $\num{120}$ & $\num{100}$ & - \\\hline
				$\left[\frac{\text{Fe}}{\text{H}}\right]$ & $>\num{0.1}$ & $-\num{0.5}-(+\num{0.3})$ & $(-\num{1.6})-(-\num{0.4})$ & $-\num{1}-(-\num{1})$ & $-\num{4.5}-(-\num{0.5})$ & - \\
			\end{tabular}
		\end{table}
\end{itemize}
\subsection{Kinematik der Galaxis}
\begin{figure}[H]
	\begin{tikzpicture}[scale=2]
		\draw (0,0)ellipse(2cm and 0.8cm);
		\draw[white,fill=white] (-2,0)--(2,0)--(2,1)--(-2,1)--cycle;
		\draw (-2,0)--(-2,1)(2,0)--(2,1);
		\draw (0,1)ellipse(2cm and 0.8cm);
		\coordinate (x) at (0,1);
		\node at (x){$\times$};
		\node[name=r1] at (0.9,1.5){};
		\node[name=r2] at (1.5,1.4){};
		\node[name=r3] at (1.9,1.1){};
		\draw (x)--(3,2)node[right]{galaktisches Zentrum};
		\draw[->] (r1) .. controls (r2) and (r2) .. (r3)node[midway,name=r]{};
		\draw (r)--(3,1.5)node[right]{Rotation};
		\draw[->] (x)--(0,2)node[above]{$z$};
		\draw (-1.5,0.9)circle(0.1 cm)coordinate(s);
		\draw (s)--(-0.9,1.25)coordinate(f);
		\coordinate (a) at ([yshift=0.25 cm]f);
		\node at (a){$\ast$};
		\draw (s)--(x);
		\draw (f)--(x);
		\pic[draw,"$l$",angle eccentricity=1.5] {angle=x--s--f};
		\pic[draw,"$\theta$",angle eccentricity=1.5] {angle=f--x--s};
		\draw[dashed] (a)--([yshift=0.25cm]x)(a)--(f)(a)--(s);
	\end{tikzpicture}
\end{figure}
\begin{itemize}
	\item sphärische Galaktische Koordinaten $(l,b)$ mit der Sonne als Zentrom: $l=$ galaktische Länge, $b=$ galaktische Breite $b=\SI{90}{\degree}\hat{=}$ galaktischer Nordpol (NGP)
	\item zylindrische Galaktische Koordinaten $(R,\theta,z)$ mit Geschwindkigkeitskomponenten $(U,V,W)$
	\item Körper mit der Bahnkurve $(R(t),\theta(t),z(t))$ hat die Geschwindigkeitskomponenten:
		\begin{equation*}
			U=\frac{dR}{dt};\quad V=R\cdot\frac{d\theta}{dt};\quad W=\frac{dz}{dt}
		\end{equation*}
	\item fiktives Ruhesystem: Local Standard of Rest (LSR) mit $U_\text{LSR}=0$, $V_\text{LSR}=0$, $W_\text{LSR}=0$\\
		wobei $v_0=V(R_0)$ Kreisbahngeschwindigkeit am Ort der Sonne entspricht
	\item Pekuliargeschwindigkeit (=Geschwindigkeit relativ zum LSR):
		\begin{equation*}
			\vec{v}=(u,v,w)=(U-U_\text{LSR},V-V_\text{LSR},W-W_\text{LSR})=(U,V-V_0,W)
		\end{equation*}
		$\vec{v}_\odot$: Sonnenbewegung relativ zum LSR
		\begin{itemize}
			\item $\vec{v}=\vec{v}_\odot+\underset{\text{Geschwindigkeit eines Sterns relativ zur Sonne}}{\underset{\uparrow}{\Delta\vec{v}}}$
		\end{itemize}
	\item Mittelwert der Pekuliargeschwindigkeitskomponenten:
		\begin{equation*}
			\langle u\rangle=0,\langle w\rangle=0,\langle v\rangle\neq 0
		\end{equation*}
		$\langle v\rangle = -C\cdot \langle u^2\rangle$
		\begin{equation*}
			\Rightarrow \vec{v}_\odot = (-\langle\Delta u\rangle , (-C\cdot\langle u^2\rangle -\langle\Delta v\rangle),-\langle\Delta w\rangle)
		\end{equation*}
	\item Wie kann $C$ gefunden werden? $\Rightarrow$ Messen von $\langle\Delta v\rangle$ und $\langle u^2\rangle$ von verschiedenen Sternpopulationen
		\begin{figure}[H]
			\begin{multicols}{2}
				\begin{tikzpicture}
					\draw[->] (-0.1,0)--(2,0)node[below right]{$\langle u^2\rangle$};
					\draw[->] (0,-0.1)--(0,2)node[above left]{$\langle\Delta v\rangle$};
					\draw (-0.1,1.5)--(0.1,1.5)node[right]{0};
					\draw (0,1)--(1.5,0.25);
					\draw[<->,xshift=-0.1 cm] (0,1)--(0,1.5)node[midway,left]{$\vec{v}_\odot$};
				\end{tikzpicture}\columnbreak
				\begin{equation*}
					\vec{v}_\odot=(-10,7,5)\si{\km\per\s}
				\end{equation*}
			\end{multicols}
		\end{figure}
	\item Asymmetrischer Drift:
		\begin{figure}[H]
			\begin{tikzpicture}
				\draw[->] (-1,0)--(6,0)node[below right]{$\frac{v}{\si{\frac{\km}{\s}}}$};
				\draw[->] (0,-1)--(0,6)node[above left]{$\frac{v}{\si{\frac{\km}{\s}}}$};
				\draw[fill=black] (5,3)circle(0.1cm)node[name=lsr,above]{LSR};
				\draw (5,3)circle(0.7 cm);
				\draw[->] (7,2)node[right]{junge metallreiche A-Zwerge} .. controls +(-0.5,0) and +(0.5,-0.5) .. ({5+0.5*cos(300)},{3+0.5*sin(300)});
				\draw (5,3)circle(1cm);
				\draw[->] (7,1.5)node[right]{K-Riesen mittleren Alters} .. controls +(-0.5,0) and +(0.5,-0.5) .. ({5+0.8*cos(290)},{3+0.8*sin(290)});
				\draw (4,5) .. controls (1,5) and (1,5) .. (1,3) .. controls (1,1) and (1,1) .. (4,1) .. controls (6.5,1) and (6.5,1) .. (6.5,3) .. controls (6.5,5) and (6.5,5) .. cycle;
				\node at (2.5,4){alte,};
				\node at (2.5,3.65){metallarme};
				\node at (2.5,3.3){Sterne};
				\draw[dashed] (5,0) .. controls +(3,1) and +(4,-1) .. (4,6.5);
				\node[name=x] at (0.5,3){$\times$};
				\draw[->] (7,1)node[right]{mittelpunkt der Einhüllenden} .. controls +(-3,0) and +(0.5,-4.5) .. (x);
				\node at (2,5.8){Halosterne};
			\end{tikzpicture}
		\end{figure}
		\begin{itemize}[label={$\cdot$}]
			\item $(u,v)$-Verteilung junger Sterne eng um $u=v=0$, die für ältere Sterne breiter wird (Dispersion wg. Gravitationswechselwirkungen)
			\item $v\approx -\SI{220}{\frac{\km}{\s}}$ Mittelpunkt der kreisförmigen Einhüllenden der Halopopulation (mit der älteren Sterne)
			\item Annahme: Halo rotiert nicht (oder nur langsam) $\Rightarrow V_0=V(R_0)=\SI{220}{\frac{\km}{\s}}$
		\end{itemize}
	\item GG-Bedingung für eine Kreisbahn: Zentrifugalkraft=Gravitationskraft
		\begin{align*}
			\Leftrightarrow \frac{mV^2}{R}&=G\frac{mM}{R^2} \text{ mit } M=M(R)=\text{ Masse im Inneren der Kugelschale}\\
			\Rightarrow M(R_0)&=\frac{V_0^2R_0}{G}=\SI{8.8e10}{\sm}
		\end{align*}
		\begin{itemize}
			\item Umlaufzeit des LSR um die Galaxis:
				\begin{equation*}
					P=\frac{2\pi R_0}{V}=\SI{230e6}{a}
				\end{equation*}
		\end{itemize}
\end{itemize}

\begin{enumerate}[label={$\arabic*.$}]
	\item Scheibe: $n(R,z)=n_0\cdot\left(e^{-\frac{|z|}{h_\text{thin}}}+\num{0.02}\cdot e^{-\frac{|z|}{h_\text{thick}}}\right)\cdot e^{-\frac{R}{h_R}}$\\
		$h_R=\SI{3.5}{k\ps}$\\
		$h_\text{thin}=\SI{325}{\ps},\ h_\text{thick}=\SI{1.5}{k\ps}$
	\item Bulge:\\
		Skalenhöhe ($\propto e^{-\frac{|z|}{h_z}}$): $h_z=\SI{0.4}{k\ps}$\\
		$I(R)=I_e\cdot e^{-\num{7.669}\cdot\left(\left(\frac{R}{R_e}\right)^\frac{1}{4}-1\right)}$ de Vancouleurs-Profil\\
		$R_e$: Effektivradius, innerhalb dessen die Hälfte der Leuchtkraft emittiert wird.
	\item Stellarer Halo:\\
		$n(r)\propto r^{-\num{3.5}}$ Dichteverteilung\\
		mit de Vancouleurs-Profil: $r_e\approx \SI{3}{k\ps}$
	\item DM Halo: \\
		quasi-isothermal: $n(r)\propto\frac{1}{a^2+r^2},a=\SI{12}{k\ps} \begin{pmatrix} a & \text{definiert} \\ n & \text{bei $r=0$}\end{pmatrix}$\\
		Navarro-Frenk-White Modell: $n_{NFM}\propto\frac{1}{\frac{r}{r_s}\left(1+\frac{r}{r_s}\right)^2}$ $r_s\approx\SI{12}{k\ps}$
\end{enumerate}
\subsection{Die Rotationskurve der Galaxis}
\begin{itemize}
	\item Motivitation: Bestimmung der Rotationsgeschwindigkeit $V=V(R)$ as Funktion des Abstands $R$ von galaktischem Zentrum (GC)
		\begin{figure}[H]
			\begin{tikzpicture}
				\coordinate (a) at (0,0);
				\coordinate (b) at (0,-3);
				\coordinate (c) at (2,-1.5);
				\coordinate (d) at (0,-1.5);
				\node[name=v0] at (2,0){$V_0$};
				\draw[->, shorten <= 4pt,shorten >= 2pt] (a)circle(3pt)--(v0);
				\draw (a)--(b)node[midway,left]{$R_0$};
				\draw[->] (b)--(c)node[below right]{$\vec{r}$}node[midway,below right]{$R$};
				\draw[dashed] (d)--(c);
				\draw (a)--(c);
				\pic[draw,"$\theta$",angle eccentricity=1.5] {angle = c--b--a};
				\pic[draw,"$l$",angle eccentricity=1.5] {angle=b--a--c};
			\end{tikzpicture}
		\end{figure}
	\item (kreisförmige) Bewegung in der galaktischen Ebene: $\vec{r}=R\begin{pmatrix}\sin(\theta) \\ \cos(\theta)\end{pmatrix}, \vec{V}=\dot{\vec{r}}=V(R)\cdot\begin{pmatrix} \cos(\theta) \\ -\sin(\theta) \end{pmatrix}$
	\item Mit Hilfe der Abbildung: $\vec{r}=\begin{pmatrix} D\cdot\sin(l) \\ R_0-D\cdot\cos(l) \end{pmatrix}$
		\begin{itemize}
			\item $\sin(\theta)=\frac{D}{R}\sin(l)$\\
				$\cos(\theta)=\frac{R_0}{R}-\frac{D}{R}\cos(l)$\\
				mit $\vec{V}_\odot\approx \vec{V}_\text{LSR}=\begin{pmatrix} V_0 \\ 0 \end{pmatrix}$\\
					$\Delta\vec{V}=\vec{V}-\vec{V}_\odot=\begin{pmatrix} V\frac{R_0}{R}-V\frac{D}{R}\cos(l)-V_0 \\ -V\frac{D}{R}\sin(l)\end{pmatrix}= \begin{pmatrix} R_0-(\Omega-\Omega_0)-\Omega\cdot D\cdot\cos(l) \\ -\Omega D \sin(l) \end{pmatrix}$ mit $\Omega(R)=\frac{V(R)}{R}$ der Winkelgeschwinditkeit $\Omega_0=\frac{V_0}{R_0}$
		\end{itemize}
	\item Komponenten der Relativbewegung zwischen Sonne und Objekten ergibt sich durch Projektion von $\vec{V}:$
		\begin{align*}
			v_r&=\Delta \vec{V}\cdot\begin{pmatrix}\sin(l)\\-\cos(l)\end{pmatrix}=(\Omega-\Omega_0)\cdot R_0\cdot\sin(l) \quad \text{Radialgeschwindigkeit}\\
				v_t&=\Delta \vec{V}\cdot\begin{pmatrix}\cos(l)\\ \sin(l)\end{pmatrix}=(\Omega-\Omega_0)\cdot R_0\cdot\cos(l)-\Omega\cdot D\quad\text{Tangentialgeschwindigkeit}
		\end{align*}
		Messung von $l$ und $v_r$ (Dopplereffekt, siehe 2.8.2) möglich über Eigenbewegung $\mu=\frac{v_t}{D}$ (siehe 2.8.2) erhält man $\Omega$ und $D$
	\item $R=\sqrt{R_0^2+D^2-2R_0D\cos(l)}$
	\item \underline{Problem}: Nicht möglich bei großen $D$ wegen Extinktion in der galaktischen Scheibe ($A_v\sim\SI{28}{\mag}$)
	\item Für kleine $D<<R_0$, lineare Näherung:
		\begin{equation*}
			(\Omega-\Omega_0)\approx\left(\frac{d\Omega}{dR}\right)|_{R_0}\cdot(R-R_0)+\cdots
		\end{equation*}
		\begin{align*}
			\Rightarrow v_r&= (R-R_0)\frac{d\Omega}{dR}|_{R_0}\cdot R_0\sin(l)\\
			&=(R-R_0)\frac{d}{dR}\left(\frac{V}{R}\right)|_{R_0}\cdot\sin(l)\\
			&\approx\left(\left(\frac{dV}{dR}\right)_{R_0}-\frac{V_0}{R_0}\right)\cdot\sin(l)(R-R_0)\\
			&\text{und } v_t=\left(\left(\frac{dV}{dR}\right)|_{R_0}-\frac{V_0}{R_0}\right)\cdot(R-R_0)\cos(l)-\Omega_0\cdot D
		\end{align*}
		Für $(R-R_0)<<R_0 \ \Rightarrow \ R_0-R\approx D\cos(l)$\\
		\begin{equation*}
			\Rightarrow v_r=AD\sin(l),\ v_r=AD\cos(2l)+B\cdot D
		\end{equation*}
		mit den Oortschen Koordinaten:
		\begin{align*}
			A&:=-\frac{1}{2}\left[\left(\frac{dV}{dR}\right)_{R_0}-\frac{V_0}{R_0}\right]\\
			B&:=-\frac{1}{2}\left[\left(\frac{dV}{dR}\right)_{R_0}+\frac{V_0}{R_0}\right]\\
			&\Rightarrow \Omega_0=\frac{V_0}{R_0}=A-B,\quad \left(\frac{dV}{dR}\right)_{R_0}=-(A+B)
		\end{align*}
		\begin{figure}[H]
			\begin{multicols}{2}
				\begin{figure}[H]
					\centering
					\begin{tikzpicture}
						\draw[->] (-0.5,0)--(4,0)node[below right]{$l$};
						\draw[->] (0,-0.5)--(0,2)node[above right]{$v$};
						\draw[domain=0:4,step=1000] plot(\x,{0.5*sin(540*\x/3.14156)+1});
						\draw[domain=0:4,step=1000] plot(\x,{-0.5*sin(540*\x/3.14156)+1});
						\node[above] at (0.5,1.5){$v_r$};
						\node[above] at (1.5,1.5){$v_t$};
					\end{tikzpicture}
				\end{figure}\columnbreak
				Messung ergibt:
				\begin{align*}
					A&=(\num{14.8}\pm\num{0.8})\si{\frac{\km}{\s}\frac{1}{k\ps}}\\
					B&=(-\num{12.4}\pm\num{0.6})\si{\frac{\km}{\s}\frac{1}{k\ps}}
				\end{align*}
			\end{multicols}
		\end{figure}
	\item Bestimmung von $V(R)$ für $R<R_0$:
		\begin{figure}[H]
			\begin{multicols}{2}
				\begin{figure}[H]
					\centering
					\begin{tikzpicture}
						\node[name=o,shape=circle,draw] at (0,0){};
						\draw (o)--(0,-4)node[name=gc,below]{GC};
						\draw (o)--(4,-5)coordinate(b);
						\pic[draw,"$l$",angle eccentricity=1.5] {angle=gc--o--b};
						\foreach \x in {1,...,5}{
							\draw[fill=white,pattern=north west lines] (0.8*\x,-\x)circle(0.3 cm)coordinate(x\x);
							\node[above right] at (0.8*\x+0.1,-\x+0.1){\x};
						};
						\draw (gc)--(x3)node[midway,below right]{$R_{min}$};
					\end{tikzpicture}
				\end{figure}\columnbreak
				\begin{figure}[H]
					\centering
					\begin{tikzpicture}
						\draw[->] (-1,0)--(6,0);
						\draw[->] (0,0)--(0,3);
						\foreach \y in {1,1.5,2.5,3,4}{
							\draw[domain=0:6,samples=500] plot(\x,{3*exp(-20*(\x-\y)^2)});
						}
						\node at (1,1){1};
						\node at (1.5,1){5};
						\node at (2.5,1){2};
						\node at (3,1){4};
						\node at (4,1){3};
					\end{tikzpicture}
				\end{figure}
			\end{multicols}
		\end{figure}
		\begin{itemize}
			\item Messung von $v_r$ mit Hilfe der $\SI{21}{\cm}$ Emissionslinie von neutralem Wasserstoff (galaktische Scheibe transparent für Radiowellen) mit  Hilfe des Doppler-Effekts
		\end{itemize}
	\item Unter der Annahme, dass sich das Gas der Galaxis auf Kreisbahnen um das GC bewegt, ist zu erwarten, dass für die Wolke im Tangentialpunkt (Wolke 3 im Bild) die gesamte Geschwindigkeit auf $v_r$ projeziert wird und sie daher die größte Radialgeschwinditkeit aufweist:
		\begin{align*}
			D&=R_0\cos(l), R_\text{min}=R_0\sin(l) \text{ und } \\
			V_{r,max}&=(\Omega(R_{min}-\Omega_0)R_0\sin(l)=V(R_{min})-V_0\\
			\Rightarrow V(R)&=\frac{R}{R_0}+V_0+V_{r,max}|_{\sin(l)=\frac{R}{R_0}}
		\end{align*}
		$A\neq 0\Rightarrow$ Galaxis rotiert nicht starr.
	\item Bestimmung von $V(R)$ für $R>R_0$: $v_r$ Messung an Objekten deren Entfernung $D_{max}$ bestimmen kann, z.B. Cepheiden
\end{itemize}
\underline{Resultat:}\\
\begin{figure}[H]
	\begin{tikzpicture}
		\draw[->] (-1,0)--(5,0);
		\draw[->] (0,0)--(0,3);
		\draw[domain=0:5,samples=100] plot(\x,{20*\x*exp(-5*\x)+2-2*exp(-\x)});
	\end{tikzpicture}
\end{figure}
\begin{itemize}
	\item Die Rotationskurve fällt nach außen hin ab, trotz einer Stern- und Gasdichte, die exponentiell abfällt:
		\begin{equation*}
			n(R,z)\approx n_0e^{-\frac{|z|}{h_z}}e^{-\frac{R}{h_z}} \text{ mit } h_z\approx\SI{0.3}{k\ps},\ h_R\approx\SI{3.5}{k\ps}
		\end{equation*}
	\item wenn es nur sichtbare Materie gäbe, würde Ian + Kepler gelten: $V(R)\sim R^{-\frac{1}{2}}$
	\item aber man erhält $V(R)\sim cnst.\ \Rightarrow M(R)\sim R$\\
		$\Rightarrow$ "`dunkle Materie"' (?)
\end{itemize}

\subsection{Die Welt der Galaxien}
\begin{itemize}
	\item Erkenntnis, dass die Milchstraße nur eine Galaxie von viele ist $\sim \num{100}$ Jahre alt\\
		vorher: Katalog von Charles Messier (1730-1817)\\
		enthält 103 diffuse Objekte (z.B. M31: Andromeda-Galaxie)\\
		19. Jhd.: NGC=New General Catalogue, John Preyer (1852-1926)\\
		1925: Beobachtung von Cepheiden in M31 durch Edwin Hubble\\
		\begin{itemize}
			\item $D=\SI{285}{k\ps}$ (Aktueller: $\SI{770}{k\ps}$)
		\end{itemize}
		1928: Beobachtung des Auseinanderstrebens der Galaxien
		\begin{itemize}
			\item[] Hubble'sche Gesetz:
				\begin{equation*}
					v=H_0\cdot r \text{ mit } H_0\simeq \SI{72}{\frac{\km}{\s}\cdot\frac{1}{M\ps}} \text{ der Hubble'schen Konstante}
				\end{equation*}
		\end{itemize}
	\item Morphologische Klassifizierung (Hubble-Sequenz)
		\begin{figure}[H]
			\begin{multicols}{2}
				\begin{figure}[H]
					\centering
					\begin{tikzpicture}
						\draw[fill=black!50!white] (0,0)circle(1cm);
						\node at (1.5,0){$\cdots$};
						\draw[fill=black!50!white,xshift=3cm] (0,0)ellipse(1cm and 0.5cm);
					\end{tikzpicture}
					\caption{$E_0\cdots E_7$}
				\end{figure}\columnbreak
				\begin{figure}[H]
					\centering
					\begin{tikzpicture}
						\draw[fill=black!50!white] (0,0)circle(0.5cm);
						\draw[thick] (-2,0)--(2,0);
					\end{tikzpicture}
					\caption{$S_0$}
				\end{figure}
			\end{multicols}
			\begin{multicols}{3}
				\begin{figure}[H]
					\centering
					\begin{tikzpicture}
						\draw[domain=0:90,samples=50] plot({\x*cos(\x)/45},{\x*sin(\x)/45})plot({\x*cos(\x+180)/45},{\x*sin(\x+180)/45});
						\draw[fill=black!50!white] (0,0)circle(0.5cm);
					\end{tikzpicture}
					\caption{$S_a$}
				\end{figure}\columnbreak
				\begin{figure}[H]
					\centering
					\begin{tikzpicture}
						\draw[domain=0:180,samples=50] plot({\x*cos(\x)/90},{\x*sin(\x)/90})plot({\x*cos(\x+180)/90},{\x*sin(\x+180)/90});
						\draw[fill=black!50!white] (0,0)circle(0.5cm);
					\end{tikzpicture}
					\caption{$S_b$}
				\end{figure}\columnbreak
				\begin{figure}[H]
					\centering
					\begin{tikzpicture}
						\draw[domain=0:720,samples=50] plot({\x*cos(\x)/360},{\x*sin(\x)/360})plot({\x*cos(\x+180)/360},{\x*sin(\x+180)/360});
						\draw[fill=black!50!white] (0,0)circle(0.5cm);
					\end{tikzpicture}
					\caption{$S_c$}
				\end{figure}
			\end{multicols}
			\begin{multicols}{3}
				\begin{figure}[H]
					\centering
					\begin{tikzpicture}
						\draw[very thick] (-2,0)--(2,0);
						\draw[fill=black!50!white] (0,0)circle(0.5cm);
						\draw[domain=0:170,samples=50] plot({2*cos(\x)},{2*sin(\x)})plot({2*cos(\x+180)},{2*sin(\x+180)});
					\end{tikzpicture}
					\caption{$S_{B_a}$}
				\end{figure}\columnbreak
				\begin{figure}[H]
					\centering
					\begin{tikzpicture}
						\draw[very thick] (-1.5,0)--(1.5,0);
						\draw[fill=black!50!white] (0,0)circle(0.5cm);
						\draw[domain=0:100,samples=50] plot({(\x/200+1.5)*cos(\x)},{(\x/200+1.5)*sin(\x)})plot({(\x/200+1.5)*cos(\x+180)},{(\x/200+1.5)*sin(\x+180)});
					\end{tikzpicture}
					\caption{$S_{B_b}$}
				\end{figure}\columnbreak
				\begin{figure}[H]
					\centering
					\begin{tikzpicture}
						\draw[very thick] (-1.5,0)--(1.5,0);
						\draw[fill=black!50!white] (0,0)circle(0.5cm);
						\draw[domain=0:60,samples=50] plot({(\x/40+1.5)*cos(\x)},{(\x/40+1.5)*sin(\x)})plot({(\x/40+1.5)*cos(\x+180)},{(\x/40+1.5)*sin(\x+180)});
					\end{tikzpicture}
					\caption{$S_{B_c}$}
				\end{figure}
			\end{multicols}
		\end{figure}
		\begin{itemize}[label={}]
			\item Elliptische Galaxien E
			\item Spiralgalaxien mit und ohne Balken (S,S\textsubscript{B})
			\item Irreguläre Galaxien I
			\item Milchstraße S\textsubscript{B\textsubscript{bc}}
		\end{itemize}
\end{itemize}
\subsubsection{Elliptische Galaxien Allgemein}
Normale Ellipsen: $E$ mit Elliptizität $0\leq\epsilon\leq\num{0.7}$\\
$E_n$ wobei $n=\SI{10}{\epsilon}\Rightarrow E_0-E_7$
\begin{itemize}
	\item $S_0$ Galaxien: Übergang zwischen Elliptischen und Spiralgalaxien (linsenförmig mit Bulge und Scheibe ohne Spiralarme)
	\item CD Galaxien elliptische Riesengalaxien
		\begin{table}[H]
			\begin{tabular}{p{3cm}|p{3cm}|p{3cm}|p{3cm}}
				& $E$ & $S_0$ & $CD$ \\\hline
				Absolute Helligkeit & $-\num{15}-(-\num{23})$ & $(-\num{17})-(-\num{22})$ & $(-\num{22})-(-\num{25})$ \\\hline
				$\frac{M}{M_0}$ & $\num{100}-10^{13}$ & $10^{10}-10^{12}$ & $10^{13}-10^{14}$ \\\hline
				$\frac{2R}{\si{k\ps}}$ & $\num{1}-\num{200}$ & $\num{10}-\num{100}$ & $\num{300}-\num{1000}$ \\\hline
				$\frac{\frac{M}{L_B}}{\frac{M_\odot}{L_\odot}}$ & $\num{10}-\num{100}$ & $\sim\num{10}$ & $>\num{100}$
			\end{tabular}
		\end{table}
	\item Leuchtkraft: de Vancouleurs-Profil: $\log\left(\frac{I(R)}{I_0}\right)=-\num{3.3307}\left(\left(\frac{R}{R_e}\right)^\frac{1}{4}-1\right)$\\
		$I(R)$: Flächenhelligkeit in $\frac{L_\odot}{\si{\ps^2}}$\\
		$R_l$: Effektivradius, mit $\int\limits_0^{R_l}dRR_I(R)=\frac{1}{2}\int\limits_0^\infty dR R_I(R)$
	\item woher kommt die Abplattung der Elliptischen Galaxie?
		\begin{enumerate}[label={$\roman*)$}]
			\item Rotation? In diesem Fall gelte: $\frac{rot}{\sigma}\approx\sqrt{\frac{\epsilon}{1-\epsilon}}$\\
				\textbf{Aber}: Beobachtet $v_r\ <<\sigma$
			\item Selbst gravitierendes Gleichgewichtssystem (Elliptizität ergibt sich aus den Anfangsbedingungen) Zeitlich stabil? $\to$ "`thermalisierung"' durch 2er Stöße?
		\end{enumerate}
	\item Betrachte Relaxationszeit druch 2er Stöße in einem System von $N$ Sternen der Masse $m$ (Gesamtmasse $M=N\cdot m$) der Ausdehnung $R$: $t_{relax}=\frac{R}{v}\cdot\frac{N}{\ln(N)}=\underset{\underset{\underset{\text{eines Sterns beim durchqueren}}{\text{charakteristische Zeit}}}{\uparrow}}{t_{cross}}\cdot\frac{N}{\ln(N)}$\\
		charakteristische Zeit, in der ein Stern durch 2er-Stöße  seine Richtung um $\SI{90}{\degree}$ ändert\\
		für eine Galaxie $N\sim 10^{12}, t_{cross}\approx 10^8\si{a}\Rightarrow t_{relax}\approx\SI{14e9}{a}$ (älter als Universum) $\Rightarrow $ stabil
\end{itemize}
\subsubsection{Spiralgalaxien}
\begin{table}[H]
	\def\k{2.3cm}
	\begin{tabular}{p{\k}|p{\k}|p{\k}|p{\k}|p{\k}}
		& $S_a$ & $S_b$ & $S_c$ & $S_{d}$ \\\hline
		$M_B$ & $-\num{17}-(-\num{23})$ & $-\num{17}-(-\num{23})$ & $(-\num{10})-(-\num{22})$ & $(-\num{15}-(-\num{20})$ \\\hline
		$\frac{M}{M_\odot}$ & $10^9-10^{12}$ & $10^9-10^{12}$ & $10^9-10^{12}$ & $10^8-10^{10}$ \\\hline
		$\frac{\frac{M}{L_B}}{\frac{M_\odot}{L_\odot}}$ & $\num{6.0}\pm\num{0.6}$ & $\num{4.5}\pm\num{0.4}$ & $\num{2.6}\pm\num{0.2}$ & $-\num{1}$ \\\hline
		$\frac{L_{Bulge}}{L_{tot}}$ & $\num{0.3}$ & $\num{0.13}$ & $\num{0.05}$ & - \\\hline
		$\frac{2R}{\si{k\ps}}$ & $\num{5}-\num{100}$ & $\num{5}-\num{100}$ & $\num{5}-\num{100}$ & $\num{0.5}-\num{50}$ \\\hline
		$\frac{v_{max}}{\left(\si{\frac{\km}{\s}}\right)}$ & $\num{300}$ & $\num{220}$ & $\num{175}$ & - 
	\end{tabular}
\end{table}
Profil des Bulges folgt einem de Vancouleurs-Profil:\\
Mit Flächenhelligkeit $U\sim\num{2.5}\log(I):$\\
$M_{Bulge}(R)=U_e+\num{8.326}\cdot\left(\left(\frac{R}{R_e}\right)^\frac{1}{4}-1\right)$\\
$M_{Scheibe}(R)=U_0+\num{1.09}\left(\frac{R}{n_R}\right)$\\
Ian Freeman: $U_0$ ist konstant für verschiedene Spiralgalaxien\\
$S_a-S_c$: $U_0=\num{21.5}\pm\SI{0.39}{\frac{\text{mag}}{\text{arc}\s^2}}$\\
$S_d$: $U_0=\num{22.07}\pm\SI{0.4}{\frac{\text{mag}}{\text{arc}\s^2}}$
\begin{itemize}
	\item Rotationsachsen von Spiralgalaxien verlaufen nicht wie durch die Lichtverteilung erwartet für $R\geq n_R$, sondern im wesentlichen flach, (vgl. Kap 3.3)
		\begin{itemize}
			\item Spiralgalxien sind von einem Halo dunkler Materie umgeben!\\
				Es gilt (Gleichgewicht zwischen Zentrifugal- und Gravitationskraft):
				\begin{equation*}
					V^2(R)=\frac{G\cdot M(R)}{R} \quad\text{Gesamtmasse von $R$}
				\end{equation*}
				für sichtbare Materie:
				\begin{equation*}
					V^2_{enm}=\frac{GM_{enm}(R)}{R} \ \Rightarrow\ V^2_{dark}=V^2-V_{enm}^2=\frac{G\cdot M_{dark}}{R}
				\end{equation*}
			\item $M_{dark}(R)=\frac{R}{G}\left(V^2(R)-V^2_{enm}(R)\right)$
				\begin{itemize}[label={\textbullet}]
					\item Problem: äußerer Radius ist Halo und daher nicht klar\\
						Sterne: $R\geq\SI{40}{k\ps}$\\
						Satelliten: $R\geq\SI{140}{k\ps}$
				\end{itemize}
				Zusammensetzung:
				\begin{itemize}[label={}]
					\item Elliptische Galaxien: alte Sterne (orange und rot)
					\item Spiralgalaxien: Bulge ähnlich wie elliptische Galaxien, Spiralarme mit geringem blauanteil
				\end{itemize}
		\end{itemize}
\end{itemize}
\subsubsection{Skalengesetze}
Tully-Fisher (177)
\begin{equation*}
	L\sim V^4_{max}
\end{equation*}
\begin{itemize}[label={$\cdot$}]
	\item Korrelation zwischen maximaler Rotationsgeschwindigkeit von Spiralen und Leuchtkraft
		\begin{itemize}
			\item recht genaue Abschätzung der Leuchtkraft aus der Rotationskurve der Galaxie
		\end{itemize}
	\item Durch Vergleich mit scheinbarer Helligkeit $\Rightarrow$ Abstand
	\item heuristische Herleitung:\\
		Rotationskurve bei $V(R)=const. $ impliziert
		\begin{equation*}
			M=\frac{V_{max}^2\cdot R}{G}\ \Rightarrow\ L=\left(\frac{M}{L}\right)^{-1}\cdot\frac{V_{max}\cdot R}{G}
		\end{equation*}
		mittlere Flächenhelligkeit: $\langle I\rangle = \frac{L}{R^2}$
		\begin{equation*}
			\Rightarrow L=\left(\frac{M}{L}\right)^{-1}\cdot\frac{V_{max}\cdot R}{G\cdot M}\cdot\frac{V_{max}\cdot R}{G}=\left(\frac{M}{L}\right)^{-2}\cdot\frac{V_{max}}{G}\cdot\frac{1}{\langle I\rangle}
		\end{equation*}
		Freeman: $\langle I\rangle\sim cnst.$ für Spiralgalaxien\\
		$\frac{M}{L}$ variiert nur wenig $\Rightarrow\ L\sim V_{max}^4$\\
		(kein Beweis, aber macht Skalengesetz plausibel)
\end{itemize}

\subsection{Ist das Universum unendlich, euklidisch und statisch?}
\begin{itemize}
	\item Naive Annahme eines
		\begin{itemize}[label={\textbullet}]
			\item räumliche unendlichen
			\item euklidischen
			\item statischen
		\end{itemize}
		Universums ist im Widerspruch zu (I) und (VIII)
\end{itemize}
Zu (I) Olders-Paradoxon:
\begin{itemize}[label={\textbullet}]
	\item[] Der Nachthimmel in solch einem Universum wäre (ungemütlich) hell!
	\item Betrachte dazu:
		\begin{itemize}[label={}]
			\item $n_\ast$: mittlere Anzahldichte der Sterne
			\item $R_\ast$: mittlerer Radius eines Sterns
		\end{itemize}
	\item Eien Kugelschale mit Radius $r$ und Dicke $dr$ um die $\odot$ enthält $4\pi r^2n_\ast dr$ Sterne, jeder mit Raumwinkel $\frac{\pi R^2_\ast}{r^2}\Rightarrow $ gesamter von $\ast$ eingenommener Raumwinkel:\\
	$d\omega = 4\pi r^2drn_\ast\frac{\pi R_\ast^2}{r^2}=4\pi^2n_\ast R_\ast^2$ dr unabhängig von $r$ $\Rightarrow $ im gesamten Universums
		$\omega=\int\limits_0^\infty dr\frac{d\omega}{dr}=4\pi^2n_\ast R_\ast^2\int\limits_0^\infty dr=\infty$ !
	\item Offensichtlich haben wir den Effekt von sich überlappenden Sternscheiben an der Sphäre nicht berücksichtigt
		\begin{itemize}
			\item Jedoch zeigt diese Betrachtung, dass der Himmel von Sternscheiben vollständig gefüllt wäre
			\item Der Himmel wäre so hell wie die Oberfläche eines typischen Sterns (z.B. die Sonne)
		\end{itemize}
\end{itemize}
zu (VIII)
\begin{itemize}[label={\textbullet}]
	\item Sei $n(>L)$ die räumliche Anzahldichte von Radioquellen mit Leuchtkräften $>L$.
	\item eine Kugelschale mit Radius $r$ und Dicke $dr$ um die $\odot$ enthält wiederum $4\pi r^2 drn(>L)$ Quellen
	\item $L=4\pi r^2\cdot S$, mit $S$: beobachteter Fluss
		\begin{itemize}
			\item $dN(>S)=4\pi r^2 dr n(>(4\pi r^2S))$
			\item $N(>S)=\int\limits_0^\infty dr 4\pi r^2n(>(4\pi r^2S))\underset{\underset{r=\sqrt{\frac{L}{4\pi S}}}{\uparrow}}{=}\int\limits_0^\infty\frac{dL}{2\sqrt{4\pi LS}}\frac{L}{4\pi S}n(>L)$\\
				$=\frac{1}{16\pi^\frac{3}{2}}S^{-\frac{3}{2}}\int\limits_0^\infty DL\sqrt{L}n(>L)\propto S^{-\frac{3}{2}}$
		\end{itemize}
		\begin{itemize}
			\item Wenigstens eine der drei Ausganshypothesen ist falsch.\\
				Rotverschiebung der Galaxien/Hubble-Gesetz $\Rightarrow $ nicht-statisches Universum
		\end{itemize}
\end{itemize}
zu (V) $\Rightarrow $ Alter des Universums $>\SI{12e9}{\year}$
zu (II) und (IV) $\Rightarrow $ Das Universum scheint auf ausreichend großen Skalen isotrop zu sein.
\begin{itemize}
	\item Falls unser Standort im Kosmos nicht ausgezeichnet ist
		\begin{itemize}
			\item Das Universum ist auch homogen.
		\end{itemize}
\end{itemize}
\textbf{\underline{\smash{Kosmologisches Prinzip}}: Das Universum ist homogen und Isotrop}
\begin{itemize}
	\item Homogenität ist nicht direkt beobachtbar und auf kleinen Skalen hinfällig (bis zu $\sim \SI{100}{h^{-1}M\ps}$), allerdings bisher keine Hinweise auf Strukturen $>>\SI{100}{h^{-1}M\ps}$
	\item Dies ist klein im Vergleich zum Hubble-Radius (= charakteristische Größe des beobachtbaren Universums)
		\begin{equation*}
			R_H=\frac{c}{H_0}=\SI{3000}{h^{-1}M\ps}
		\end{equation*}
		\begin{itemize}
			\item $\underset{\underset{\text{1. Annäherung (später zu präzisieren)}}{\uparrow}}{\text{Homogenität}}$ und Isotropie auf Skalen von $(100-3000)\ \si{h^{-1}M\ps}$
		\end{itemize}
\end{itemize}

\subsection{Ein expandierendes Universum}
\begin{itemize}
	\item Betrachte eine homogene Kugel der Massendichte $\rho=\rho(t)$
	\item Ort eines Volumenelements:
		\begin{equation*}
			\vec{r}(t)=\underset{\underset{\underset{a(t_0)=1, t_0\hat{=}\text{heute}}{\text{Skalenfaktor}}}{\uparrow}}{a(t)}\vec{r}(t_0)
		\end{equation*}
		\begin{itemize}
			\item Position eines mitbewegten Beobachters mit Weltlinie $\left(\vec{r},t\right)=\left(a(t)\cdot \vec{r}_0,t\right)$ und Geschwindigkeit
				\begin{equation*}
					\vec{v}=\frac{d}{dt}\vec{r}=\frac{da}{dt}\cdot\vec{r}_0=\frac{\dot{a}}{a}\cdot\vec{r}(t)=:H(t)\cdot\vec{r}(t)
				\end{equation*}
				$\leadsto$ Expansionsrate: $H(t):=\frac{\dot{a}}{a}$
		\end{itemize}
	\item insbesondere Relativgeschwindigkeit zweier mitbewegter Punkte:
		\begin{equation*}
			\Delta\vec{v}=\vec{c}\left(\vec{r}+\Delta\vec{r},t\right)-\vec{v}\left(\vec{r},t\right)=H(t)\cdot\Delta\vec{r}
		\end{equation*}
		$\hat{=}$ Verallgemeinerung des Hubble-Gesetzes, für das gilt: $H(t_0)=H_0$
\end{itemize}
\subsubsection{Newtonsche Kosmologie}
\begin{itemize}
	\item Radius $\vec{r}(t)\equiv a(t)r$ einer Kugel der Masse:
		\begin{equation*}
			M(r_0)=\frac{4\pi}{3}\rho_0r_0^3=\frac{4\pi}{3}\rho(t)\big(a(t)\cdot r_0\big)^3
		\end{equation*}
		\begin{itemize}
			\item $\rho(t)=\rho_0\cdot a(t)^{-3}$
		\end{itemize}
	\item Bewegungsgleichung:
		\begin{align*}
			\ddot{r(t)}&=-\frac{G\cdot M(r_0)}{r^2}=-\frac{4\pi G}{3}\frac{\rho_0\cdot r_0^3}{r^2}\\
			\ddot{a}&=\frac{\ddot{r}}{r}=-\frac{4\pi G}{3}\rho\cdot a\qquad\text{unabhängig von $r_0$}
		\end{align*}
	\item "`Energieerhaltung"':
		\begin{align*}
			\dot{a}^2&=\frac{8\pi G}{3}\rho_0\frac{1}{a}-K\cdot c^2\\
			&=\frac{8\pi G}{3}\rho(t)\cdot a(t)-Kc^2
		\end{align*}
		mit $K\propto $ Gesamtenergie eines mitbewegten Teilchens
		\begin{itemize}[label={$\cdot$}]
			\item wenn $K<0\Rightarrow \dot{a}>0\Rightarrow$ Universum expandiert ewig
			\item wenn $K>0\Rightarrow \dot{a}<0\Rightarrow$ bei größeren $t\Rightarrow$ Universum rekollabiert
			\item wenn $K=0\Rightarrow$ kritische Dichte:
				\begin{equation*}
					\left(\frac{\dot{a}(t_0)}{a(t_0)}\right)^2=\frac{8\pi G}{3}\rho_0
				\end{equation*}
		\end{itemize}
		\begin{itemize}
			\item $\rho_c=\frac{3H_0^2}{8\pi G}\approx\SI{1.88e-29}{h^2\frac{\g}{\cm^3}}$
			\item Dichteparameter: $\Omega_0:=\frac{\rho_0}{\rho_c}$
		\end{itemize}
\end{itemize}
\subsubsection{Relativitätstheorie}
\begin{itemize}
	\item \textbf{Relativitätsprinzip} (Einstein 1905):\\
		Die Naturgesetze haben in allen Inertialsystemen die selbe Form.
	\item betrifft Wechsel zwischen Bezugssystemen mit konstanter Relativgeschwindigkeit ($\Rightarrow$ spezielle Lorentz-Transformation)
	\item daraus folgende Vorhersagen wurden spektakulär nachgewiesen (Längenkontraktion, Zeitdilatation, etc.) und betrifft alle Untergebiete der modernen Physik
	\item Für die Kosmologie muss die Gravitation miteinbezogen werden, da es die einzige \textbf{bekannte} Kraft ist, die auf kosmischen Distanzen wirkt
	\item \textbf{Äquivalenzprinzip} (Einstein 1914):\\
		In jedem Punkt der Raumzeit (mit und ohne Gravitationsfelder) kann man ein (in Zeit und Raum) \textbf{lokales} Inertialsystem so wählen, dass die physikalischen Gesetze denen eines unbeschleunigten kartesischen Bezugssystems entsprechen
	\item Mathematisch kann gezeigt werden, dass die Gravitation die Geometrie der Raumzeit beeinflusst
		\begin{enumerate}[label={$(\roman*)$}]
			\item \textbf{ohne} Gravitation:
				\begin{equation*}
					ds^2=c^2dt^2-dx^2-dy^2-dz^2
				\end{equation*}
			\item \textbf{mit} Gravitation:
				\begin{equation*}
					ds^2=\sum\limits_{\mu,\nu=0}^3g_{\mu\nu}dx^\mu dx^\nu \qquad \text{ mit } x^\mu=(c\cdot t, x, y, z)
				\end{equation*}
		\end{enumerate}
	\item Der metrische Tensor $g_{\mu\nu}=g_{\nu\mu}$ bestimmt die geometrischen Eigenschaften der Raumzeit
	\item $g_{\mu\nu}$ ergibt sich aus den Einstein'schen Feldgleichungen der Allgemeinen Relativitätstheorie $\rightarrow$ $\num{10}$ nichtlineare, partielle, gekoppelte Differentialgleichungen für die $\num{10}$ unabhängigen Komponenten der Metrik
		\begin{equation*}
			\underset{\underset{\underset{\underset{\text{ihre 1. und 2. Ableitungen}}{\text{enthält $g_{\mu\nu}$ und}}}{\text{Einstein-Tensor}}}{\uparrow}}{G_{\mu\nu}}=\underset{\kappa=\frac{8\pi G}{c^4}\ \left[\text{SI}\right]}{\underset{\underset{\underset{\underset{\underset{\text{Einheitensystem)}}{\text{dem gewählten}}}{\text{(ergibt sich aus}}}{\text{Konstante}}}{\uparrow}}{\kappa}}\cdot \underset{\underset{\text{Energieimpulstensor}}{\uparrow}}{T_{\mu\nu}}
		\end{equation*}
		\begin{align*}
			G_{\mu\nu}&=R_{\mu\nu}-\frac{1}{2}g_{\mu\nu}R\\
			R_{\mu\nu}&=\sum\limits_{\alpha=0}^3R^\alpha_{\mu\alpha\nu} \qquad R=\sum\limits_{\mu,\nu=0}^3g^{\mu\nu}R_{\mu\nu}\\
			R_{\nu\sigma\tau}^\mu&=\partial_\sigma \Gamma^\mu_{\nu\tau}-\partial_\tau \Gamma^\mu_\nu\sigma + \sum\limits_{\alpha=0}^3\Gamma^\mu_{\sigma\alpha}\Gamma^\alpha_{\nu\tau}-\sum\limits_{\alpha=0}^3\Gamma^\mu_{\tau\alpha}\Gamma^\alpha_{\mu\sigma}\\
			\Gamma^\sigma_{\mu\nu}&=\sum\limits_{\alpha=0}^3\frac{1}{2}g^{\sigma\alpha}\left(\partial_\nu g_{\alpha\mu}+\partial_\mu g_{\alpha\nu}-\partial_\alpha g_{\mu\nu}\right)
		\end{align*}
\end{itemize}
\subsubsection{Gekrümmte Räume}
\begin{itemize}
	\item Die Ebene ist flach, Vektoren können parallel verschoben werden, ohne dass sie ihre Richtung ändern
		\begin{figure}[H]
			\centering
			\begin{tikzpicture}
				\draw[->,>=stealth] (0,0)node[below left]{$\vec{u}$}--(0,2)node[midway,name=x]{};
				\draw[->,>=stealth] (2,0)node[below right]{$\vec{u}'$}--(2,2)node[midway,name=y]{};
				\draw[->,densely dashed,shorten <= 2pt, shorten >= 2pt] (x)--(y);
			\end{tikzpicture}
		\end{figure}
	\item Ist die Kugel flach?
		\begin{figure}[H]
			\begin{multicols}{2}
				\begin{figure}[H]
					\centering
					\begin{tikzpicture}[>=stealth]
						\draw (0,0)circle(2cm);
						\draw (0,0)ellipse(2cm and 0.5cm);
						\draw[densely dashed,->,domain=194:133] plot({0.5*cos(\x)},{2*sin(\x)});
						\draw[densely dashed,->,domain=133:56] plot({0.5*cos(\x)},{2*sin(\x)});
						\draw[green,->,domain=76:90,thick] plot({2*cos(\x)},{0.5*sin(\x)});
						\node[green,above] at ({2*cos(83)},{0.5*sin(83)}){$\vec{v}'$};
						\draw[densely dashed,->,domain=56:14] plot({0.5*cos(\x)},{2*sin(\x)});
						\draw[->,domain=76:16] plot({2*cos(\x)},{0.5*sin(\x)});
						\draw[->,domain=16:-36] plot({2*cos(\x)},{0.5*sin(\x)});
						\draw[->,domain=16:-90] plot({2*cos(\x)},{0.5*sin(\x)});
						\draw[red,->,domain=256:242,thick] plot({2*cos(\x)},{0.5*sin(\x)});
						\node[red,below] at ({2*cos(249)},{0.5*sin(249)}){$\vec{v}$};
						\draw[blue,->,domain=256:270,thick] plot({2*cos(\x)},{0.5*sin(\x)});
						\node[blue,above] at ({2*cos(263)},{0.5*sin(263)}){$\vec{v}''$};
					\end{tikzpicture}
				\end{figure}\columnbreak
				Parallelverschiebung von $\vec{v}$ liefert nach Abschluss des gestrichelten Weges den Vektor $\vec{v}'$ und dann mit Verschiebung entlang des Äquators den Vektor $\vec{v}''=-\vec{v}$
			\end{multicols}
		\end{figure}
	\item Weitere Beispiele:
		\begin{figure}[H]
			\centering
			\begin{multicols}{3}
				\begin{figure}[H]
					\centering
					\begin{tikzpicture}
						\draw (0,0)circle(1.5cm);
						\draw (0,1)ellipse(1.08cm and 0.3cm);
						\draw[domain=90:150] plot({0.3*cos(\x)},{1.5*sin(\x)});
						\node[below right] at ({0.3*cos(120)},{1.5*sin(120)}){$a$};
					\end{tikzpicture}\vspace{0.2cm}\\
					\begin{tikzpicture}
						\foreach \y in {0,...,5}{
							\draw[domain={\y*60+5}:{(\y+1)*60-5}] (0,0)--plot({cos(\x)},{sin(\x)})--(0,0);
						}
					\end{tikzpicture}
				\end{figure}
				$C<2\pi a$
				\columnbreak
				\begin{figure}[H]
					\centering
					\begin{tikzpicture}[scale=0.7]
						\draw (-3,-2)rectangle(3,2);
						\draw (0,0)circle(1cm);
					\end{tikzpicture}\vspace{0.2cm}\\
					\begin{tikzpicture}[scale=0.7]
						\draw (0,0)circle(1cm);
					\end{tikzpicture}
				\end{figure}
				$C=2\pi a$\\\columnbreak
			\end{multicols}
		\end{figure}
	\item Wie kann man Krümmung messen?\\
	$to$ Vergleich von Kreisumfang $C$ und Kreisfläche $A$ eines Kreises mit Radius $a$
		\begin{table}[H]
			\begin{tabular}{l|c|c|r}
				\hline Kugel & $C<2\pi a$ & $A<\pi a^2$ & positiv\\
				\hline Ebene & $C=2\pi a$ & $A=\pi a^2$ & null (=flach)\\
				\hline Sattel & $C>2\pi a$ & $A>\pi a^2$ & negativ \\\hline
			\end{tabular}
		\end{table}
		\noindent\textbf{Krümmung ist eine intrinsische Eigenschaft!}\\
		d.h. man kann nie messen, ohne den \glqq Raum\grqq{} zu verlassen (essentiell in der Kosmologie)
\end{itemize}
\subsubsection{Modifikation der Newtonschen Kosmologie}
\begin{enumerate}[label={$(\alph*)$}]
	\item Äquivalenz von Masse und Energie ($E=mc^2$)
		\begin{itemize}[label={$\Rightarrow$}]
			\item $\rho$ in den kosmologischen Bewegungsgleichungen enthält nicht nur die Materiedichte
		\end{itemize}
	\item Die Einsteinschen Feldgleichungen der ART erlauben einen zusätzlichen Term, die \textbf{Kosmologische Konstante $\Lambda$}
	\item Die Interpretatino der Expansion des Kosmos ändert sich: Das Universum ist keine expandierende Kugel, sondern der \textbf{Raum selbst expandiert}
		\begin{itemize}[label={$\Rightarrow$}]
			\item Die Rotverschiebung ist eine Eigenschaft der expandierenden Raumzeit ($a=a(t)=\text{Skalenfaktor}$)
		\end{itemize}
\end{enumerate}
\begin{itemize}
	\item \textbf{Erster Hauptsatz der Thermodynamik}\\
		\begin{equation*}
			\underset{\underset{\underset{\text{inneren Energie}}{\text{Änderung der}}}{\uparrow}}{dU}=-\underset{\underset{\text{Druck}}{\uparrow}}{P}\cdot\underset{\underset{\underset{\text{Volumenänderung}}{\text{(adiabatische)}}}{\uparrow}}{dV}
		\end{equation*}
		Aus den Gleichungen der Allgemeinen Relativitätstheorie folgt eine analoge Relation für einen homogenen und isotropen Kosmos:
		\begin{equation*}
			\frac{d}{dt}(\underset{\underset{\text{Energiedichte}}{\downarrow}}{\underbrace{c^2\rho}}a^3)=-P\cdot\frac{d(a^3)}{dt}
		\end{equation*}
		(d.h. für "`normale"' Materie ist $\rho$ die Massendichte, $P$ der Druck der Materie und $V=a^3(t)\cdot V_0$ das Volumen)
	\item Die Friedmann-Lena\^itre Expansionsgleichung:
		\begin{align*}
			\left(\frac{\dot{a}}{a}\right)^2&=\frac{8\pi G}{3}\rho -\frac{Kc^2}{a^2}+\frac{\Lambda}{3} \qquad (F1)\\
			\frac{\ddot{a}}{a}&=-\frac{4\pi G}{3}\left(\rho+\frac{3P}{c^2}\right)+\frac{1}{3} \qquad (F2)
		\end{align*}
		$\cdot$  Unterschiede zu Newton:
		\begin{enumerate}[label={$(\roman*)$}]
			\item zusätzlicher Druckterm um (F2)
			\item kosmologische Konstante
		\end{enumerate}
	\item Die \textbf{Kosmologische Konstante}:
		NB: Falls $\Lambda =0$, gibt es keine Lösung der Friedmann-Lena\^itre-Gleichungen mit $\dot{a}=0$ (siehe Übungsblatt 7, Aufgabe 2)
		\begin{itemize}
			\item $\Lambda\neq 0$ eingeführt von Einstein um das statische Universum "zu retten"
			\item[1923] Eddington diskutiert die Möglichkeit eines nicht statischen Universums
			\item[1929] Hubble beobachtet eine systematische Expansion\\
				Quantenfeldtheorie: auch das Vakuum enthält eine nicht verschwindende Energie
			\item mathematisch äquivalent zu $\Lambda\neq 0$ (aber Größenordungen stimmen nicht!)
		\end{itemize}
\end{itemize}
\subsubsection{Die Materiekomponenten des Universums}
\begin{itemize}
	\item Druckfreie Materie ("`Staub"')
		\begin{equation*}
			P_m=0
		\end{equation*}
		Druck eines Gases $\propto$ thermische Bewegung\\
		z.B. Moleküle der Luft mit $v\sim v_{\text{Schall}}\sim\SI{300}{\frac{\m}{\s}}\ \Rightarrow P_m<<\rho_mc^2$
	\item Strahlung $P_r=\frac{1}{3}\rho_rc^2$\\
		z.B. Photonen des CMB
	\item Vakuumenergie
		\begin{equation*}
			P_V=-\underset{\underset{\underset{\text{mit $\rho_V=cnst.$}}{\text{aus dem 1. HS (s.o.)}}}{\uparrow}}{\rho_V}c^2
		\end{equation*}
		Das Vakuum übt einen negativen Druck aus
\end{itemize}
\subsubsection{Heuristische Herleitung der Friedmann-Lena\^itre-Expansionsgleichungen}
\underline{IVB}: Eine korrekte Herleitung ergibt sich direkt aus der ART!
\begin{itemize}
	\item Hier nutzen wir die Newtonschen Gleichungen und reinterpretieren die "`Energieerhaltung"'
		\begin{equation*}
			\dot{a}^2=\frac{8\pi G}{3}{\rho}a^2-Kc^2\overset{\frac{d}{dt}}{\Rightarrow} 2\dot{a}\ddot{a}=\frac{8\pi G}{3}\left(\dot{\rho}a^3+2\rho a\dot{a}\right)
		\end{equation*}
	\item 1. HS $\left(\frac{d}{dt}\left(c^2\rho a^3\right)=-P\frac{d(a^3)}{dt}\right)$\\
		\begin{align*}
			\dot{\rho}a^3+3\rho a^2\dot{a}&=-3\frac{P}{c^2}a^2\dot{a}\\
			\Rightarrow \dot{\rho}&=-3\rho\frac{\dot{a}}{a}-3\frac{P}{c^2}\frac{\dot{a}}{a} \quad \text{ und einsetzen ergibt:}\\
			\Rightarrow \dot{a}\ddot{a}&=\frac{4\pi G}{3}\left(-\rho\dot{a}a-3\frac{P}{c^2}\dot{a} a\right)
			\Rightarrow \frac{\ddot{a}}{a}&=-\frac{4\pi G}{3}\left(\rho+\frac{3P}{c^2}\right)
		\end{align*}
	\item neu dabei:
		\begin{align*}
			\rho&=\underset{\underset{\text{Materie}}{\uparrow}}{\rho_m}+\underset{\underset{\underset{\text{(Photonen)}}{\text{Strahlung}}}{\uparrow}}{\rho_r}+\underset{\underset{\text{Vakuum}}{\uparrow}}{\rho_V}\\
			P&=\underset{\underset{\text{Materie}}{\uparrow}}{\sigma}+\underset{\underset{\underset{\text{(Photonen)}}{\text{Strahlung}}}{\uparrow}}{P_r}+\underset{\underset{\text{Vakuum}}{\uparrow}}{P_V}
		\end{align*}
		Schreibe $\rho_V=\frac{1}{8\pi G}$ und $\rho=\rho_m+\rho_r, P=P_R P_V=-\rho_Vc^2$
		\begin{align*}
			\Rightarrow \frac{\ddot{a}}{a}&=-\frac{4\pi G}{3}\left(\rho+\frac{3P}{c^2}+\frac{\Lambda}{8\pi G}-\frac{3}{c^2}\left(\frac{\Lambda}{8\pi G}\right)\cdot c^2\right)\\
			&=-\frac{4\pi G}{3}\left(\rho + \frac{3P}{c^2}\right)+\frac{\Lambda}{3}
		\end{align*}
	\item Und schließlich in der Energieerhaltung:
		\begin{align*}
			\left(\frac{\dot{a}}{a}\right)^2&=\frac{8\pi G}{3}\left(\rho_m+\rho_r+\frac{\Lambda}{8\pi G}\right)-\frac{Kc^2}{a^2}\\
			&=\frac{8\pi G}{3}\rho-\frac{Kc^2}{a^2}+\frac{\Lambda}{3} \Rightarrow \text{(F1) und (F2)} \square
		\end{align*}
		N.B: Wert von $\Lambda$ lässt sich bisher nicht aus mikroskopischen Theorien herleiten.
		\begin{itemize}
			\item Eines der großen Rätsel der heutigen Physik
		\end{itemize}
\end{itemize}
\subsubsection{Diskussion der Expansionsgleichungen}
\begin{itemize}
	\item Entwicklung der kosmischen Dichte (folgt aus dem 1. Hauptsatz)
		\begin{itemize}[label={$\cdot$}]
			\item "`Staub"': $P_m=0\Rightarrow \rho_m(t)\sim a(t)^{-3}$
			\item "`Strahlung"': $P_r=\frac{1}{3}\rho_rc^2\Rightarrow \frac{d}{dt}\left(\rho_ra^3\right)=-\frac{1}{3}\rho_r\frac{da^3}{dt}$
				\begin{equation*}
					\dot{rho}_ra^3=-4a^2\dot{a}\rho_r \Rightarrow \frac{\dot{\rho}_r}{\rho_r}=-4\frac{\dot{a}}{a}\Rightarrow \rho_r(t)\sim a(t)^{-4}
				\end{equation*}
				Vakuum: $\rho_V=cnst.$\\
				\begin{equation*}
					\Rightarrow \rho_m(t)=\rho_{m,0}a(t)^{-3},\rho_r(t)=\rho_{r,0}a^{-4}(t),\rho_V(t)=\rho_V
				\end{equation*}
		\end{itemize}
	\item Grund für $\rho_r\sim a^{-4}$: Nicht nur die Anzahldichte der Photonen nimmt ab (mit $a^{-3}(t)$), sondern auch ihre Energie (Rotverschiebung)
		\begin{equation*}
			E=h\cdot\nu = h\cdot\frac{c}{\lambda}\qquad\text{ mit }\qquad\lambda\sim a(t)\Rightarrow \rho_r(t)\sim a^{-3}(t)\cdot a^{-1}(t)=a^{-4}(t)
		\end{equation*}
\end{itemize}

\begin{itemize}
	\item Falls $\rho_V\neq 0$ wird die Vakuumdichte ab einem gewissen Zeitpunkt dominant!
	\item dimensionslose Dichteparameter:
		\begin{equation*}
			\Omega_m :=\frac{\rho_{m,0}}{\rho_c},\quad\Omega_r :=\frac{\rho_{r,0}}{\rho_c},\quad\Omega_\Lambda :=\frac{\Lambda}{3H_0^2}
		\end{equation*}
		mit der kritischen Dichte $\rho_c=\frac{3H_0^2}{8\pi G}$
	\item heutige Werte:
		\begin{itemize}[label={}]
			\item Staub:
				\begin{itemize}[label={}]
					\item Galaxien (inklusive ihrer dunklen Halos): $\Omega_m\gtrsim\num{0.02}$
					\item Galaxienhaufen $\Omega_m\gtrsim\num{0.1}$
					\item Kosmologie $\Omega_m\sim\num{0.3}$
				\end{itemize}
			\item Strahlung: Photonen der CMB + Neutrinos aus dem frühen Universum $\Omega_r\sim\num{4.2e-5}\cdot \underset{\underset{\approx\num{0.72}}{\rotatebox{90}{=}}}{h^{-2}}$
			\item Vakuum: $\Omega_\Lambda\sim\num{0.7}$
		\end{itemize}
	\item Da $H(t)=\frac{\dot{a}(t)}{a(t)}$ und $\rho=\rho_{m,0}\cdot a^{-3}(t)+\rho_{r,0}\cdot a^{-4}(t)$
		\begin{itemize}
			\item $H^2(t)=H_0^2\left[a^{-4}(t)\cdot\Omega_r+a^{-3}(t)\Omega_m-a^{-2}(t)\frac{K\cdot c^2}{H_0^2}+\Omega_\Lambda\right]$
		\end{itemize}
	\item Für $t=t_0$ (heute) mit $a(t_0)=1$ ergibt sich (mit $H(t_0)=H_0$):
		\begin{equation*}
			K=\left(\frac{H_0}{c}\right)^2\cdot\big(\Omega_m+\underset{\underset{\text{(für $t=t_0$)}}{\text{vernachlässigbar}}}{\underbrace{\Omega_r}}+\Omega_\Lambda -1\big)
		\end{equation*}
		und schließlich:
		\begin{equation*}
			\left(\frac{\dot{a}}{a}\right)^2=H^2(t)=H_0^2\left(a^{-4}(t)\cdot\Omega_r+a^{-3}(t)\cdot\Omega_m+a^{-2}(t)\cdot (1-\Omega_m-\Omega_\Lambda)+\Omega_\Lambda\right)\qquad (\ast)
		\end{equation*}
\end{itemize}

\subsection{Thermische Geschichte des Universums}
\begin{itemize}
	\item wegen $T\propto 1+z$ war das Universum früher heißer:
		\begin{align*}
			z&=\num{0} \text{ (heute)} & T&\approx\SI{2.7}{\K}\\
			z&=\num{1100} & T&=\SI{3000}{\K}\\
			z&=10^9 & T&\sim\SI{3e9}{\K} \text{ heißer als das innere von Sternen}
		\end{align*}
		\begin{itemize}
			\item energetische Prozesse wie z.B. Kernfusion im frühen Universum\vspace{1mm}\\
				\textbf{\underline{\smash{Ziel}}}: Extrapolation der physikalischen Gesetze für das frühe Universum, um dieses zu beschreiben (Annahme Naturgesetze haben sich zeitlich nicht geändert)
		\end{itemize}
	\item Vorbemerkungen:
		\begin{itemize}[label={\textbullet}]
			\item $\SI{1}{\eV}\approx\SI{1.1905e4}{k_B\K}$
			\item Anzahldichte und Energieverteilung von Teilchen im thermodynamischen und chemischen Gleichgewicht hängt allein von ihrer Temperatur ab
			\item Die Elementarteilchenphysik ist für Energien $\lesssim\SI{1}{G\eV}$ sehr gut verstanden und wird durch die Quantenmechanik beschrieben
			\item Notwendige Bedingung zum Erreichen eines chemischen Gleichgewichts ist die Möglichkeit der Paarerzeugung- und vernichtung, z.B. $2\gamma\mapsto e^++e^-$
		\end{itemize}
\end{itemize}
\subsubsection{Expansion in strahlungsdominierter Phase}
\begin{itemize}
	\item Für $z>>z_\text{eq}=a^{-1}_\text{eq}-1\approx\SI{23900}{\Omega_mh^2}$ ist die Energiedichte der Strahlung $\rho_r\sim T^4\Leftrightarrow \rho\sim a^{-4}$ dominant.
	\item $(F1)$ wird zu:
		\begin{equation*}
			\left(\frac{\dot{a}}{a}\right)^2=\frac{8\pi G}{3}\cdot\text{const}\cdot a^{-4} + \text{ vernachlässigbare Terme}
		\end{equation*}
	\item Lösung durch Ansatz $a(t)\sim t^\beta \Rightarrow t^{-2}\sim t^{-4\beta} \Rightarrow \beta=\frac{1}{2}$
		\begin{equation*}
			\Rightarrow a(t)\sim t^\frac{1}{2}, t\sim T^{-2}, t\sim\rho^{-\frac{1}{2}}
		\end{equation*}
		wobei die Proportionalitätskonstante von der Anzahl der relativistischen Teilchen abhängt
	\item Unter der Annahme des thermodynamischen Gleichgewichts (Hypothese!) ist diese Anzahl bekannt $\Rightarrow$ Verlauf der frühen Expansion komplett bekannt
\end{itemize}
\subsubsection{Entkopplung der Neutrinos}
\begin{itemize}
	\item Beginn der Betrachtung bei $T=10^{12}\si{\K}\hat{=}\SI{100}{M\eV}$\\
		Proton, $m_p\approx\SI{938.3}{\frac{M\eV}{c^2}}$\\
		Neutron, $m_n\approx\SI{936.6}{\frac{M\eV}{c^2}}$\\
		Elektron, $m_e\approx\SI{0.511}{\frac{M\eV}{c^2}}$\\
		Myon, $m_\mu\approx\SI{140}{\frac{M\eV}{c^2}}$
	\item Protonen und Neutronen (Baryonen) sind zu schwer um bei der betrachteten Temperatur erzeugt zu werden, sie müssen vorher erzeugt worden sein
	\item \textbf{Alle} Baryonen, die es heute gibt, müssen damals schon vorhanden gewesen sein
	\item Auch Paare von Myonen können nicht mehr effizient in der Reaktion $\gamma +\gamma \mapsto \mu^++\mu^-$ erzeugt werden\\
		($\mu^\pm$ sind instabil mit $\tau\sim\SI{2}{\mu\s}$)
		\begin{itemize}
			\item relativistische Teilchensorte, die zur Strahlungsdichte beitragen:
				\begin{itemize}[label={$\cdot$}]
					\item Elektronen/Positronen $e^-/e^+$
					\item Photonen $\gamma$
					\item Neutrinos/Antineutrinos $\nu/\bar{\nu}$ mit $m_\nu<\SI{1}{\eV}\approx 0$ (Grenze aus der Kosmologie)
				\end{itemize}
			\item nichtrelativistische Teilchen, die zu $\rho_m$ beitragen:
				\begin{itemize}[label={$\cdot$}]
					\item Protonen/Neutronen $p/n$
					\item Konstituenten der dunklen Materie WIMPs (?) mit Masse $\gtrsim\SI{100}{G\eV}$
				\end{itemize}
		\end{itemize}
	\item Die Reaktionen:
		\begin{align*}
			e^\pm+\gamma&\leftrightarrow e^\pm+\gamma & &\text{Comptonstreuung}\\
			e^++e^-&\leftrightarrow \gamma+\gamma & &\text{Paarerzeugung und Annihilation}\\
			\nu+\bar{\nu}&\leftrightarrow e^++e^- & &\text{Neutrino-Antineutrinostreuung}\\
			\nu+e^\pm&\leftrightarrow\nu+e^\pm & &\text{Neutrino-Elektro-Streuung}
		\end{align*}
		halten die relativistischen Teilchen im Gleichgewicht.
	\item Energiedichte zu dieser Zeit:
		\begin{equation*}
			\rho=\rho_r=\num{10.75}\frac{\pi^2}{30}\cdot\frac{\left(k_BT\right)^4}{\hbar\cdot c^3} \Rightarrow t=\SI{0.3}{\s}\cdot\left(\frac{T}{\SI{1}{M\eV}}\right)^{-2}
		\end{equation*}
	\item Damit die Teilchen im Gleichgewicht bleiben, müssen die obigen Reaktionen genügend häufig ablaufen, d.h. die mittlere Zeit zwischen zwei Reaktionen muss sehr viel kürzer sein als die Zeitskala, auf der sich die Gleichgewichtsbedingungen aufrund der Expansion ändern (Reaktionsraten müssen größer als $H(t)$ sein)
	\item insbesondere Neutrinos interagieren nur per schwacher Wechselwirkung
	\item Reaktionsrate: $\Gamma\sim n\sigma$\\
		Anzahldichte: $n\sim a^{-3}\sim t^{-\frac{3}{2}}$\\
		Wirkungsquerschnitt für Neutrinos: $\sigma\sim E^2\sim T^2\sim a^{-2}$
		\begin{itemize}
			\item $\Gamma\sim n\sigma\sim a^{-3}\cdot a^{-2}=a^{-5}\sim t^{-\frac{5}{2}}\sim T^5$
		\end{itemize}
	\item Zu Vergleichen mit Expansionsrate $\frac{\dot{a}}{a}=H(t)\sim t^{-1}\sim T^2$
	\item Aus $\sigma$ der schwachen Wechselwirkung kann man den Zeitpunkt bzw. die Temperatur des Übergangs berechnen:
		\begin{equation*}
			\frac{\Gamma}{H}\sim\left(\frac{T^3}{\SI{1.6e10}{\K}}\right)
		\end{equation*}
		\begin{itemize}
			\item Für $T\lesssim 10^{10} \si{\K}$ sind die Neutrinos nicht mehr mit den anderen Teilchen im Gleichgewicht. Nach diesem zeitpunkt ($t=\SI{1}{\s}$) bewegen sie sich ohne weitere Wechselwirkung bis zum heutigen Tage.\\
				heute: $n_\nu=\SI{113}{\cm^{-3}}$ für jede Neutrinoart
				\begin{equation*}
					T_\nu=\SI{1.9}{\K} \quad (\text{s.u.})
				\end{equation*}
				\begin{itemize}
					\item[$\leadsto$] leider sehr schwach nachweisbar
				\end{itemize}
		\end{itemize}
\end{itemize}
\subsubsection{Paarvernichtung}
\begin{itemize}
	\item Für $T\lesssim\SI{5e9}{\K}$ bzw. $k_BT\lesssim \SI{500}{k\eV}$ dominert die Annihilation $e^++e^-\to 2\gamma$ über die Paarerzeugung.
		\begin{itemize}
			\item Dichte der $e^+e^--Paare$ nimmer sehr schnell ab
			\item Photonengas wird erhitzt (Neutrinos nicht, da sie bereits entkoppelt sind)
				\begin{equation*}
					T_\gamma=\left(\frac{11}{4}\right)^\frac{1}{3}\cdot \underset{\underset{\text{vor Annihilation}}{\uparrow}}{T}=\left(\frac{11}{4}\right)^\frac{1}{3}\underset{\underset{\underset{\text{entkoppelten Neutrinos}}{\text{Temperatur der}}}{\uparrow}}{T_\nu}\to \text{ siehe Übung}
				\end{equation*}
		\end{itemize}
	\item Nach der Annihilation gilt das Expansionsgesetz $t=\SI{0.55}{\s}\left(\frac{T}{\SI{1}{M\eV}}\right)^{-2}$ und das Verhältnis von Baryonendichte und Photonendichte bleibt Konstant: $\eta:=\left(\frac{n_b}{n_\gamma}\right)=\SI{2.74e-8}{\underset{=\num{0.02}}{\Omega_bh^2}}$
	\item Nach der Annihilation sind \textbf{fast} alle Elektronen zerstrahlt, aber eine kleine Zahl $n_e=n_p$ muss übrig bleiben, damit das Universum elektrisch neutral bleibt $\Rightarrow \frac{n_{e^-}}{n_\gamma}=\num{0.8}\eta$ ($\eta$ beinhaltet Protonen und Neutronen)
\end{itemize}
\subsubsection{Primordiale Nukleosynthese}
\begin{itemize}
	\item Entstehung von Atomkernen aus $p$ und $n$ im frühen Universum
	\item wichtigste Reaktionen im chemischen Gleichgewicht:
		\begin{equation*}
			p+e^-\leftrightarrow n+\nu_e,\quad p+\bar{\nu}_e\leftrightarrow n+e^+,\quad n\to p+e^-+\bar{\nu}_e
		\end{equation*}
		Zerfallszeit des freien Neutrons: $\tau_n=\SI{887}{\s}$
	\item im thermischen Gleichgewicht: $\frac{n_n}{n_p}=\left(\frac{m_n}{m_p}\right)^\frac{3}{2}\cdot e^{-\frac{\Delta m c^2}{k_BT}}\qquad (\ast)$\\
		mit $\Delta m=m_n-m_p=\SI{1.293}{\frac{M\eV}{c^2}}$
	\item Gleichgewichts-Reaktionen werden selten, nachdem die Neutrinos ausgefroren sind. Dies geschieht bei $T\approx\SI{0.8}{M\eV}$
		\begin{itemize}
			\item $\frac{n_n}{n_p}\approx e^{-\frac{\SI{1.3}{M\eV}}{\SI{0.8}{M\eV}}}\approx \num{0.2}$
		\end{itemize}
	\item Nach der Entkopplung von $n$ und $p$ wird ihr Verhältnis nicht mehr durch $(\ast)$ beschrieben, sondern nur noch durch den Zerfall der freien Neutronen auf der Zeitskala $\tau_n$ modifiziert $\Rightarrow $ heutige Neutronen wurden schnell in Atomkerne gebunden
\end{itemize}

\begin{figure}[H]
	\centering
	\begin{tikzpicture}
		\draw[->] (0,-1)node[below]{$t=\SI{0}{\s}$}--++(0,2)++(0,-1)--++(2,0)++(0,-0.5)node[below]{\begin{minipage}{2cm} Entkopplung von $p$ und $n$\end{minipage}}--++(0,1)node[above]{$t=\SI{1}{\mu\s}$}++(0,-0.5)--++(3,0)++(0,-0.5)node[below]{\begin{minipage}{2cm} Entkopplung von Neutrinos\end{minipage}}--++(0,1)node[above]{$t=\SI{1}{\s}$}++(0,-0.5)--++(3,0)++(0,-0.5)node[below]{\begin{minipage}{2cm} Paarvernichtung $e^++e^-\to 2\gamma$\end{minipage}}--++(0,1)node[above]{$t=\SI{10}{\s}$}++(0,-0.5)--++(4,0)++(0,-0.5)node[below]{\begin{minipage}{2cm} primordiale Nukleosynthese\end{minipage}}--++(0,1)node[above]{$t=\SI{3}{\min}$}++(0,-0.5)--++(3,0)node[below]{$t$};
	\end{tikzpicture}
\end{figure}
\begin{enumerate}[label={$(\arabic*)$}]
	\item Deuteriumbildung $\text{D}=^2\text{H}$
		\begin{equation*}
			p+n\to D+\gamma \text{ mit Bindungsenergie $E_b=\SI{2.225}{M\eV}$}
		\end{equation*}
		\begin{itemize}
			\item Aber: Erst wenn $k_BT\approx E_b$ kann Deuterium in größeren Mengen vorhanden sein, da bei höheren Temperaturen Photodissoziation dominiert
			\item Dies geschieht bei $T\sim \SI{8e8}{\K}$ bzw. $t=\SI{3}{\min}$
			\item Zu diesem Zeitpunkt beträgt das Verhältnis $\frac{n_n}{n_p}\approx\frac{1}{7}$, danach werden praktisch alle Neutronen in $D$ gebunden
		\end{itemize}
	\item Helium-Häufigkeit:
		\begin{align*}
			&\begin{aligned}&\left\{\begin{aligned}\text{D}+\text{D}&\to\text{\textsuperscript{3}He}+n\\
			\text{\textsuperscript{3}He}+\text{D}&\to\text{\textsuperscript{4}He}+p\end{aligned}\right.\\
			&\left\{\begin{aligned}\text{D}+\text{D}&\to\text{\textsuperscript{3}H}+p\\
			\text{\textsuperscript{3}H}+\text{D}&\to\text{\textsuperscript{4}He}+n\end{aligned}\right.\end{aligned} & &\begin{aligned}\text{insgesamt } 3D&\to\text{\textsuperscript{4}He}+p+n \\ E_b&\approx\SI{28}{M\eV}\end{aligned}
		\end{align*}
		Praktisch alle vorhandenen Neutronen werden so in \textsuperscript{4}He gebunden. ($t=\SI{3}{\min}$)
		\begin{itemize}
			\item Anzahldichte von \textsuperscript{4}He:
				\begin{align*}
					n_\text{He}&=\frac{1}{2}n_n \text{ (da e Neutronen in jedem \textsuperscript{4}He)}\\
					n_\text{H}&=n_e-n \text{Anzahldichte von Protonen nach Bildung von \textsuperscript{4}He}
				\end{align*}
				\begin{itemize}
					\item Massenanteil von \textsuperscript{4}He an der Baryonendichte:
						\begin{equation*}
							y=\frac{4n_\text{He}}{4n_\text{He}+n_\text{H}}=\frac{2n_n}{n_p+n_n}=\frac{2\cdot\frac{n_n}{n_p}}{1+\left(\frac{n_n}{n_p}\right)}\approx\num{0.25}
						\end{equation*}
						Etwa $\frac{1}{4}$ der baryonischen Materie im Universum sollte als \textsuperscript{4}He gebunden sein! Dies ist eine robuste Vorhersage der Big-Bang-Modelle und in Übereinstimmung mit Beobachtung VI, Abschnitt (4.1)!
				\end{itemize}
		\end{itemize}
	\item Der Baryonenanteil im Universum
		\begin{figure}[H]
			\centering
			\begin{tikzpicture}
				\node[name=p,draw,rectangle] at (0,0){$p$};
				\node[name=d,draw,rectangle] at (2,0){D};
				\node[name=H3,draw,rectangle] at (4,0){\textsuperscript{3}H};
				\node[name=n,draw,rectangle] at (2,-2){$n$};
				\node[name=he3,draw,rectangle] at (2,2){\textsuperscript{3}He};
				\node[name=he4,draw,rectangle] at (2,4){\textsuperscript{4}He};
				\node[name=li5,draw,rectangle] at (6,6){\textsuperscript{7}Li};
			\end{tikzpicture}
		\end{figure}
		\begin{itemize}
			\item \textsuperscript{5}Li, \textsuperscript{8}Be keine stabilen Kerne
				\begin{align*}
					&\Rightarrow \text{\textsuperscript{4}He}+\text{\textsuperscript{4}He}\to\text{ instabil}\\
					&\Rightarrow \text{\textsuperscript{4}He}+p\to\text{ instabil}
				\end{align*}
				\begin{itemize}
					\item Nach 4 Minuten: $\SI{25}{\%}$ \textsuperscript{4}He, $\SI{75}{\%}$ $p$ und Spuren von D, \textsuperscript{3}He, \textsuperscript{7}Li
				\end{itemize}
			\item Dichte von \textsuperscript{4}He und D hängt von $\Omega_b$ ab:
				\begin{itemize}[label={\textbullet}]
					\item je größer $\Omega_b$, desto größer $\eta$, desto früher kann sich D bilden, desto größer $\frac{n_n}{n_p}$ und $Y$
					\item je größer $\Omega_b$, desto größer ist $n_b$ und desto effektiver die Umwandlung von D in \textsuperscript{4}He $\Rightarrow$ weniger D
				\end{itemize}
				\begin{figure}[H]
					\centering
					\begin{tikzpicture}
						\draw[->] (0,0)--++(2,0)++(0,-0.5)--++(0,1)node[above]{$\num{0.005}$}++(0,-0.5)--++(3,0)++(0,-0.5)--++(0,1)node[above]{$\num{0.01}$}++(0,-0.5)--++(2,0)++(0,-0.5)--++(0,1)node[above]{$\num{0.02}$}++(0,-0.5)--++(2,0)++(0,-0.5)--++(0,1)node[above]{$\num{0.03}$}++(0,-0.5)--++(1,0)node[below right]{$\Omega_bh^2$};
						\draw (0,0)--++(0,-0.5)++(-0.5,0)node[left]{$\num{0.21}$}--++(1,0)++(-0.5,0)--++(0,-1)++(-1,0)node{$Y$}++(1,0)--++(0,-1)++(-0.5,0)node[left]{$\num{0.22}$}--++(1,0)++(-0.5,0)--++(0,-0.5)++(0,-1)--++(0,-0.5)++(-0.5,0)node[left]{$10^{-4}$}--++(1,0)++(-0.5,0)--++(0,-1)++(-1,0)node{$\frac{n}{n_\text{H}}$}++(1,0)--++(0,-1)++(-0.5,0)node[left]{$10^{-5}$}--++(1,0)++(-0.5,0)--++(0,-0.5)++(0,-1)--++(0,-0.5)++(-0.5,0)node[left]{$10^{-9}$}--++(1,0)++(-0.5,0)--++(0,-2)++(-0.5,0)node[left]{$10^{-10}$}--++(1,0)++(-0.5,0)--++(0,-0.5);
						\draw[densely dashed] (6.5,0)--(6.5,-11)(9,-11)--(9,0);
						\fill[pattern=north east lines] (6.55,0)rectangle(8.95,-11);
						\draw[thick] (4,-0.75)rectangle(9.5,-2.25);
						\draw[domain=0:9.5,samples=50] plot({\x},{-3*exp(ln(0.75/3)/9.5*\x)})(3,-1.5)node{\textsuperscript{4}He};
						\draw[thick] (6.5,-5.75)rectangle(9,-7);
						\draw (2,-4)--(9,-7)node[midway,above right]{D};
						\draw[thick] (1.5,-8)rectangle(10,-10.5);
						\fill[draw=black,pattern=north west lines] (1,-8.75) parabola bend (3,-9.75) (10,-8) -- (10,-8.5) parabola bend (3,-10.25) (1,-9.25) -- cycle;
						\node[above left] at (3,-9.75){\textsuperscript{7}Li};
					\end{tikzpicture}
				\end{figure}
			\item bemerkenswerte Übereinstimmung zwischen Theorie und Messungen für die drei Kerne
			\item bisher beste Messung für D:
				\begin{equation*}
					\num{0.012}\leq\Omega_bh^2\leq\num{0.019}
				\end{equation*}
				\begin{itemize}
					\item $\Omega_b\approx\num{0.03}-\num{0.04}$
				\end{itemize}
			\item Aber $\Omega_m>\num{0.1}\Rightarrow$ größter Teil der Materie ist nicht-baryonische dunkle Materie!
			\item Neutrinos ein Kandidat für dunkle Materie?\\
				$\to$ siehe Übung
			\item bester Kandidat als Knstituent für dunkle Materie: WIMPs (=weakly interacting massive particles)
			\item experimenteller Nachweis steht (noch?) aus
		\end{itemize}
\end{enumerate}
\subsubsection{Rekombination}
\begin{itemize}
	\item Nach ca. $\SI{3}{\min}$ ist die primordiale Nukleosynthese abgeschlossen $T\sim\SI{8e8}{\K}$
	\item Bei $z\approx z_{eq}\approx\SI{23900}{\Omega_mh^2}$ beginnt die Materie (d.h. der Staub) zu dominieren. Dhaer wird (F1) zu:
		\begin{equation*}
			H^2(t)\approx H_0^2\frac{Omega_m}{a^3}
		\end{equation*}
	\item Ansatz $a(t)\sim t^\beta$ ergibt $\beta=\frac{2}{3}$ und damit:
		\begin{equation*}
			a(t)=\left(\frac{3}{2}\sqrt{\Omega_m}\cdot H_0\cdot t\right)^\frac{2}{3}
		\end{equation*}
		für $a_{eq}<<a<<1$
	\item nächste wichtige Schwelle: $T\sim\SI{3000}{\K} (\hat{=} t\sim\SI{3e5}{a})$\\
		Rekombination $p+e^-\to$ neutraler Wasserstoff
\end{itemize}

\subsection{Erfolge und Probleme des Standardmodells}
\begin{itemize}
	\item Standardmodell des Friedmann-Lema\^itre-Universums hat viele beeindruckende Erfolge vorzuweisen
	\item Vorhersagen:
		\begin{itemize}[label={-}]
			\item Hubblesches Gesetz: Absorption von Strahlung von Quellen mit Rotverschiebung $z$ erfolgt nur bei $z'<z$\\
				(experimentell bisher kein Gegenbeispiel gefunden)
			\item Wenig prozessiertes (d.h. metallarmes) Gas hat einen Heliumanteil von $\SI{25}{\%}$ (passt hervorragend zu den Beobachtungen, vgl. (IV) Abschnitt 4.1)
			\item $\exists$ Mikrowellenhintergrund
			\item Es sagt die richtige Anzahl von Neutrino-Familien vorher ($N_\nu=3$), wie durch den Zerfall des Z-Boson (CERN) bestätigt wurde
				\begin{equation*}
					t\propto\frac{1}{\sqrt{\rho}}\text{ im strahlungsdominierten Universum}
				\end{equation*}
				Falls $N_\nu>3 \Rightarrow$ Expansion verläuft schneller
				\begin{itemize}[label={$\Rightarrow$}]
					\item weniger Zeit bis zum Abkühlen auf $T_D$
					\item mehr freie Neutronen
					\item höherer \isotope{4}{He}-Gehalt
				\end{itemize}
			\item Neutrinomassen sind $\lesssim\SI{1}{\eV}$ (inzwischen $\underset{\lesssim\SI{1.1}{\eV}\text{ KATRIN}}{\lesssim\SI{0.1}{\eV}}$)
		\end{itemize}
	\item Nicht erklärt:
		\begin{itemize}[label={\textbullet}]
			\item Anfangswerte bei $t\sim\SI{1}{\s}$
			\item Homogenität und Isotropie
		\end{itemize}
		Welche physikalischen Prozesse liegen dahinter?
	\item Insbesondere zwei Probleme des Standardmodells:
		\begin{enumerate}[label={(\arabic*)}]
			\item \textbf{Horizontproblem}
				\begin{itemize}[label={$\to$}]
					\item Im Zeitintervall $dt$ legt das Licht die Strecke $dr=cdt$ zurück
						\begin{itemize}[label={$\Rightarrow$}]
							\item mitbewegtes Längenintervall $dx=\frac{cdt}{a(t)}$
								\begin{align*}
									\Rightarrow\ \underset{\underset{\underset{\underset{\underset{\text{bis zur Rotverschiebung } z}{\text{Entfernung von Urknall}}}{\text{Licht zurückgelegte}}}{\text{mitbewegte vom}}}{\uparrow}}{r_H(z)}&=\int\limits_0^t\frac{cdt'}{a(t')}=\int\limits_0^{(1+z)^{-1}}\frac{cda}{a^2 H(a)},\quad dt=\frac{da}{\dot{a}}=\frac{da}{a\cdot H}\\
									\Rightarrow r_H&=\begin{cases} \frac{2c}{H_0}\cdot\sqrt{\frac{1}{(1+z)\cdot\Omega_m}} & \text{für }0<<z<<z_\text{eq}\\ \frac{c}{H_0}\cdot\frac{1}{\sqrt{\Omega_r}(1+z)} & \text{für }z_\text{eq}<<z\end{cases}
								\end{align*}
						\end{itemize}
					\item Zum Zeitpunkt der Rekombination $z\approx z_\text{eq}\sim 1000$\\
						Eigenlänge $r_\text{eq}=a\cdot r_H=2\frac{c}{H_0}\frac{1}{(1+z_\text{eq})^{\frac{3}{2}}\sqrt{\Omega_m}}$\\
						$\hat{=}$ Winkel am Himmel: $\Omega_{H,rec}=\SI{1}{\degree}\cdot\left(\frac{\Omega_m}{\num{0.3}}\right)^{\frac{1}{2}}\cdot\left(\frac{z_{rec}}{1000}\right)^{-\frac{1}{2}}$
						\begin{itemize}[label={$\Rightarrow$}]
							\item CMB ist bis auf kleine Anisotropien der relativen Amplitude $\sim 10^{-5}$ isotrop $\lightning$ (?)\\
								$\to$ Universum isotrop und homogen?
						\end{itemize}
				\end{itemize}
			\item \textbf{Krümmung}
				\begin{itemize}[label={$\to$}]
					\item totaler Dichteparameter für eine beliebige Rotverschiebung
						\begin{equation*}
							\Omega_0(z)=\frac{\rho_m(z)+\rho_r(z)+\rho_v}{\rho_c(z)}
						\end{equation*}
						mit kritischer Dichte $\rho_c(z)=\frac{3H^2(z)}{8\pi G}$
						\begin{equation*}
							\Rightarrow \Omega_0(z)=\left(\frac{H_0}{H}\right)^2\left(\frac{\Omega_m}{a^3}+\frac{\Omega_r}{a^4}+\Omega_\Lambda\right)
						\end{equation*}
						und mit den Lema\^itre-Friedmann-Gleichungen:
						\begin{equation*}
							1-\Omega_0(z)=F(z)(1-\Omega(0))
						\end{equation*}
						mit $F(z)=\left(\frac{H_0}{a\cdot H(a)}\right)^2 >0$\\
						$\Omega_0$: totaler Dichteparameter heute:
						\begin{itemize}[label={\textbullet}]
							\item falls $\Omega(0)=1 \Rightarrow \Omega_0(z)=1 \forall z$
							\item falls $\Omega(0)<1 \Rightarrow \Omega_0(z)>1$ bzw. $\Omega(0)<1\Rightarrow \Omega(z)<1$ da $\Omega(z)-1\sim $ Krümmung $K\Rightarrow\forall z$ bleibt $K$ erhalten.\\
								$F(z)$ charakterisiert die Abweichung von einem flachen Universum
						\end{itemize}
					\item Beispiel: für strahlungsdominiertes Universum
						\begin{equation*}
							F(z)\approx\left[\Omega_r\cdot(1+z)^2\right]^{-1}\underset{\underset{\begin{minipage}{3cm}\begin{tiny} bei $z\sim 10^{10}$ (Ausfrieren der Neutrinos)\end{tiny}\end{minipage}}{\uparrow}}{\sim} 10^{-15}
						\end{equation*}
					\item Aus Beobachtungen (CMB) wissen wir, dass:
						\begin{equation*}
							\num{0.97}<\Omega(0)<\num{1.04}\Rightarrow \left|\Omega(0)-1\right|\lesssim\num{0.04}\Rightarrow\left|\Omega_0(10^10)-1\right|\lesssim 10^{-15}
						\end{equation*}
						\begin{itemize}[label={$\Rightarrow$}]
							\item Flachheitsproblem: Damit der totale Dichteparameter heute von der Größenordnung 1 sein kann, muss er zu sehr frühen Zeiten extrem nahe bei 1 gewesen sein!
						\end{itemize}
					\item Wie war eine solch präzise Feinabstimmung dieser Größe möglich? (sehr spezielle Anfangsbedingungen bei $t=\SI{1}{\s}$) $\to$ antropisches Prinzip? $\to$ unbefriedigend
						\begin{itemize}[label={$\Rightarrow$}]
							\item (spekulative) Erweiterung des Standardmodells: \textbf{Inflation} (A. Guth, 1980)
								\begin{itemize}[label={$\to$}]
									\item Teilchenphysik erwartet neue Phänomene (GUT = grand unified theories) bei $T\sim 10^{14} \si{G\eV}\hat{=}t\sim 10^{-34}\si{\s}$
									\item Szenario der Inflation: $\Omega_\Lambda$ bei sehr frühen Zeiten viel größer als heute
										\begin{align*}
											\Rightarrow \frac{\dot{a}}{a}&\approx\sqrt{\frac{\Lambda}{3}} \Rightarrow \text{ exponentielle Expansion des Universums}\\
											\Leftrightarrow a(t)\propto e^{\sqrt{\frac{\Lambda}{3}}\cdot t}
										\end{align*}
									\item Annahme: Nach einer Phase der Expansion kommt es zu einem Phasenübergang, bei dem die Vakuumenergiedichte in normale Materie und Strahlung umgewandelt wirdq
										\begin{align*}
											\text{zu (1): } r_M(a_1,a_2)&\sim \Omega_\Lambda^{-\frac{1}{2}}\int\limits_{a_1}^{a_2}\frac{cda}{a}>>1 & &\text{falls } a_1<<1
										\end{align*}
										$\Rightarrow$ Das gesamte beobachtbare Universum war in Kausalem Kontakt $\Rightarrow$ Homogenität und Isotropie.\\
									\item[] zu (2):
									\item Durch die gewaltige Ausdehnung wird jede ursprüngliche Krümmung "weggeglättet":
										\begin{equation*}
											H(t)=\sqrt{\frac{\Lambda}{3}}\Rightarrow\Omega_\Lambda=\frac{\Lambda}{3H^2}=1\Rightarrow \Omega_0(1)
										\end{equation*}
										$\Rightarrow$ Das Universum ist flach und auch heute gilt noch sehr genau $\Omega_0=1$
										\begin{figure}[H]
											\centering
											\begin{tikzpicture}
												\draw (0,0)circle(1cm);
												\draw[->] (1.5,0)--(2.5,0);
												\draw (3.5,-1)arc(-90:90:1cm);
												\draw[->] (5,0)--(6,0);
												\draw ({5*cos(-10)+2},{5*sin(-10)})arc(-10:10:5cm);
											\end{tikzpicture}
										\end{figure}
									\item Weiterhin bietet Inflation eine Erklärung für den Ursprung der Dichteschwankungen im Universum (Keime der Strukturbildung): Quantenfluktuationen (Quantengravitation)
										\begin{figure}[H]
											\centering
											\begin{tikzpicture}[]
												\draw[->] (0,-0.5)--(0,5.5)node[above left]{$r_H/\si{\cm}$};
												\foreach \x \y in {1/10^{-60},2/10^{-40},3/10^{-20},4/1,5/10^{20}}{
													\draw (-0.1,\x)node[left]{$\y$}--(0.1,\x);
												};
												\draw[->] (-0.5,0)--(6.5,0)node[below right]{$t/\si{\s}$};
												\foreach \x \y in {1/10^{-40},2/10^{-30},3/10^{-20},4/10^{-10},5/1,6/\cdots}{
													\draw (\x,-0.1)node[below]{$\y$}--(\x,0.1);
												}
												\draw[thick,domain=1:3,samples=50] plot({\x},{0.1*\x+2.5+1.5*tanh(5*(\x-1.7))});
												\draw[thick,domain=3:6,samples=2] plot({\x},{0.1*\x+4});
												\draw[thick,domain=0:1,samples=2] plot({\x},{0.1*\x+1});
												\draw[thick,domain=0:6,samples=50,blue!50!black] plot({\x},{0.1*\x+4});
												\draw[densely dashed] (1.5,0)--(1.5,5.5)(2,5.5)--(2,0);
												\fill[pattern=north west lines] (1.55,0)--(1.55,5.5)--(1.95,5.5)--(1.95,0)--cycle;
												\node[name=s] at (4,5){Standardmodell};
												\node[name=i] at (4,2){Inflationstheorie};
												\draw[->,shorten >= 3pt, >=stealth,very thick] (i)--(1.7,2.5);
												\draw[->,shorten >= 3pt, >=stealth, very thick,blue!50!black] (s)--(1,4.1);
												\draw[->,shorten >= 3pt, >=stealth, very thick,blue!50!black] (s)--(5,4.5);
											\end{tikzpicture}\\
											Ausdehung während Inflation: Faktor $\sim 10^{40}$ von $ct_i\sim 10^{-24}\si{\cm}$ auf $ct_s=10^{18}\si{\cm}$ von $t_i\sim 10^{-34}\si{\s}$ auf $t_s\sim 10^{-32}\si{\s}$ weitere "normale" kosmische Expansion: Faktor $10^{25}$ auf $10^{41}\si{\cm}$
										\end{figure}
								\end{itemize}
						\end{itemize}
				\end{itemize}
		\end{enumerate}
\end{itemize}

\section{Galaxienhaufen und -gruppen}
\begin{itemize}
	\item Milchstraße ist Mitglied der lokalen Gruppe (local group): $\num{35}$ Galaxien (+$\sim\num{20}$ zusätzlich sehr lichtschwache (gefunden mit dem SDSS)) innerhalb $\lesssim\SI{1}{M\pc}$
	\item wichtige Mitglieder:
		\begin{itemize}[label={$\cdot$}]
			\item Magellansche Wolke (LMC,SMC) sind irreguläre Galaxien
			\item 3 Spiralgalaxien:
				\begin{itemize}
					\item[] Milchstraße ($\mathcal{M}_B=-\num{20}$)
					\item[] Andromeda (M31, $\mathcal{M}_B=-\num{20}$)
					\item[] Dreiecksgalaxie (M33, $\mathcal{M}_B=-\num{18.9}$)
				\end{itemize}
		\end{itemize}
\end{itemize}
\subsection{Massenabschätzung der lokalen Gruppe}
\begin{itemize}
	\item M31 im Abstand von $D=\SI{770}{k\pc}$ ist eine der wenigen Galaxie mit einer Planverschiebung $v\approx -\SI{120}{\frac{\km}{\s}}$ zwischen den Zentren
		\begin{itemize}
			\item Kollision auf einer Zeitskala von $\SI{6e9}{\year}$
		\end{itemize}
	\item Milchstraße + M31 $\hat{=}\ \SI{90}{\%}$ der Leuchtkraft der lokalen Gruppe
		\begin{itemize}
			\item Dynamik sollte von diesen Galaxien dominiert sein (falls Massedichte $\sim$ Lichtverteilung)
			\item Abschätzung der Gesamtmasse der lokalen Gruppe:
				\begin{enumerate}[label={(\roman*)}]
					\item M31 und Milchstraße waren einander sehr nahe in der Frühzeit des Universums
					\item daraufhin haben sie sich durch die kosmologische Expansion voneinander entfernt
					\item \textbf{Aber}: Gravitation bremst relative Fluchtgeschwindigkeit ab, bis zum Stillstand $t=t_\text{max}$
					\item Für $t>t_\text{max}$ bewegen sie sich aufeinander zu
						\begin{figure}[H]
							\centering
							\begin{minipage}[l]{0.48\textwidth}
								\begin{tikzpicture}[>=stealth]
									\draw[->] (-0.5,0)--(3,0)node[below right]{$t$};
									\draw[->] (0,-0.5)--(0,3)node[above left]{$r$};
									\draw (0,0) parabola bend (1.25,2) (2.5,0);
									\draw[densely dashed] (0,2)node[left]{$r_\text{max}$}--(1.25,2)--(1.25,0)node[below]{$t_\text{max}$};
								\end{tikzpicture}
							\end{minipage}
							\begin{minipage}[r]{0.48\textwidth}
								Aus der Energieerhaltung folgt:
								\begin{equation*}
									\frac{1}{2}v^2=\frac{GM}{r}-C
								\end{equation*}
								mit $M=$ Gesamtmasse Milchstraße + M31 und $C$ zu bestimmende Integrationskonstante
							\end{minipage}
						\end{figure}
						Bei $t=t_\text{max}$ ist $r=r_\text{max}$ und $r=\delta\Rightarrow C=\frac{G}{M}{r_\text{max}}$
						\begin{equation*}
							\Rightarrow \left(\frac{dr}{dt}\right)^2=2\cdot G\cdot M\cdot\left(\frac{1}{r}-\frac{1}{r_\text{max}}\right)
						\end{equation*}
							$\text{da } v=\frac{dr}{dt}\ (\text{Radialgeschwindigkeit})$ $\text{und } r(0)=0$
						\begin{equation*}
							\Rightarrow\text{ Lösung: } t_\text{max}=\int\limits_0^{t_\text{max}} t=\int\limits_0^{r_\text{max}}\frac{dr}{\sqrt{2GM\left(\frac{1}{r}-\frac{1}{r_\text{max}}\right)}}=\frac{\pi}{2}\cdot\frac{r_\text{max}^\frac{3}{2}}{\sqrt{2GM}}
						\end{equation*}
						\begin{itemize}[label={\textbullet}]
							\item DGL ist symmetrisch bzgl. $v\to -v\Rightarrow $ Kollision bei $t=2t_\text{max}$
							\item vereinfachende Abschätzung: Relativgeschwindigkeit von heute bis zur Kollision ist Konstant, d.h:
								\begin{equation*}
									\frac{r(t_0)}{v(t_0)}=\frac{D}{V}=\frac{\SI{770}{k\pc}}{\SI{120}{\frac{\km}{\s}}}\Rightarrow 2t_\text{max}=\overset{\overset{\begin{minipage}{2.5cm} Alter des Universums\end{minipage}}{\downarrow}}{t_0}+\frac{D}{V}
								\end{equation*}
								und schließlich:
								\begin{equation*}
									\frac{1}{2}v^2=\frac{GM}{r}-\frac{1}{2}\left(\frac{\pi GM}{t_\text{max}}\right)^\frac{2}{3}
								\end{equation*}
								Mit $r=r(t_0)=D$ und $v=v(t_0)$ erhält man ($t_0\approx \SI{14}{G\year}$):
								\begin{equation*}
									M\approx\SI{3e12}{M_\odot}\Rightarrow \frac{M}{L}\sim 70\frac{M_\odot}{L_\odot}
								\end{equation*}
								\textbf{Aber}:
								\begin{equation*}
									\frac{M}{L}\sim\num{3}-\SI{5}{\frac{M_\odot}{L_\odot}} \text{ (vgl. Tabelle, Abschnitt)}
								\end{equation*}
								für S\textsubscript{b\textsubscript{c}} Spiralgalaxien
							\item[$\Rightarrow$] \textbf{Nur etwa $\SI{5}{\%}$ der gravitativen Masse der lokalen Gruppe kann direkt beobachtet ("gesehen") werden $\Rightarrow$ weitere Hinweise auf dunkle Materie!}
						\end{itemize}
				\end{enumerate}
		\end{itemize}
\end{itemize}
\subsection{Galaxienhaufen}
\begin{itemize}
	\item[] $\gtrsim \SI{50}{\text{Mitglieder}}, \gtrsim\SI{1.5}{h^{-1}M\pc}$
	\item dynamische Zeitskala (Zeit, die deine typische Galaxie benötigt, um den Haufen einmal zu durchqueren):
		\begin{equation*}
			t_\text{cross}\approx\underset{\underset{\begin{minipage}{3cm} $\sigma_v=\SI{1000}{\frac{\km}{\s}}$ 1D Geschwindigkeitsdispersion\end{minipage}}{\uparrow}}{\frac{\SI{1.5}{h^{-1}M\pc}}{\sigma_v}}\approx\SI{1.5e9}{h^{-1}\year} << t_0=\SI{14}{G\year}
		\end{equation*}
	\item[$\Rightarrow$] gravitativ gebundenes System $\Rightarrow$ Massenabschätzung möglich, da viriales Gleichgewicht vorhanden
	\item Virialtheorem (s. Übung 1, Aufgabe 1):
		\begin{equation*}
			2\bar{T}+\bar{V}=0 (\ast)
		\end{equation*}
		wobei $T=\frac{1}{2}\sum m_iv_i^2, V=-\frac{1}{2}\sum\limits_{i\neq j} \frac{G m_im_j}{r_{ij}}$
	\item Gesamtmasse $M=\sum\limits_i m_i$, massengewichtete Geschwindigkeitsdispersion $\left\langle v^2\right\rangle := \frac{1}{M}\cdot\sum\limits_i m_i v_i^2$, gravitativer Radius: $r_G:=2M^2\left(\sum\limits_{i\neq j}\frac{m_im_j}{r_{ij}}\right)^{-1}$
		\begin{align*}
			\Rightarrow T&=\frac{M}{2}\left\langle v^2\right\rangle\\
			\Rightarrow V&=-\frac{GM^2}{r_G}\\
			\Rightarrow M&\overset{(\ast)}{=}\frac{r_G\cdot\left\langle v^2\right\rangle}{G}\underset{\underset{\begin{minipage}{1.5cm}\begin{align*} \left\langle v^2\right\rangle &= 3\sigma_v^2\\ r_G&=\frac{\pi}{2}R_G\\=\frac{\pi}{2}2M^2&\cdot\left(\sum\limits_{i\neq j}\frac{m_im_j}{R_{ij}}\right)^{-1}\end{align*}\end{minipage}}{\uparrow}}{\approx}\SI{1.1e15}{m_\odot\cdot\left(\frac{\sigma_v}{\SI{1000}{\frac{\km}{\s}}}\right)^2\cdot\left(\frac{R_G}{\SI{1}{M\pc}}\right)}
		\end{align*}
		mit $R_{ij}=$ projezierter Abstand zwischen Galaxien $i$ und $j$
	\item[$\Rightarrow$] $M\sim 10^{15}\si{M_\odot}$ für massenreiche Galaxienhaufen und wiederum:
		\begin{equation*}
			\frac{M}{L}\sim \SI{300}{h\left(\frac{M_\odot}{L_\odot}\right)} \text{ Masse-Leuchtkraft-Verhältnis}
		\end{equation*}
		übersteigt das $\frac{M}{L}$-Verhältnis von Frühtyp-Galaxien um mindestens einen Faktor $\num{10}\Rightarrow$ missing mass problem (Fritz Zwicky, 1933, Coma-Haufen)
	\item[] \textbf{Die in Galaxien sichtbaren Sterne machen weniger als etwa $\SI{5}{\%}$ der Gesamtmasse von Galaxienhaufen aus.}
\end{itemize}
\subsection{Röntgenstrahlung von Galaxienhaufen}
\begin{itemize}
	\item Röntgenstrahlung stammt aus einem heißen, diffus verteilgten Gas (intra-cluster Gas): Bremsstrahlung + Linien-Emission (Ly$\alpha$ etc.)
	\item Aus dem radialen Verlauf von Dichte und Temperatur des Gases lässt sich das Massenprofil $M(r)$ bestimmen
	\item[$\Rightarrow$] $\left[\begin{aligned} \text{Masse von Galaxienhaufen:}\\ \sim\SI{3}{\%} \text{ direkt beobachtbare Sterne in Galaxien}\\ \sim\SI{15}{\%} \text{ intergalaktisches Gas}\\ \sim\SI{80}{\%}\text{ "{}dunkle Materie"{}}\end{aligned}\right]$
	\item Masse-zu-Leuchtkraftverhältnis $\frac{M}{L}$ als Funktion der Längenskala kosmischer Objekte:
		\begin{figure}[H]
			\centering
			\begin{minipage}[l]{0.7\textwidth}
				\begin{tikzpicture}[>=stealth]
					\draw[->] (0,0)--(1,0)++(0,-0.1)node[below]{$\SI{1}{k\pc}$}--++(0,0.2)++(0,-0.1)--++(1,0)++(0,-0.1)--++(0,0.2)++(0,-0.1)--++(1,0)++(0,-0.1)node[below]{$\SI{100}{k\pc}$}--++(0,0.2)++(0,-0.1)--++(1,0)++(0,-0.1)--++(0,0.2)++(0,-0.1)--++(1,0)++(0,-0.1)node[below]{$\SI{10}{M\pc}$}--++(0,0.2)++(0,-0.1)--(6,0)node[below right]{Längenskala};
					\draw[->] (0,0)--++(0,0.5)++(-0.1,0)node[left]{$\num{1}$}--++(0.2,0)++(-0.1,0)--++(0,0.5)++(-0.1,0)node[left]{$\num{10}$}--++(0.2,0)++(-0.1,0)--++(0,0.5)++(-0.1,0)node[left]{$\num{100}$}--++(0.2,0)++(-0.1,0)--++(0,0.5)++(-0.1,0)node[left]{$\num{1000}$}--++(0.2,0)++(-0.1,0)--++(0,2)node[above left]{$\frac{\frac{M}{L}}{\frac{M_\odot}{L_\odot}}$};
					\xdef\todraw{(6,0)--++(0,0.5)}
					\foreach \x in {0.001,0.01,0.1,1}{
						\xdef\todraw{\todraw++(-0.1,0)--++(0.2,0)node[right]{$\num{\x}$}++(-0.1,0)--++(0,0.5)};
					}
					\draw[->] \todraw--++(0,1.5)node[above right]{$\Omega_m$};
					\draw[densely dashed] (0,2)--(6,2)node[midway,above]{Universum geschlossen};
					\draw (0.5,0.5)ellipse(0.15cm and 0.35cm)coordinate(s)(1.1,0.7)ellipse(0.15cm and 0.3cm)coordinate(h)(1.3,1)ellipse(0.4cm and 0.15cm)coordinate(p)(3,1.5)ellipse(0.5cm and 0.3cm)coordinate(G)(4.5,1.6)ellipse(0.6cm and 0.3cm)coordinate(H);
					\node[name=se,draw,shape=circle] at ([xshift=-1cm,yshift=-2cm]s) {\begin{minipage}{1cm}Sa,Sb\\ Sc,Irr\end{minipage}};
					\draw[->] (se) .. controls +(0.5,1) and +(0.5,-0.5) .. (s);
					\node[right] at ([xshift=1mm, yshift=-1mm]h){Halos};
					\node at ([yshift=4mm]p){Paare};
					\node at ([yshift=-5mm]G){Gruppen};
					\node at ([yshift=-5mm,xshift=1mm]H){Haufen};
				\end{tikzpicture}
			\end{minipage}
			\begin{minipage}[r]{0.28\textwidth}
				\begin{itemize}[label={$\Rightarrow$}]
					\item $\frac{M}{L}$ von kosmischen Quellen steigt an mit $r\to\infty$, aber scheint bei $\gtrsim\SI{100}{M\pc}$ konstant zu werden
					\item $\Omega_m\sim\num{0.2}$ aus dieser Methode (aber hohe Ungenauigkeit)
				\end{itemize}
			\end{minipage}
		\end{figure}
\end{itemize}
\subsection{Entstehung von Inhomogenitäten}
\subsubsection{Mögliche Ursachen}
\begin{itemize}
	\item auf kleinen Skalen ist das Universum inhomogen, z.B. ein massereicher Galaxiehaufen mit $\varnothing=\SI{1.5}{h^{-1}M\pc}$ enthält mehr als $\num{200}$ mal so viel Masse wie eine mittlere Kugel der gleichen Größe im Universum
	\item \textbf{Idee}: anfängliche Dichtefluktuation
		\begin{itemize}
			\item Dichte wächst lokal
			\item zusächtliches Gravitationsfeld
			\item Kosmologische Expansion wird lokal abgebremst
			\item Dichtekontrast wächst an
			\item Dichte wächst lokal
			\item $\cdots$
			\item \textbf{gravitative Instabilitäten}!
		\end{itemize}
	\item \textbf{Problem}: Um die heutigen Strukturen (Galaxienhaufen, -gruppen, etc.) zu erklären, müssten die CMB-Flutkuationen von der Größenordnung $\frac{\Delta T}{T}\sim 10^{-3}$ ($\lightning$ zur Beobachtung $\frac{\Delta T}{T}\sim 10^{-5}$!)
	\item \textbf{Mögliche Lösung}: Dunkle Materie dominiert, ihr (postuliert) größerer Dichtekontrast führt zur Strukturbildung
		\begin{enumerate}[label={(\alph*)}]
			\item heiße dunkle Materie = relativistische Teilchen kann ausgeschlossen werden, da $\lightning$ in den Beobachtungen (kleinere Strukturen werden durch das freie Strömen der relativistischen Teilchen ausgewaschen)
			\item kalte dunkle Materie = nicht-relativistisch (eventuell mit einer kleinen heißen Komponente, wie z.B. kosmologische Neutrinos) $\Rightarrow$ scheint sehr gut zu funktionieren
		\end{enumerate}
\end{itemize}
\subsubsection{Berechnung der Dichtefluktuationen}
\begin{itemize}
	\item relativer Dichtekontrast:
		\begin{equation*}
			\delta(\vec{r},t)=\frac{\rho(\vec{r},t)-\bar{\rho}(t)}{\bar{\rho}(t)}
		\end{equation*}
		mit $\bar{\rho}(t)$: mittlere komische Materiedichte zur Zeit $t$
	\item Da $\frac{\Delta T}{T}\sim 10^{-5}$ zur Zeit der Rekombination bei $z\approx\num{1100}$ sollte $\delta << 1$ für $z\to\infty$
	\item heute $\delta\sim 1$ (auf $\sim\SI{10}{M\pc}$) bzw. $\delta >> 1$ (auf $\sim\SI{2}{M\pc}$)
	\item Idee (s.o.): Dort wo die Dichte größer als im Mittel ist, d.h. $\delta>0$ ist das Gravitationsfeld größer $\Rightarrow$ langsamere Expansion
		\begin{itemize}
			\item $\delta$ steight weiter $\Rightarrow$ Instabilitäten wachsen mit der Zeit
		\end{itemize}
	\item Vereinfachtes Modell für kleines $\delta$:
		\begin{itemize}[label={$\cdot$}]
			\item Radius der Struktur $R<<$ Hubble-Radius $R_H=\frac{c}{H_0}=\SI{3000}{h^{-1}M\pc}$
			\item Bewegungen nicht-relativistisch
			\item nur Staubteilchen, durch die Flüssigkeitsnäherung (Kontinuum)
			\item[$\Rightarrow$] Newtonsche Mechanik eines Fluids der Dichte $\rho(\vec{r},t)$ mit Geschwindigkeitsfeld $\vec{v}(\vec{r},t)$
		\end{itemize}
	\item[$\Rightarrow$] Bewegungsgleichungen:
		\begin{enumerate}[label={(\arabic*)}]
			\item Kontinuitätsgleichung $\frac{\partial\rho}{\partial t}+\nabla\cdot(\rho\vec{v})=0$
			\item Euler-Gleichung $\underset{\begin{minipage}{3cm}\begin{tiny}Zeitliche Ableitung von $\vec{v}$, die von einem mitbewegten Beobachter gemessen wird\end{tiny}\end{minipage}}{\underbrace{\frac{\partial\vec{v}}{\partial t}+(\vec{v}\cdot\vec{\nabla})\vec{v}}}=-\frac{\nabla P\tikz[remember picture]{\coordinate (Druck) at (0,0);}}{\rho}-\nabla\Phi$\\
				mit Druck $P$ (null, denn wir betrachten Staub) und Gravitationsfeld $\Phi$
		\end{enumerate}
\end{itemize}

\end{document}
