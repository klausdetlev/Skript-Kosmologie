\subsection{Erfolge und Probleme des Standardmodells}
\begin{itemize}
	\item Standardmodell des Friedmann-Lema\^itre-Universums hat viele beeindruckende Erfolge vorzuweisen
	\item Vorhersagen:
		\begin{itemize}[label={-}]
			\item Hubblesches Gesetz: Absorption von Strahlung von Quellen mit Rotverschiebung $z$ erfolgt nur bei $z'<z$\\
				(experimentell bisher kein Gegenbeispiel gefunden)
			\item Wenig prozessiertes (d.h. metallarmes) Gas hat einen Heliumanteil von $\SI{25}{\%}$ (passt hervorragend zu den Beobachtungen, vgl. (IV) Abschnitt 4.1)
			\item $\exists$ Mikrowellenhintergrund
			\item Es sagt die richtige Anzahl von Neutrino-Familien vorher ($N_\nu=3$), wie durch den Zerfall des Z-Boson (CERN) bestätigt wurde
				\begin{equation*}
					t\propto\frac{1}{\sqrt{\rho}}\text{ im strahlungsdominierten Universum}
				\end{equation*}
				Falls $N_\nu>3 \Rightarrow$ Expansion verläuft schneller
				\begin{itemize}[label={$\Rightarrow$}]
					\item weniger Zeit bis zum Abkühlen auf $T_D$
					\item mehr freie Neutronen
					\item höherer \isotope{4}{He}-Gehalt
				\end{itemize}
			\item Neutrinomassen sind $\lesssim\SI{1}{\eV}$ (inzwischen $\underset{\lesssim\SI{1.1}{\eV}\text{ KATRIN}}{\lesssim\SI{0.1}{\eV}}$)
		\end{itemize}
	\item Nicht erklärt:
		\begin{itemize}[label={\textbullet}]
			\item Anfangswerte bei $t\sim\SI{1}{\s}$
			\item Homogenität und Isotropie
		\end{itemize}
		Welche physikalischen Prozesse liegen dahinter?
	\item Insbesondere zwei Probleme des Standardmodells:
		\begin{enumerate}[label={(\arabic*)}]
			\item \textbf{Horizontproblem}
				\begin{itemize}[label={$\to$}]
					\item Im Zeitintervall $dt$ legt das Licht die Strecke $dr=cdt$ zurück
						\begin{itemize}[label={$\Rightarrow$}]
							\item mitbewegtes Längenintervall $dx=\frac{cdt}{a(t)}$
								\begin{align*}
									\Rightarrow\ \underset{\underset{\underset{\underset{\underset{\text{bis zur Rotverschiebung } z}{\text{Entfernung von Urknall}}}{\text{Licht zurückgelegte}}}{\text{mitbewegte vom}}}{\uparrow}}{r_H(z)}&=\int\limits_0^t\frac{cdt'}{a(t')}=\int\limits_0^{(1+z)^{-1}}\frac{cda}{a^2 H(a)},\quad dt=\frac{da}{\dot{a}}=\frac{da}{a\cdot H}\\
									\Rightarrow r_H&=\begin{cases} \frac{2c}{H_0}\cdot\sqrt{\frac{1}{(1+z)\cdot\Omega_m}} & \text{für }0<<z<<z_\text{eq}\\ \frac{c}{H_0}\cdot\frac{1}{\sqrt{\Omega_r}(1+z)} & \text{für }z_\text{eq}<<z\end{cases}
								\end{align*}
						\end{itemize}
					\item Zum Zeitpunkt der Rekombination $z\approx z_\text{eq}\sim 1000$\\
						Eigenlänge $r_\text{eq}=a\cdot r_H=2\frac{c}{H_0}\frac{1}{(1+z_\text{eq})^{\frac{3}{2}}\sqrt{\Omega_m}}$\\
						$\hat{=}$ Winkel am Himmel: $\Omega_{H,rec}=\SI{1}{\degree}\cdot\left(\frac{\Omega_m}{\num{0.3}}\right)^{\frac{1}{2}}\cdot\left(\frac{z_{rec}}{1000}\right)^{-\frac{1}{2}}$
						\begin{itemize}[label={$\Rightarrow$}]
							\item CMB ist bis auf kleine Anisotropien der relativen Amplitude $\sim 10^{-5}$ isotrop $\lightning$ (?)\\
								$\to$ Universum isotrop und homogen?
						\end{itemize}
				\end{itemize}
			\item \textbf{Krümmung}
				\begin{itemize}[label={$\to$}]
					\item totaler Dichteparameter für eine beliebige Rotverschiebung
						\begin{equation*}
							\Omega_0(z)=\frac{\rho_m(z)+\rho_r(z)+\rho_v}{\rho_c(z)}
						\end{equation*}
						mit kritischer Dichte $\rho_c(z)=\frac{3H^2(z)}{8\pi G}$
						\begin{equation*}
							\Rightarrow \Omega_0(z)=\left(\frac{H_0}{H}\right)^2\left(\frac{\Omega_m}{a^3}+\frac{\Omega_r}{a^4}+\Omega_\Lambda\right)
						\end{equation*}
						und mit den Lema\^itre-Friedmann-Gleichungen:
						\begin{equation*}
							1-\Omega_0(z)=F(z)(1-\Omega(0))
						\end{equation*}
						mit $F(z)=\left(\frac{H_0}{a\cdot H(a)}\right)^2 >0$\\
						$\Omega_0$: totaler Dichteparameter heute:
						\begin{itemize}[label={\textbullet}]
							\item falls $\Omega(0)=1 \Rightarrow \Omega_0(z)=1 \forall z$
							\item falls $\Omega(0)<1 \Rightarrow \Omega_0(z)>1$ bzw. $\Omega(0)<1\Rightarrow \Omega(z)<1$ da $\Omega(z)-1\sim $ Krümmung $K\Rightarrow\forall z$ bleibt $K$ erhalten.\\
								$F(z)$ charakterisiert die Abweichung von einem flachen Universum
						\end{itemize}
					\item Beispiel: für strahlungsdominiertes Universum
						\begin{equation*}
							F(z)\approx\left[\Omega_r\cdot(1+z)^2\right]^{-1}\underset{\underset{\begin{minipage}{3cm}\begin{tiny} bei $z\sim 10^{10}$ (Ausfrieren der Neutrinos)\end{tiny}\end{minipage}}{\uparrow}}{\sim} 10^{-15}
						\end{equation*}
					\item Aus Beobachtungen (CMB) wissen wir, dass:
						\begin{equation*}
							\num{0.97}<\Omega(0)<\num{1.04}\Rightarrow \left|\Omega(0)-1\right|\lesssim\num{0.04}\Rightarrow\left|\Omega_0(10^10)-1\right|\lesssim 10^{-15}
						\end{equation*}
						\begin{itemize}[label={$\Rightarrow$}]
							\item Flachheitsproblem: Damit der totale Dichteparameter heute von der Größenordnung 1 sein kann, muss er zu sehr frühen Zeiten extrem nahe bei 1 gewesen sein!
						\end{itemize}
					\item Wie war eine solch präzise Feinabstimmung dieser Größe möglich? (sehr spezielle Anfangsbedingungen bei $t=\SI{1}{\s}$) $\to$ antropisches Prinzip? $\to$ unbefriedigend
						\begin{itemize}[label={$\Rightarrow$}]
							\item (spekulative) Erweiterung des Standardmodells: \textbf{Inflation} (A. Guth, 1980)
								\begin{itemize}[label={$\to$}]
									\item Teilchenphysik erwartet neue Phänomene (GUT = grand unified theories) bei $T\sim 10^{14} \si{G\eV}\hat{=}t\sim 10^{-34}\si{\s}$
									\item Szenario der Inflation: $\Omega_\Lambda$ bei sehr frühen Zeiten viel größer als heute
										\begin{align*}
											\Rightarrow \frac{\dot{a}}{a}&\approx\sqrt{\frac{\Lambda}{3}} \Rightarrow \text{ exponentielle Expansion des Universums}\\
											\Leftrightarrow a(t)\propto e^{\sqrt{\frac{\Lambda}{3}}\cdot t}
										\end{align*}
									\item Annahme: Nach einer Phase der Expansion kommt es zu einem Phasenübergang, bei dem die Vakuumenergiedichte in normale Materie und Strahlung umgewandelt wirdq
										\begin{align*}
											\text{zu (1): } r_M(a_1,a_2)&\sim \Omega_\Lambda^{-\frac{1}{2}}\int\limits_{a_1}^{a_2}\frac{cda}{a}>>1 & &\text{falls } a_1<<1
										\end{align*}
										$\Rightarrow$ Das gesamte beobachtbare Universum war in Kausalem Kontakt $\Rightarrow$ Homogenität und Isotropie.\\
									\item[] zu (2):
									\item Durch die gewaltige Ausdehnung wird jede ursprüngliche Krümmung "weggeglättet":
										\begin{equation*}
											H(t)=\sqrt{\frac{\Lambda}{3}}\Rightarrow\Omega_\Lambda=\frac{\Lambda}{3H^2}=1\Rightarrow \Omega_0(1)
										\end{equation*}
										$\Rightarrow$ Das Universum ist flach und auch heute gilt noch sehr genau $\Omega_0=1$
										\begin{figure}[H]
											\centering
											\begin{tikzpicture}
												\draw (0,0)circle(1cm);
												\draw[->] (1.5,0)--(2.5,0);
												\draw (3.5,-1)arc(-90:90:1cm);
												\draw[->] (5,0)--(6,0);
												\draw ({5*cos(-10)+2},{5*sin(-10)})arc(-10:10:5cm);
											\end{tikzpicture}
										\end{figure}
									\item Weiterhin bietet Inflation eine Erklärung für den Ursprung der Dichteschwankungen im Universum (Keime der Strukturbildung): Quantenfluktuationen (Quantengravitation)
										\begin{figure}[H]
											\centering
											\begin{tikzpicture}[]
												\draw[->] (0,-0.5)--(0,5.5)node[above left]{$r_H/\si{\cm}$};
												\foreach \x \y in {1/10^{-60},2/10^{-40},3/10^{-20},4/1,5/10^{20}}{
													\draw (-0.1,\x)node[left]{$\y$}--(0.1,\x);
												};
												\draw[->] (-0.5,0)--(6.5,0)node[below right]{$t/\si{\s}$};
												\foreach \x \y in {1/10^{-40},2/10^{-30},3/10^{-20},4/10^{-10},5/1,6/\cdots}{
													\draw (\x,-0.1)node[below]{$\y$}--(\x,0.1);
												}
												\draw[thick,domain=1:3,samples=50] plot({\x},{0.1*\x+2.5+1.5*tanh(5*(\x-1.7))});
												\draw[thick,domain=3:6,samples=2] plot({\x},{0.1*\x+4});
												\draw[thick,domain=0:1,samples=2] plot({\x},{0.1*\x+1});
												\draw[thick,domain=0:6,samples=50,blue!50!black] plot({\x},{0.1*\x+4});
												\draw[densely dashed] (1.5,0)--(1.5,5.5)(2,5.5)--(2,0);
												\fill[pattern=north west lines] (1.55,0)--(1.55,5.5)--(1.95,5.5)--(1.95,0)--cycle;
												\node[name=s] at (4,5){Standardmodell};
												\node[name=i] at (4,2){Inflationstheorie};
												\draw[->,shorten >= 3pt, >=stealth,very thick] (i)--(1.7,2.5);
												\draw[->,shorten >= 3pt, >=stealth, very thick,blue!50!black] (s)--(1,4.1);
												\draw[->,shorten >= 3pt, >=stealth, very thick,blue!50!black] (s)--(5,4.5);
											\end{tikzpicture}\\
											Ausdehung während Inflation: Faktor $\sim 10^{40}$ von $ct_i\sim 10^{-24}\si{\cm}$ auf $ct_s=10^{18}\si{\cm}$ von $t_i\sim 10^{-34}\si{\s}$ auf $t_s\sim 10^{-32}\si{\s}$ weitere "normale" kosmische Expansion: Faktor $10^{25}$ auf $10^{41}\si{\cm}$
										\end{figure}
								\end{itemize}
						\end{itemize}
				\end{itemize}
		\end{enumerate}
\end{itemize}
